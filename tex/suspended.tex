%; whizzy section

%% Leave the above line for didier
%% No macros before \documentclass

\documentclass[acmsmall,screen,nonacm,review]{acmart}

%% \newcommand{\Comments}{\False}
\newcommand{\Draft}{\False}
\newcommand{\Long}{\True}
\newcommand{\Extended}{\False}
\newcommand{\Beamer}{\False}
\newcommand{\Anonymous}{\False}

\newcommand{\acmart}{\True}
\usepackage{suspended}

%% \Xfirstname defined in {mycomments}
%% Use either
%%   \Xfistname[text to comment]{your comment on the text}
%% or
%%   \Xfirstname{free comment}
%% Uncomment this line to hide all comments.
% \UNXXX{}

\usepackage{marginnote}

\title{Omnidirectional type inference for \ML: principality any way}

\begin{document}

\begin{abstract}
We propose a new concept of \emph{omnidirectional} type inference, which is
the ability to resolve \ML-style typing constraints in disorder, by
contrast with all known implementations that always typecheck the
bindings before the bodies of let-expressions.
%
This relies on two technical devices: \emph{partial type schemes}
and \emph{suspended match constraints}. Partial here means the
possibility of taking instances of a type scheme that is not yet
completely solved, with a mechanism to update their instances when
the type scheme is refined, incrementally.
\emph{Suspended} match constraints are a mechanism to delay the resolution of
some constraints, and to discharge them later on when some type
variables have been instantiated.
%
The benefits of omnidirectional type inference are striking for several
advanced \ML extensions, typically those that rely on optional type
annotations for which the principal type property is often fragile.  We
illustrate them with \OCaml's static overloading of record labels and
constructors, semi-explicit first-class polymorphism, and tuple projections
\`a la \SML.
\end{abstract}
\maketitle

\section{Introduction}
\label{sec/introduction}

\parcomment {Introduction (ML, principality)}

The Damas-Hindley-Milner (\HM) \cite{Damas-Milner/W@popl82} type system has
long occupied a sweet spot in the design space of strongly typed programming
languages, as it enjoys the \emph{principal type property}: every well-typed
expression $\e$ has a most general type $\ts$ from which all other valid
types for $\e$ are instances of $\ts$. For example, the identity function
$\efun \x \x$ has the principal type $\tfor \tv \tv \to \tv$, generalizing
types like $\tint \to \tint$ and $\tbool \to \tbool$.

\parcomment {Benefits of principality}

This property ensures predictable and efficient inference. Local typing
decisions are always optimal, yielding most general types without guessing or
backtracking. As a result, inference of subexpressions can proceed in any
order, and well-typedness is preserved under common program transformations
such as let-contraction, let-expansion, and argument reordering.

\parcomment {Extensions often break pincipality}

Over the years, many extensions of \ML have been proposed. Some of
them, such as extensible records with row-polymorphism, higher-kinded
types, or dimensional types, fit perfectly into the \ML
framework. Others such as GADTs, higher-rank polymorphism, or static
overloading, are \emph{fragile}, as they sometimes require explicit
type annotations. The return type of overloaded data constructors can
be annotated; polymorphic expressions can be annotated with a type
scheme; for GADTs, the type of the \texttt{match} scrutinee and return
type can be annotated to a rigid type which will be refined by type
equalities in each branch. Those type annotations may sometimes but
not always be omitted.

Consider impredicative higher-rank polymorphism,
\MLF~\cite{LeBotlan-Remy/recasting-mlf} for instance:
\begin{program}[input]
  let self f = f f
\end{program}
With higher-rank types, one could \emph{guess} the type of \code{f} to be
either $\tfor \tv \tv \to \tv$ or $\tfor \tv \tv \to \tv \to \tv$ in order
to typecheck \code{self}---neither of which is more general than the other,
violating principality.

\parcomment {Current approaches are kinda bad}

To fix this, inference algorithms require a minimal amount of
\emph{known} type information to restore principality; in this example
the binding of \code{f} should be annotated with a polymorphic type
scheme. Yet specifying such requirements declaratively is
difficult. As a result, the specifications are often twisted with some
direct or indirect algorithmic flavor in order to preserve
principality and completeness.
%
Moreover, these (more or less) ad-hoc restrictions commonly reject examples
whose type could easily be guessed. For instance, \MLF~ rejects:
\begin{program}[input]
  let self' f = if true then f f else (f : $\forall$'a. 'a -> 'a)
\end{program}

\parcomment {Annotations fixes all the issues (but they're problematic)}



To each fragile construct
corresponds a robust construct where the type annotation is mandatory. The
robust constructs fit perfectly into the \ML framework, but are
significantly more cumbersome to use, as they always require explicit type
annotations. Fragile constructs can be defined by elaboration into their
robust counterpart.
%
% The elaboration determines which annotations can be
% omitted and rebuilt from context, a point of view already taken by~\citet
% {Pottier-Regis-Gianas/stratified@popl06} in their work on stratified type
% inference.
%
The difficulty lies in finding a specification
that is sufficiently expressive, principled, intuitive for the user,
and for which we have a complete and effective elaboration algorithm.

The solutions proposed so far all enforce some ordering in which type
inference is performed, which can then be used to propagate both inferred
types and user-provided type annotations as \emph{known} types that can be
used for disambiguation and enable the omission of some annotations.

\paragraph{Bidirectional type inference}

Bidirectional type inference is a standard alternative to unification for
propagating type information. It can be presented by adding to the type
inference algorithm an optional expected type in addition to the expression
to be typechecked and the context in which it should be typechecked. Type
inference is said in \emph{checking mode} when the expected type is present,
and in (the usual) \emph{inference mode} when it is absent.

For example the type-system designer can decide to type-check function
applications $\eapp \ea \eb$ by \emph{inferring} that $\ea$ has some
function type $\t \tarrow \tp$, and then \emph{checking} $\eb$ against
$\t$. This is not the only possible choice: bidirectional type
inference is actually a framework that must be instantiated by
a particular choice of modes for each of the language construct. There
usually is no optimal combination of modes for the whole set of
language constructs. Once modes have been fixed, there are usually
principal solutions to the type inference problem, but with respect to
a specification that made non-principal choices.

Bidirectional type inference has been largely used for languages with
higher-rank polymorphism, dependent types, or subtyping.  Still, both \OCaml
and \Haskell only use a limited form of bidirectional type checking with an
underlying first-order-unification based type inference engine, which limits
the downsides of bidirectional type checking.

\paragraph{\Geninst-directional type inference}

\ML languages enforce an ordering in the typechecking of let-bindings
$\elet \x \ea \eb$, where (the polymorphic part of) the definition
$\ea$ must be typed before the body $\eb$. \OCaml relies on this
ordering to propagate type information in a principal way: if an
expression has a polymorphic type, then this type can be used for
type-directed disambiguation, whereas monomorphic type information may
depend on inference order in a fragile way. We call this
\textbf{\geninst}-directional (to be read \textbf{pi}-directional)
type inference, to mean that \textbf{p}olymorphic expressions must be
typed before their \textbf{i}nstances.

This mechanism was introduced for semi-explicit first-class
polymorphism by~\citet {Garrigue-Remy/poly-ml}, and later used
by~\citet {LeBotlan-Remy/recasting-mlf} for empowering \MLF. It has
also been used for overloading of record fields and variant
constructors in \OCaml, which we use for illustration here as it is
simpler to explain.\footnote {Semi-explicit polymorphism will be
  presented in~\cref{sec/constraints/polytypes}.}

The user may define two nominal record types with overlapping fields:
\begin{program}[input]
type 'a one = {x : 'a}
type two = {x : int; y : int}
\end{program}
In \OCaml, both definitions are visible and the compiler must
statically disambiguate field usage. It knows that \lstinline!{x = 1}! can
only be a \code{one}, and that \code{r.y} can only be a \code{two},
but field accesses \code{r.x} must be disambiguated. Consider for
example:
\begin{program}[input]
let e_1 r = r.x
let e_2 = let r = {x = 1} in r.x
let e_3 = (fun r -> r.x) {x = 1}
\end{program}
To be able to disambiguate the projection \code{r.x}, the type of the
\code{r} should be \emph{known} to be either (an instance of)
\ocaml{'a one} or \ocaml{two}. The definition \ocaml[indices]{e_1} is
clearly ambiguous since there is no clue on the type of \ocaml{r} and
its typechecking fails.\footnote {In fact, \OCaml uses a default
  resolution strategy instead of failing when the type is ambiguous,
  which is to emit a warning use the last definition in scope. To
  check these examples, you should use the options \code{-principal -w
    +41}, which enforce principality checks and enables the warning on
  default resolution.}
By contrast, the only possible type for \ocaml{r} is \ocaml {'a one}
in both \ocaml{e_2} and \ocaml {e_3} -- . Still, there is a difference. Indeed, the type
of~\ocaml{e_2} is considered to be unambiguous, while the type of
\ocaml{e_3} is ambiguous.
%
To understand why \ocaml{e_3} fails, consider the equivalent versions of
\ocaml{e_3} where \texttt{@@} and \ocaml{|>} are the application and the
reverse application functions.
\begin{program}[input,escapechar={}]
let e_3_2 = (fun r -> r.x) @@ {x = 1}
let e_3_3 = {x = 1} |> (fun r -> r.x)
\end{program}
\OCaml does not make any difference between \ocaml[indices]{e_3},
\ocaml[indices]{e_3_2}, or \ocaml[indices]{e_3_3} and consider all
subexpressions in an application, including the function and all
arguments as being inferred simultaneously, until they are
let-bound. More precisely, polymorphic types\footnote{\Xgabriel{I
    propose to remove this footnote which I find too technical and
    I don't think is necessary or helpful to follow the
    introduction.} Technically, all type nodes are annotated with
  a special variable $\av$ called an annotation variable, so that we
  may distinguish between the polymorphic binding
  $\xa : \all \av \t^\av$ that binds $r$ to the known (raw) type $\t$
  and the monomorphic binding $\xb : \t^\av$ that binds $\xa$ to the
  unknown (raw) type $\t$.} are considered to be known while
monomorphic types are consider to be unknown---or, rather,
not-yet-known. This criterion also warns on the following example
where \code{r} has a monomorphic type.
\begin{program}[input,escapechar={}]
let f p r = if p then r.x else (r : two).x
\end{program}
Warning here is preferable to working or not depending on
the order of inference of \code{if} branches.

\paragraph{Limitations of directional type inference}

Directional type inference has been in used in different languages and
somehow proven to work in practice. The main downside of \emph{bidirectional} type inference is
that it requires arbitrary choices in the specification of the flow of
information, typically in applications: should the function be typed first
and its codomain be used to improve the typing of its argument or, on the
opposite, should the argument be typed first and be used as the type of the
codomain of the function?  There are examples when the former is a better
choice and others when the latter is preferable---but typing rules have to
choose one of the two alternative forever.  One may also allow information
to flow between multiple arguments passed to the same function, and several orders are possible. We eventually
have a complete algorithm with respect to a specification that made somewhat
arbitrary choices.

On the other hand, \emph{\Geninst-directional} type inference does not even
allow to propagate user-provided type annotations from a function to its
argument! For example, the following would be rejected as ambiguous with
\geninst-directional type inference alone:
\begin{program}[input]
let g : ('a one -> int) -> int = fun f -> f {x = 1} in g (fun r -> r.x)
\end{program}

\OCaml uses \geninst-directional type inference as the
primary mechanism, but it also uses a weak form of bidirectional propagation: in this example the type of the argument of \code{g} is known \Geninst-directionally, but \OCaml then propagates this expected type within the function definition in bi-directional fashion, so that this example may be considered non-ambiguous.

Besides, the implementation of \geninst-directional type inference has an
algorithmic cost: for technical reasons, variable annotations must un-share
types (from acyclic graphs as naturally produced by unification to trees),
which may increase the size of types and the cost of type inference. For
that reason, the implementation of \OCaml cheats and is incomplete by
default. The user must explicitly pass the \texttt{-principal} flag to
require the more expensive computation of principal types when desired.

\paragraph{The relative completeness of directional type inference}

Both bidirectional and \geninst-directional type inference rely on an
ordering for type propagation that is partially specified, explicitly or
implicitly, to take advantage of user-provided type annotations and
already-inferred types to alleviated the need for extra annotations.
%
While they come with \emph{complete} algorithms, this is with respect to
their specifications, which include some choices that are subjective and may
sometimes look arbitrary.

\locallabelreset

Indeed, they all reject examples as ambiguous when sometimes there
would be a unique well-typed solution. 
%
Let us illustrate these situations with a few more examples:
\begin{program}[error]
let e_6 r = (r.x, r.y)
let e_7 r = let x = r.x in x + r.y
let e_8 = let getx r = r.x in getx {x = 1}
let e_9 r = (r.x : bool)
let e$_{10}$ r = r.x.x
\end{program}
All these examples are ambiguous for \OCaml\footnote{Its succeeds on \ocaml{e_6},
using the default choice, but it fails if we reverse the order of the type
declarations.}.  However, \ocaml{r} can only be of type
\ocaml{two} in \ocaml{e_6}. Indeed, considering the second projection first,
we should learn that \ocaml{r} is of type \ocaml{two} and since it is
$\lambda$-bound, this should then make the first projection unambiguous.
The failure is a matter of solving the constraint in the right order, which
should be accepted by omnidirectional type inference.

A similar failure occurs in \ocaml{e_7}, where the type of the
$\lambda$-bound variable \code{r} is initially ambiguous and
unknown. It is only upon typing the projection \code{r.y} that
\code{r} is forced to have the type \ocaml{two}.

\OCaml also fails on \ocaml$e_8$, and omnidirectional inference
supports this example, despite the fact that the disambiguation
information flows from an instance to a polymorphic definition, opposite to the \Geninst-order. We
call this \emph{back-propagation}.

%
Our proposal with omnidirectional type inference succeeds
on \ocaml{e_6}, \ocaml{e_7}, and \ocaml{e_8}.

The example \ocaml{e_9} can be disambiguated from the return type of
the projection, rather than from its source. The typing rules for
records that we present on this work restricts disambiguation to the
source type only and rejects this example. But we believe that other
typing rules using omni-directional type inference could support this
example as well.

Finally, \ocaml{e$_{10}$} is an example where none of the field
projections has enough type information to be disambiguated, but the
constraints they impose can be combined to deduce that the type of
\ocaml{r} must be \ocaml{one}, as the \code{x} field of \code{two}
does not have a record type. This lies outside the framework of
omnidirectional type inference, in which suspended constraints must be
discharged one by one in some order, independently of other
still-suspended constraints.
%
We believe that this reasonable restriction keep type inference effective,
since the complexity of general overloading without this restriction is
NP-hard, even in the absence of let-polymorphism, as shown by an encoding of
3-SAT problem by~\citet*
{Chargueraud-Bodin-Dunfield-Riboulet/jfla2025}.

\paragraph{Omnidirectional type inference}

In absence of polymorphism, type inference is solely based on
unification constraints which can be solved in any order;
omnidirectional inference is then natural and easy to implement.  The
difficulty originates from \ML let-polymorphism for which all known
implementations choose to always infer the type of a let-binding
first, to then turn it into a type scheme that is assigned to the
let-bound variable to extend the typing environment in which the body
is finally typed. The Hindley-Milner algorithm $\mathcal{J}$, one of
its variant $\mathcal{W}$ or $\mathcal{M}$~\cite
{Lee_Yi/algoM@toplas1998}, or more flexible constraint-based type
inference
implementations~\citep*{Remy/mleth,Remy/thesis,Odersky-Sulzmann-Wehr@tpos,Pottier-Remy/emlti}
all follow this strategy, to the best of our knowledge. However, this
state of affairs is not a necessity.

To efficiently achieve omnidirectional type inference for fragile \ML
extensions:
\begin{enumerate}
\item
  we introduce \emph{suspended match constraints} as a way to suspend
  ambiguity resolution until sufficient information has been found from
  context so that they can be discharged and hopefully resolved.
\item
  we work with \emph{partial types schemes}, \ie with the ability to instantiate type
  schemes that are not yet fully determined and consequently revisit their
  instances when they are being refined, incrementally. This allows
  inferring parts of a \texttt{let}-body to disambiguate its definition,
  without duplicating constraint-solving work.
\end{enumerate}

These technical devices are introduced once and for all---in a general
framework of constraint-based type inference. Each fragile \ML construct can
then be implemented by suspended constraints that expand to its robust
counterpart once the annotation has been inferred. This generality comes at
a cost, which is that everything is hard:
\begin{itemize}
\item Implementing partial types schemes (without duplicating
  constraint-solving work) is hard.
\item Giving an adequate semantics for suspended constraints is also hard, as we
  must capture declaratively the intuition that some type information must be
  \emph{known} rather than \emph{guessed}, but also that some programs
  that are semantically non-ambiguous (there is a unique fully-annotated
  version that is well-typed) must still be rejected as ambiguous to prevent
  us from looking into suspended constraints.
\end{itemize}
In return, we found that the declarative semantics of suspended constraints
immediately suggests a systematic way to present user-facing typing rules
for each fragile construct, for which the implementation is correct and
complete.

\subsubsection* {Plan}

The rest of the paper is organized as follows.
\begin{enumerate*}[label={}]
\item
  In \cref{sec:constraints}, we give an overview of suspended constraints
  and their application to various extensions for \ML.
\item
  In \cref{sec:semantics}, we describe suspended constraints and their semantics.
\item
  In \cref{sec:language}, we define \OML, an extension of \ML featuring static
  overloading of record labels, overloaded tuple projections, and
  semi-explicit first-class polymorphism. We also present its typing rules.
  This section gives a precise definition of the constraint generation
  translation and states the theorems of soundness and completeness.
\item
  In \cref{sec:solving}, we provide a formal definition of our constraint
  solver as a series of non-deterministic rewriting rules.
\item In \cref{sec:discussion}, we discuss some extensions of suspended
  constraints that we have prototyped but whose theory is less clear.
\item
  In \cref{sec:related-work} and \cref{sec:future-work}, we discuss related
  and future work.
\end{enumerate*}
All proofs are postponed to appendices.

\subsubsection* {Our contributions}

Our contributions are
\begin{enumerate*}
\item
  A novel \emph{omnidirectional} type inference framework for
  extensions of \ML with advanced features, based on two new devices,
  suspended constraints and partial type schemes;

\item A declarative semantics of suspended constraints that captures the
  idea that they wait on information that must be propagated from the
  context, not \emph{guessed}.

  This includes, in particular, a new declarative caracterisation of
  \emph{known} type information.

\item
  A complete yet efficient constraint-solving type inference algorithm.

\item
  Three instantiation of our framework that give new declarative type
  systems and their implementation using suspended constraints for tuple
  projection in the style of \SML, static overloading of record fields and
  datatype constructors, and for semi-explicit first-class polymorphism.

\end{enumerate*}

\section{Suspended constraints: an overview}
\label{sec:constraints}

\begin{bnffig}[t]%
  {fig:constraint-syntax}%
  {Syntax of types and constraints}
\entryset[Type variables]{\tva, \tvb, \tvc}{\TyVars}{}
\\
%% \entryset[Types]{\t}{\Types}\\
\entry[Types]{\t}{
    \tv \and
    \tunit \and
    \tone \to \ttwo \color{gray} \and
    \Pi\iton \ti \and
    \tys \T \and
    \tpoly \ts
}\\
\entry[Type schemes]{\ts}{
    \t \and
    \all \tv \ts
}\\[1ex]
\entry[Constraints]{\c}{
        \ctrue
  \and  \cfalse
  \and  \ca \cand \cb
  \and  \cexists \tv \c
  \and 	\cfor \tv \c
  \and  \cunif \tone \ttwo
  \nextline
  \and  \clet \x \tv \ca \cb
  \and  \capp \x \t
  \nextline
  \and  \cmatch \t \cbrs
}\\[1ex]
\entry[Branches]{\cbr}{\cbranch \cpat \c} \\
\entry[Patterns]{\cpat}{}{} \\[1ex]
\entry[Constraint contexts]{\C}{
  \square
  \and \C \cand \c
  \and \c \cand \C
  \and \cexists \tv \C
  \and \cfor \tv \C
  \nextline
  \and \clet \x \tv \C \c
  \and \clet \x \tv \c \C
} \\
\entryset[Shapes] {\Sh, \sh} {\Shapes} {}
\end{bnffig}

\parcomment{Syntax!}

The syntax of types and constraints is given
in~\Cref{fig:constraint-syntax}. Monotypes (or just types) are as
usual including: type variables $\tv$, the unit type $\tunit$, arrow
types as usual---and also structural tuples $\Pi\iton \ti$, nominal
types $\tys \T$, and polytypes $\tpoly \ts$, in gray, as they which
will be introduced in the following subsections.  Type schemes $\ts$
are of the form $\all \tvs \t$, they are equal up to the reordering of
binders. We write $\TyVars$ the set of type variables.

Building atop the constraint-based type inference framework of
\citet{Pottier-Remy/emlti}, we adopt a constraint language that includes both
term and type variables, described in~\Cref{fig:constraint-syntax}.

%
The constraint language contains tautological ($\ctrue$) and
unsatisfiable ($\cfalse$) constraints, conjunctions
($\cone \cand \ctwo$). The constraint form $(\cexists \tv \c)$ binds an
existentially quantified type variable $\tv$ in $\c$, while the
constraint $(\cfor \tv \c)$ binds $\tv$ universally. The constraint form
$(\cunif \tone \ttwo)$ asserts that the types $\tone$ and $\ttwo$ are
equal.
%
When $\ts$ is a polymorphic type scheme $\tfor \tvs \tp$, we use the
notation $(\cleq \ts \t)$ as syntactic sugar for the instantiation
constraint $\cexists \tvs \cunif \tp \t$.

\parcomment{Constraint abstractions}

Two constructs that deal with the introduction and elimination of
constraint abstractions. A constraint abstraction $\cabs \tv \c$ can
simply be seen as a function which when applied to some type $\t$
returns $\c \where {\tv \is \t}$. Constraint abstractions are
introduced by a let-construct $(\clet \x \tv \cone \ctwo)$ which binds
the constraint abstraction to the term variable $\x$ in
$\ctwo$---additionally ensuring the abstraction is satisfiable. They
are eliminated using the application constraint $(\capp \x \t)$ which
applies the type $\t$ to the abstraction constraint bound to $x$.

\parcomment {Suspended match constraints}

Finally, we introduce \textit{suspended match constraints}\footnote
{Previously dubbed `frozen constraints' in \citep{TODO}}
$(\cmatch \t \cbrs)$. These constraints are \emph{suspended} until
until the \textit{shape} of $\t$, such as its top-level constructor,
is known. Then they are \emph{discharged}: a unique branch is selected
and its associated constraint has to be solved. A match constraint
that is never discharged is considered unsatisfiable·s.

More precisely:
\begin{enumerate}
\item
  The matchee $\t$ is a type. The constraint remains suspended until the
  shape of $\t$ is determined, that is, while $\t$ is a type variable.
\item $\cbrs$ is a list of branches of the form $\cbranch \cpat \c$,
  where $\cpat$ is a shape pattern. For example, the pattern
  $\tva \to \tvb$ matches function types, binding the argument and
  return types to $\tva$ and $\tvb$ respectively. The constraint $\c$
  is then solved in the extended context.
  %
  To ensure determinism, the set of patterns $\bar \cpat$ must be
  \emph{shape-disjoint}---that is, no shape may be matched by more
  than one pattern in the list.
\end{enumerate}

We keep the grammar of shapes and patterns abstract in this section,
to explain the general framework of suspended constraints. Shapes are
formalized in \cref{sec:semantics}. We then introduce specific shapes
and patterns for specific language features in
\cref{sec/constraint-gen}.

\parcomment {Constraint contexts}

Throughout this paper, we will find it convenient to work with
\emph{constraint contexts}. A constraint context is simply a constraint with
a ``hole'', analogous to evaluation contexts $\E$ used extensively in
operational semantics. We write $\C\where{\c}$ to denote filling the hole of
the context $\C$ with the constraint $\c$. Hole filling may capture
variables.  (Hence, we need explicit side-conditions when we mean to avoid
capture of a particular variable.)

\paragraph{Suspended constraints in action}

The remainder of this section illustrates the role of suspended constraints
in supporting \emph{fragile} language features as defined above.
These include:
\begin{enumerate}
  \item Semi-explicit first-class polymorphism;
  \item Constructor and record label overloading for nominal algebraic
  datatypes;
  \item Overloaded tuple projection in the style of \SML.
\end{enumerate}
We demonstrate how the typability of each of these features can be elaborated
into constraints, formalized using a constraint generation function of the
form $\cinfer \e \tv$, which, given a term $e$ and expected type $\tv$,
produces a constraint $\c$ which is satisfiable if and only if $\e$ is
well-typed. A formal account of the semantics of suspended constraints and
the declarative typing rules for these features is deferred to
\cref{sec:semantics} and \cref{sec:language}, respectively.

\subsection{Semi-explicit first-class polymorphism}
\label {sec/constraints/polytypes}

\parcomment {Intro (and annotations)}
Semi-explicit first-class polymorphism \citep{Garrigue-Remy/poly-ml} uses
\textit{annotated types} to track the origins of polymorphic types.
%
The type constructor $\tapoly \ts \av$ boxes a polymorphic type scheme
$\ts$, turning it into a \textit{polytype} annotated with the annotation
variable $\av$.  Once boxed, the polytype $\tapoly \ts \av$ is considered
a monotype, thereby enabling impredicative polymorphism. Annotation variables
may themselves be generalized, yielding type schemes such as
$\tfor \av {\tapoly \ts \av}$.

\parcomment {Boxing}

The introduction form for polytypes is a boxing operator $\expoly
\e {\exi \tvs \ts}$ with an explicit polytype annotation $\exi \tvs \ts$
where the $\tvs$ are type variables that are free in $\ts$.\Xgabriel{Do you mean that $\tvs$ are \emph{all} the free variables in $\ts$, so that $\exi \tvs \ts$ is closed, or is it okay if they are only a subset of the type variables bound in $\ts$?}
%
The resulting expression has type $\tapoly {\ts \where {\tvs \is \tys}} \av$
where $\av$ is an arbitrary (typically fresh) annotation variable and $\tys$
are arbritraty types that replace the free variables $\tvs$.
The annotation variable $\av$ can thus be generalized.  That is $\expoly \e
{\exi \tvs \ts}$ can also be assigned the type scheme $\all \av {\tapoly {\ts
\where {\tvs \is \tys}} \av}$.

\parcomment {Unboxing (principality restriction)}

Conversely, to instantiate a polytype expression, one must use an explicit
unboxing operator $\einst \e$, which requires no accompanying type
annotation.  However, the operator requires $e$ to have a polytype scheme of
the form $\all \av \tapoly \ts \av$ and then assigns $\einst \e$ the type
$\t$ that is an instance of $\ts$. If, by constrast, $\e$ has the type
$\tapoly \ts \av$ for some non-generalizable annotation variable $\av$, then
$\e$ is considered of a not-yet-known polytype, and therefore $\einst \e$ is
ill-typed.  It is precisely the polymorphism of $\av$ that ensures that the
polytype is indeed known and not being inferred.

\parcomment {Example ill-typed term}

For example, the expression $\efun \x {\einst \x}$ is not
typable. Indeed, the $\lambda$-bound variable $\x$ is assigned
a monotype. The only admissible type for $\x$ is $x : \tapoly \ts \av$
for some $\ts$ and $\av$.  Since $\av$ is bound in the surrounding
context at the point of typing $\einst \x$, it cannot be generalized
prior to unboxing, rendering the term ill-typed.

\parcomment {Annotations}

However, type annotations can be used to freshen annotation variables.
We usually omit annotation variables in annotations, since we can
implicitly introduce fresh ones in their place. For example,
$\efun {\x : \tapoly \ts {}} {\einst \x}$ which is syntactic sugar
for $\efun \x {\elet \x {(\x : \tapoly \ts {})} {\einst \x}}$, is
well-typed because the explicit annotation introduces a fresh
variable annotation $\ava$, which can then be generalized, yielding
$\tfor \ava {\tapoly \ts \ava}$.

This is the type-level feature that supports polymorphic methods and polymorphic record fields in \OCaml. If a record type \ocaml[mathescape=true]{$\tv$ t} has a single polymorphic field \ocaml[mathescape=true]{f : $\sigma$}, then the record expression \ocaml[mathescape=true]!{f = $e$}! desugars into \ocaml[mathescape=true]!{f = $[e : \exists \tv. \ts]$}!, and the record projection \ocaml{r.f} desugars into $\einst {\mathtt{r.f}}$.

\parcomment {Short commings of annotation variables}

The very purpose of annotation variables is to distinguish \emph{known},
polymorphic polytypes from \emph{not-yet-known}, monomorphic ones. However,
they may be sometimes be unintuitive as some information that has just been
inferred must still be considered as yet-unknown until its generalization.
They are also sensitive to the placement of type annotation, an artifact of the
fixed directionality of generalization in \geninst-directional inference. For
instance, the following two terms differ only in the position of the
annotation, yet only the one on the left-hand side is well-typed.
\begin{mathpar}
 \efun f {\eapp {\einst {(f : \tpoly {\tfor \tv {\tv \to \tv}})}} f}

\efun f {\eapp {\einst f} {(f : \tpoly {\tfor \tv {\tv \to \tv}})}}
\end{mathpar}
The difference lies in how generalization and annotation variables interact.
In the first term, the annotation occurs in an unboxing operator introducing
fresh annotation variables and may therefore be generalized to the type
scheme $\tfor \av {\tapoly {\tfor \tv {\tv \to \tv}} \av}$, enabling
unboxing to proceed. Whereas the second term applies the annotation to the
argument $f$, which fixes $f$'s type to the monotype $\tapoly {\tfor \tv
{\tv \to \tv}} \ava$ for some fresh annotation variable $\ava$. Because this
type is assigned to $f$ at its binding site, $\ava$ is bound in the context
when typing $\einst f$ and cannot be generalized, so the second term is ill-typed despite the annotation.

\parcomment {Suspended match constraints fixes this}

Suspended match constraints eliminate this sensitivity to directionality when type-checking $\einst e$. If $\e$ is already known to have the type $[\ts]$, then we can simply
instantiate it.  However, if the type of $\e$ is not yet known, \ie  it is a
(possibly constrained) type variable $\tv$: then, we must defer until more
information is available. We capture this behaviour with a suspended match constraint:
\begin{mathpar}
\cinfer {\einst \e} \tva \wide\eqdef
    \cexists \tvb \cinfer \e \tvb
\cand
    \cmatch  \tvb {\parens {\cbranch {\tpoly s} s \leq \tva}}
\end{mathpar}
The match constraint is suspended until $\tvb$ is resolved to a polytype $\tpoly \ts$ matching the pattern
$\tpoly s$, which binds the type scheme $\ts$ to the scheme variable $s$. The
selected branch then performs the instantiation $\cleq s \tva$, that is
$\cleq \ts \tva$.
%
% If $\tvb$ is already known to be a polytype, the constraint discharges
% immediately and behaves like a standard instantiation constraint $\cleq \ts
% \tva$.
%
By waiting for the type of $e$ to be \emph{known}, we ensure principal types without annotation variables.

\subsection{Static overloading of constuctors and record labels}

% What do we mean by static overloading?

\emph{Static overloading} denotes a form of overloading in which resolution is
performed entirely at compile time, enabling the compiler to select a unique
implementation without relying on runtime information---in contrast to
\emph{dynamic overloading}, which defers resolution to runtime via
mechanisms such as dictionary-passing or dynamic dispatch.

\parcomment {Other languages}

Many languages offer statically resolved overloading to avoid the overhead
of dynamic dispatch. C++ and Java resolve overloaded functions through
compile-time specialization based on argument types. Conversely, languages
like Rust and Haskell primarily employ dynamic overloading via traits and
type classes, respectively, which can incur runtime overhead unless
optimized away by monomorphization and aggressive inlining.

\parcomment {OCaml and OCaml's approach (PI-directionality)}

As noted in the introduction, \OCaml supports a limited yet useful form of
static overloading for record labels and data type constructors. When
encountering overloaded labels or constructors, \OCaml resolves ambiguity using
local type information, guided by \geninst-directional inference. Nominal types
$\tys \Tapp$ carry annotation variables $\av$, written $\tys \Tapp^\av$. As discussed in
\cref{sec/constraints/polytypes}, this mechanism allows one to deduce that types polymorphic over their annotation variable $\tfor \av {\tys \Tapp^\av}$ are \emph{known}.

\parcomment {A note on bidirectional expected type propagation}

Because static overloading involves more intricate flows of information than
polytype inference, \OCaml supplements \geninst-directionality with a limited,
ad-hoc form of bidirectional type inference. This mechanism is folklore; no
formal account has been given.

\parcomment {Closed-world reasoning}

Beyond propagation, \OCaml also exploits \emph{closed-world reasoning} to resolve
ambiguities in record types. For instance:
\begin{program}[input]
  let e$_{11}$ = {x = 42; y = 1337}
\end{program}
\programjoin
\begin{program}[output]
  val e$_{11}$ : two
\end{program}
Here, \code{x} and \code{y} appear together only in the type \code{two},
allowing the type checker to unambiguously infer the type of \ocaml{e$_{11}$} as
\code{two}.

\parcomment {Default rules}

If local type information and closed-world reasoning are insufficient,
\OCaml falls back to a syntactic default: it selects the most recently
defined compatible type. For example:
\begin{program}[input]
  let getx r = r.x
\end{program}
\programjoin
\begin{program}[output]
  val getx : two -> int
\end{program}
The expression is compatiable with both \code{one} and \code{two},
since each defines a field \code{x}. But \code{two} is chosen simply
because it appears later in the source.
We do not treat this behaviour as principal; accordingly, we provide
no formalization of such ``default'' rules, though their implementation is
discussed further in \cref{sec:discussion}.

This fallback mechanism highlights the directionality of \OCaml inference.
Once the compiler selects a type, it commits to it---even if that choice
causes errors downstream. Consider the example where we flip the order of the
definitons of \code{one} and \code{two}:
\begin{program}[error]
  type two = {x : int; y : int}
  type 'a one = {x : 'a}
  let e$_{12}$ = fun r -> let x (* infers ['a one] *) = r.x in x + r.y
\end{program}
\programjoin
\begin{program}[error, style=message]
  Error: The expression has type int one
	 There is no field y within type one
\end{program}

\parcomment {Record field disambiguation in suspended constraints}

We assume a global typing environment $\labenv$ mapping labels to type schemes,
written $\elab : \tfor \tvs \t \to \tys \Tapp \in \labenv$. A given label $\elab$ may be defined several times in $\labenv$, but at most once at a given record type $\T$. We write
$\labenv(\elab / \T)$ for the type scheme of $\elab$ in $\T$ when it exists.

We propose an alternative account of static overloading using suspended
match constraints.  For example, in the case of an ambiguous record
projection $\efield \e \elab$, we generate the typing constraint:
\begin{mathpar}
\cinfer {\efield \e \elab} \tva \wide\eqdef
  \cexists \tvb \cinfer \e \tvb
  \cand
  \cmatch \tvb
    \parens
      {\cbranch {(\wild \Tapp[t])}
	{\labenv(\elab / t) \leq \tva \to \tvb}
      }
\end{mathpar}
This constraint suspends resolution of the return type $\alpha$ until the record type $\tvb$ of $\e$ is known. Its branch matches against the nominal type pattern $\cpatrcd t$,
binding the type constructor name to $t$. Using this, the appropriate type
scheme for $\elab$ is retrieved from $\labenv(\elab/t)$, instantiated, and the
resulting constraints are imposed on the argument and return types of the label
access.

\parcomment{Suspended constraints are better}

\OCaml programs that do not use the default rule are accepted by this approach. Certain expressions, such as \code{e}$_{12}$ are well-typed under our account but rejected by \OCaml's current type checker.

\subsection{Tuple projections \`a la \SML}

\SML supports positional projections from tuples using expressions of the form
\ocaml{#$j$ e} to extract the $j$-th component of the tuple \code{e}.
%
Internally, tuples in \SML are treated as structural records with numeric
labels, so \ocaml{#$j$ e} desugars into a structural record field access \ocaml{e.$j$}: if $\e$ has the type $\tsrecord {j = \tj; \varrho}$,
where $\varrho$ is a row describing the remaining tuple fields, then $\eproj \e
j$ has type $\tj$.

\SML enforces an additional restriction: the tail $\varrho$ must be fully
determined (\ie cannot be a polymorphic row variable).  This ensures that the
arity of the tuple is \emph{known} statically from the surrounding context,
thereby avoiding the need for row polymorphism. The restriction is not reflected
in the typing rules themselves, but is instead enforced within \SML's type
inference algorithm.

\parcomment {Tuple projection is statically overloaded}

From a typing perspective, tuple projection in \SML behaves like a form
of static overloading: the expression $\eproj \e j$ is valid only when $\e$ is
known to be an $n$-ary tuple for some fixed $n \geq j$.

\parcomment {We can precisely specify this with suspended constraints}

We can capture the typing of tuple projections precisely using suspended
constraints. For the projection $\eproj \e j$, we generate the following
constraint:
\begin{mathpar}
  \cinfer {\eproj \e j } \tv \wide\eqdef
  \cexists \tvb
    \cinfer \e \tvb
    \cand \cmatch \tvb {\cbranch {\cpatprod \tvc j} {\cunif \tva \tvc}}
\end{mathpar}
\parcomment{Tuple patterns}
The suspended constraint $\cmatch \tvb {\cbranch {\cpatprod \tvc j} {\cunif
\tva \tvc}}$ blocks until the shape of $\e$ ($\tvb$) is known to be a tuple
of sufficient arity. The pattern $\cpatprod
\tvc j$ matches only tuple types $\tProd \ti$, where $n \geq j$, binding the
$j$-th component to $\tvc$, which is then unified with the expected result type
$\tva$.

\section{Semantics of constraints}
\label{sec:semantics}

To implement a type-checker using constraint-based type inference it suffices to generate constraints from terms and to solve them. To study the meta-theory of constraint generation and solving, we follow the standard approach of defining \emph{semantics} for our constraints, as declarative as possible. The existence of declarative semantics validates the design of the constraint language.

In our work on suspended constraints, defining a satisfying semantics was the hardest problem: it needs to capture what it means for type information to be \emph{known}. Our semantics is declarative but not syntax-directed, harder to manipulate than standard constraint semantics. On the upside, the semantics directly suggest declarative typing rules for the surface language.

\parcomment {Judgement shape}

The semantics of constraints follows the standard form of a satisfiability
judgment $\semenv \th \c$. The semantic environment $\semenv$ contains a ``guess'' for each free variable of $\c$ (type and term variable), and $\semenv \th \c$ states that these guesses indeed satisfy $\c$. Let us write $\Ground$ for the set of \emph{ground} types, types without free variables.\footnote{Ground types are thus finite trees, assuming the existence
of some base types such as $\tint$. In \cref{sec/rec-types}, we dissuss the
alternative choice of regular trees for the set of ground types that models
equirecursive types.} $\semenv$ maps each type variable $\tv$ to a ground types $\gt \in \Ground$ (the type guessed for $\tv$), and each term variable $x$ to sets of ground types $\glam \subseteq \Ground$ (the set of ground instances of the type scheme guessed for $\x$).
%
We write $\semenv\where{\tv \is \gt}$ and $\semenv\where{\x \is \glam}$ for the extension of $\semenv$ with a new binding. For a type $\t$, we write $\semenv(\t)$ for the ground type obtained by substitution.

\parcomment {Definition exampled}

The judgment is defined in \cref{fig:constraint-semantics} for all constraint-formers except suspended
constraints; its definition on this fragment is standard and somewhat tautological. $\ctrue$ is satisfied in any
environment, and $\cfalse$ in none. Satisfying $\cone
\cand \ctwo$ requires satisfying both $\cone$ and $\ctwo$. Satisfying $\cexists \tv \c$ requires guessing a witness $\gt$ for $\tv$. The universal constraint $\cfor \tv \c$ requires $\c$ to be satisfiable for any binding $\where{\tv \is \gt}$. Unification $\t_1 = \t_2$ is satisfied when $\semenv(\t_1)$ and $\semenv(t_2)$ are equal ground types.

The rule for $(\clet \tv \cone \ctwo)$ states that $\cone$ must satisfied under \emph{some} instantiation of its bound variable (otherwise $\cone$ could be unsatisfiable when $\x$ is not used in $\ctwo$), and that $\ctwo$ must be satisfiable when $\x$ is bound to (the interpretation of) $\cabs \tv \cone$.

Application constraints $\capp \x \t$ are interpreted by checking that $\t$
belongs to the set of types mapped to $\x$ in $\semenv$, that is, $\semenv(\t)
\in \semenv(\x)$. Note that when $\semenv(\x)$ is of the form $\semenvp(\clam \tv \c)$, where $\semenvp$ is the environment at the binding site of $\x$, then $\semenv(\t) \in \semenv(x)$ holds iff $\semenvp\where{\tv \is \semenv(\t)} \th \c$, which corresponds to the intuition that the application $\capp {(\clam \tv \c)} \t$ should be equivalent to $\c \where {\tv \is \t}$.

\begin{mathparfig}[t]%
  {fig:constraint-semantics}%
  {Semantics of constraints (without suspended constraints)}
  \begin{bnfgrammar}
    \entry[Semantic environments]{\semenv}{\emptyset \and \semenv\where{\tv \is \gt} \and \semenv\where{\x \is \glam}
    }
  \end{bnfgrammar}

  \infer[True]
    {}
    {\semenv \th \ctrue}

  \infer[Conj]
    {\semenv \th \cone \\
     \semenv \th \ctwo}
    {\semenv \th \cone \cand \ctwo}

  \infer[Exists]
    {\semenv\where{\tv \is \gt} \th \c}
    {\semenv \th \cexists \tv \c}

  \infer[Forall]
    {\forall \gt, ~ \semenv\where{\tv \is \gt} \th \c}
    {\semenv \th \tfor \tv \c}

  \infer[Unif]
    {\semenv(\tone) = \semenv(\ttwo)}
    {\semenv \th \cunif \tone \ttwo}

  \semenv(\clam \tv \c) \eqdef \set {\gt \in \Ground : \semenv\where{\tv \is \gt} \th \c}

  \infer[Let]
    {\semenv \th \exists \tv. \cone \\
     \semenv\where{\x \is \semenv(\cabs \tv \cone)} \th \ctwo}
    {\semenv \th \clet \x \tv \cone \ctwo}

  \infer[App]
    {\semenv(\t) \in \semenv(\x)}
    {\semenv \th \capp x \t}

  \begin{array}{lll}
  \cone \centails \ctwo & \eqdef & \forall \semenv,\ \semenv \th \cone \implies \semenv \th \ctwo
  \\
  \cone \cequiv \ctwo & \eqdef & \forall \semenv,\ \semenv \th \cone \iff \semenv \th \ctwo
  \end{array}
\end{mathparfig}


Closed constraints are either satisfiable in any semantic environment (\ie they are tautologies)
or unsatisfiable. For example, consider the constraint $\cexists
\tv {\cunif \tv \tint}$, its satisfiability is established by the following derivation:
\begin{mathline}
  \infer*[Right=Exists]
    {\infer*[Right=Unif]
      {\infer*{}{\tint = \tint}}
      {\semenv\where{\tv \is \tint} \th \cunif \tv \tint}}
  {\semenv \th \cexists \tv \cunif \tv \tint}
\end{mathline}

% Equivalence and entailment

We write $\cone \centails \ctwo$ to express that $\cone$ \emph{entails} $\ctwo$,
meaning every solution $\semenv$ to $\cone$ is also a solution to $\ctwo$.
We write $\cone \cequiv \ctwo$ to indicate that $\cone$ and $\ctwo$ are equivalent,
that is, they have exactly the same set of solutions.

\subsection{Shapes
\label{sec/shapes}}

We introduce \emph{canonical shapes} for use in the syntax of suspended constraints. It is a generalization of type constructors that is capable of copying with polytypes -- a polytype $(\forall \tvb. \tvb \to (\tva_1 \to \tva_2))$ can be seen as a constructor-like structure $\any \tvc (\forall \tvb. \tvb \to \tvc)$ applied with the argument $(\tva_1 \to \tva_2)$. Canoical shapes $\sh$ are defined as a subset of shapes $\Sh$ satisying some minimality properties.

\parcomment {Definition of shape}

A shape $\Sh$ is a type with holes, written $\any \tvcs \t$, where $\tvcs$
denotes the set of type variables representing the holes.
By construction, we require $\tvcs$ to be exactly the free
variables of $\t$.  Hence, shapes are closed and considered up
to $\alpha$-conversion.  When $\t$ is a ground type, we omit the
binder and write simply $\t$.
%
We write $\bot$ for the shape $\any \tvc \tvc$, which we call the
\emph{trivial} shape. We write $\Shapes$ the set of shapes and $\Shapesz$ the
subset of non trivial shapes.
%% and use letter $\Sh$ to range
%% over non trivial shapes.

Shapes are equipped with an ordering---that of the standard
instantiation ordering, defined by:
\begin{mathpar}
  \infer[Inst-Shape]
    {\bar \tvcs_2 \disjoint \any {\tvcs_1} \t}
    {\any {\tvcs_1} \t \preceq
     \any {\tvcs_2} \t \where {\tvcs_1 \is \tys_1}}
\end{mathpar}
When writing $\Sh \preceq \Shp$, we say that $\Sh$ is more general than
$\Shp$. When $\Sh$ and $\Shp$ are more general than one another, they are
actually equal. The trivial shape $\bot$ is the most general shape.

If $\Sh$ is $\any \tvcs \t$, the shape application $\shapp[\Sh] \tys$ is defined as $\t
\where {\tvcs \is \tys}$. We say that $\Sh$ is a shape of $\t$ when there exists $\tys$ such that $\t = \shapp[\Sh] \tys$; in this case we write that the pair $(\Sh, \tys)$ is a decomposition of $\t$.

\begin{definition}
A non-trival shape $\Sh \in \Shapesz$ is the principal shape of the type $\t$ iff:
\begin{enumerate}
  \item
    $\exists \typs,\ \t = \shapp[\Sh] \typs$
  \item
    $\forall \Shp \in \Shapesz, \forall \typs,\ \t = \shapp[\Shp] \typs
    \implies \Sh \preceq \Shp$
\end{enumerate}
\end{definition}

\begin{theorem}[principal shapes]\label{th/shapes/principal}
Any nonvariable type $\t$ has a non-trivial principal shape $\Sh$.
\end{theorem}

We can give an equivalent direct description of the set of principal shapes $\sh$: they are the $\any \tvcs \t$
\begin{itemize}

\item $\tvcs$ are the linear free variables of $\t$ \ie each type variable
  $\tvc$ in $\tvcs$ occurs exactly once in $\t$

\item
  $\t$ is shallow, which means the following for our particular type formers:
\begin{itemize}

\item
  when $\t$ is not a polytype, all subterms of $\t$ are variables: $\tvca \to \tvcb$, $\Pi\iton \tvci$, or $\tvcs \T$

\item
  when $\t$ is a polytype $\tpoly {\all \tvs \tsp}$, then the only subterms of
  $\tsp$ that do not contain one of the $\tvs$
  are variables in $\tvcs$.

  For example, the polytype  $\tpoly {\all \tva
    {\parens{\tpoly {\all \tvb {\parens{\tvb \to \tint \tlist}} \tprod \tvb}
        \to \tva}
      \to \tva}}$ has principal decomposition
  $
  \any
    {\tvc}
    {\tpoly {\all \tva
       {\parens{\tpoly {\all \tvb \parens{\tvb \to \tvc} \tprod \tvb} \to \tva}
      \to \tva}}}
  $.
\end{itemize}
\end{itemize}

\parcomment {Define canonical shape}

A principal shape $\any \tvcs \t$ is \emph{canonical} if the sequence of its free
variables $\tvcs$ appear in the order in which the variables occur in
$\t$\footnote{This order is well-defned for principal shapes, but not always
well-defined for all shapes, for instance $\any {\tvca, \tvcb} (\tvca \to
\tvcb) \to (\tvcb \to \tvca)$.\\\Xgabriel{I don't understand this comment. Here clearly $\tvca$ occurs before $\tvcb$ in the printing of the shape body, so it would be reasonable to say that they are in a well-defined order. And if we refuse to consider the order on deep subnodes, then how do we define canonicity for polytypes such as the following? $\any {\tvca, \tvcb, \tvcc} \tpoly {\all \tva  {(\tvca \to (\tvcb \times \tva)) \to \tvcc}}$}}. We write $\sh$ for canonical principal
shapes.
%
Each non-variable type $\t$ has a unique canonical principal shape, which we write $\shape \t$. We write $\decomp \t$ for the decomposition of $\t$ induced by its canonical principal shape. For example, $\shape {\tys \Tapp}$ is $(\any \tvcs \tvcs \Tapp)$ and $\decomp {\tint \to \tbool}$ is the pair $((\any {\tvca, \tvcb} {\tvca \to \tvcb}), (\tint, \tbool))$.

\TODO{Consistently use prefix notation for type constructors, except in OCaml examples.}


\subsection{Suspended constraints}

We have left the syntax of shape patterns deliberately abstract. We also assume a matching relation
\begin{mathline}
  \cmatches \cpat {(\sh, \tys)} \theta
\end{mathline}
This partial function matches a pattern $\cpat$ against a principal
shape $\sh$ with components $\tys$, yielding a substitution $\theta$
for any variables bound in the pattern.\footnote{\Xgabriel{I find it
    mildly irritating that we match patterns $\cpat$ against
    decompositions $(\sh, \tys)$, instead of matching them against
    shapes $\any \tvcs \t$ and adding $\theta\where{\tvcs \is \tys}$
    to the semantic environment. That would ensure uniformity by
    construction, whereas for the current approach we need to check
    each partial definition of $\cmatches \rho {(\sh, \tys)} \theta$
    for uniformity. Can I make the change?}}
%
For our examples we define the trivial pattern $\cwild$ which matches
any shape and binds nothing:
\begin{mathline}
  \cmatches[\eqdef] \cwild {\pshapp \tys} \eset
\end{mathline}


\begin{definition}[Discharged constraint]
  Given a suspended constraint $(\cmatch \t \cbrs)$ and a canonical shape $\sh$, we introduce the syntactic sugar $(\cmatched \t \sh \cbrs)$ for the \emph{discharged constraint} that selects the branch in $\cbrs$ that matches $\sh$:
\begin{mathpar}
  \cmatched \t \sh {\cbranch \cpats \cs} \uad\eqdef\uad
    \cexists \tvs \cunif \t \shapp \tvs \cand \theta(\ci) \qquad \text{if }
    \cmatches \cpati {(\sh, \tvs)} \theta
\end{mathpar}
The first conjunct ($\tau = \shapp \tvs$) ensures that $\sh$ is indeed
the canonical shape of $\t$, and the second conjunct is $\ci$ under
the appropriate substitution. The syntax of suspended match
constraints requires that branch patterns are non-overlapping, so the
matching branch $\cbranch \cpati \ci$ is uniquely determined; but it
may not exist as branches need not be exhaustive. The discharged
constraint is only defined when a matching $\cpati$ exists.
\end{definition}

\paragraph {A natural attempt}

To provide semantics for our suspended constraints, a first idea
is to propose the following rule---henceforth referred to as the
\emph{natural semantics} of suspended constraints:
\begin{mathpar}
\infer[Susp-Nat]
  {\sh = \shape {\semenv(\t)} \and \semenv \th \cmatched \t \sh \cbrs}
  {\semenv \th \cmatch \t \cbrs}
\end{mathpar}
This rule states that a suspended constraint is satisfied by $\semenv$ whenever the corresponding discharged constraint holds for the canonical shape $\sh$ of $\t$ in the semantic environment $\semenv$. If $\sh$ matches no branch in $\cbrs$, then the discharged constraint is not defined, so this rule cannot be applied and the suspended constraint is unsatisfiable.

% Consider the constraint $\cexists \tv \cunif \tv \tint \cand \cmatch \tv
% {\cbranch \cwild \ctrue}$. This is satisfiable, as intended, under the current
% semantics, as shown by the derivation:
% \begin{mathline}
% \def \cmatchex {\cmatch \tv {\cbranch \cwild \ctrue}}
% \def \semenvex {\semenv\where{\tv \is \tint}}
%     \infer*[Right=Conj]
%     {
%      \infer*[Left=Unif]
%       {\tint = \tint}
%       % -------------------------------
%       {\semenvex \th \cunif \tv \tint}
%      \\
%      \infer*[Right=Susp-Nat]
%       {
% 	\cmatches \cwild {\pshapp[\tint]\cdot} \eset
% 	\\
% 	\infer*[Right=True]
% 	  { }
% 	  % ---------------------
% 	  {\semenvex \th \ctrue}
%       }
%       % ------------------------
%       {\semenvex \th \cmatchex}
%     \hspace{-2em}
% }{% ---------------------
%     \infer*[Right=Exists]
%       {\semenvex \th \cunif \tv \tint \cand \cmatchex}
%     % ------------------------------------------------------------
%       {\semenv \th \cexists \tv \cunif \tv \tint \cand \cmatchex}
% }
% \end{mathline}

\parcomment {The problem}

This semantics rule is nicely declarative, but unfortunately it accepts to omany constraints. Consider for example: $\cexists \tv \cmatch \tv {\cbranch \cwild {\cunif \tv \tint}}$. It is satisfiable with this natural semantics:
\begin{mathpar}
\def \cmatchex {\cmatch \tv {\cbranch \cwild {\cunif \tv \tint}}}
\def \semenvex {\semenv\where{\tv \is \tint}}
    \infer*[Right=Susp-Nat]
    {
      \cmatches \cwild {\pshapp[\tint]\cdot} \eset
      \\
      \infer*[Right=Unif]
        {\tint = \tint}
	% -------------------------------
    {\semenvex \th \cunif \tv \tint}
}{% ---------------------------------
    \infer*[Right=Exists]
    {\semenvex \th \cmatchex}
  % -----------------------------------
  {\semenv \th \cexists \tv \cmatchex}
}
\end{mathpar}
The semantics can \emph{guess} the type of $\tv$ and use it to unlock the match constraint, rather than requiring it to be \emph{known} from the surrounding context. One could call the guess of $\cunif \tva \tint$ an ``out of thin air'' behavior. This does not match the intended meaning of suspended match constraints, and raises several problems:
\begin{enumerate*}

  \item A reasonable solver---one that avoids guessing or backtracking---cannot
    be complete with respect to this semantics.

  \item This breaks the existence of principal solutions.
    Consider the function $\efun \x (\efield \x 2)$, which projects the second
    component of a tuple. Under the natural semantics, the generated constraint
    permits us to guess for $\x$ any tuple type of arity at least $2$; so there is no principal type for $\x$.
\end{enumerate*}

\paragraph {Contextual semantics}

To rule out guessing, we instead adopt a \emph{contextual} semantics in which the shape of a type must be known from the surrounding context in order to satisfy a match constraint. Its rule for suspended constraint is the only non-syntax-directed rule in our semantics:
\begin{mathpar}
  \infer[Susp-Ctx]
    {\Cshape \C \t \sh \\
      \semenv \th \C \where {\cmatched \t \sh \cbrs}
    }
    {\semenv \th \C \where {\cmatch \t \cbrs}}
\end{mathpar}
In this rule the shape $\sh$ is not guessed from $\semenv$, it must be determined from the constraint context $\C$. The \emph{unicity} condition $\Cshape \C \t \sh$, defined below, ensures that $\sh$ is known.

\begin{definition}[Erasure]
  We define the erasure $\cerase \c$ as the constraint
  where all suspended match constraints in $\c$  have been replaced by
  $\ctrue$; we define it in full in \cref{fig:constraint-erasure}
  (appendix).
\end{definition}

\begin{definition}[Simple constraints]
  We say that $\c$ is \emph{simple} if it does not contain any suspended match constraint. We write $\semenv \thsimple \c$ for a derivation of $\semenv \th \c$ that only uses the rules listed in \cref{fig:constraint-semantics}, without using \Rule{Susp-Ctx}. This judgment coincides with $\semenv \th \C$ on simple constraints.
\end{definition}

\begin{definition}[Unicity]
  We define the unicity condition $\Cshape \C t \sh$, which expresses that $\t$ has a unique canonical shape within the context $\C$, which is $\sh$.
  \begin{mathpar}
    \Cshape \C \t \sh \Wide\eqdef \forall \semenv, \gt. \uad
      \semenv \thsimple \cerase {\C\where{\cunif \t \gt}} \implies \shape \gt = \sh
  \end{mathpar}
\end{definition}

The use of erasure $\cerase {\C\where{\cunif \t \gt}}$ in the definition of $\Cshape \C t \sh$ ensures that the unicity of $\sh$ is determined only by the constraints that have already been discharged in $\C$, and cannot be justified from suspended constraints that will be discharged in the future. Implicitly, this induces a linear partial order between the suspended match constraints of a constraint, reflecting \emph{temporal} dependency: a match constraint may only be discharged once all of its dependencies have been discharged.

The erasure is a simple constraint so we use the
$(\thsimple)$ judgment, to avoid well-foundedness issues that would come from
a negative use of $(\th)$ in a premise of \Rule{Susp-Ctx}.
%
Note that, if $\t$ is not a variable, then $\Cshape \square \t \sh$
holds trivially for $\sh = \shape \t$. Also, if $\C$ is unsatisfiable,
then $\Cshape \C \tv \sh$ holds for any $\sh$. The interesting cases
arise when $\t$ is a type variable and $\C$ is satisfiable.

We summarize the definition of the unicity condition and the \Rule{Susp-Ctx} rule in \cref{fig:contextual-semantics}; together with \cref{fig:constraint-semantics} this forms the complete semantics of our consstraint language.

\begin{mathparfig}[t]
  {fig:contextual-semantics}
  {Semantics of suspended constraints}
\begin{array}{l}
\Cshape \C \t \sh \eqdef \forall \semenv, \gt. \\
\qquad
      \semenv \thsimple \cerase {\C\where{\cunif \t \gt}} \implies \shape \gt = \sh
\end{array}

  \infer[Susp-Ctx]
    {\Cshape \C \t \sh \\
      \semenv \th \C\where{\cmatched \t \sh \cbrs}
    }
    {\semenv \th \C\where{\cmatch \t \cbrs}}
\end{mathparfig}

\parcomment {Examples}

\begin{example}
Consider the two examples from above:
\begin{mathpar}
\cexists \tv \cunif \tv \tint
  \cand
  \cmatch \tv {\cbranch \cwild \ctrue}

  \cexists \tv \cmatch \tv {\cbranch \cwild {\cunif \tv \tint}}
\end{mathpar}
In the first example, we apply the contextual rule with the context $\C \is
(\exists \tv. \tv = \tint \cand \square)$. Any solution $\phi$ of this part
of the constraint necessarily satisfies $\tv = \tint$, so we have $\Cshape \C \tv \tint$ and the suspended constraint can be resolved.

In constrast, the second example has no contextual information around
the suspended constraint ($\C \is \square$), so any solution
$\semenv$ satisfies it, and $\semenv(\tv)$ can have an arbitrary shape, for example $\tint$ or $\tbool$. The uniqueness condition $\Cshape \C \tv \sh$ never holds and the constraint is unsatisfiable as intended.
\end{example}
\begin{example}
Consider the more intricate example:
\begin{mathpar}
  \cexists {\tva \tvb}
  \Parens{\begin{array}{l}
    \quad \cmatch \tva {\cbranch \cwild {\cunif \tvb \tbool}} \\
    {} \cand \cmatch \tvb {\cbranch \cpatwild \ctrue} \\
    {} \cand \tva = \tint
  \end{array}}
\end{mathpar}

Suppose we attempt to apply \Rule{Susp-Ctx} to the suspension on $\tvb$ first.
We want to show $\Cshape \C \tvb \tbool$ for the
context $\C \is \cmatch \tv {(\cbranch \cpatwild {\cunif \tvb \tbool})} \cand
\square \cand \cunif \tva \tint$. But the erasure $\cerase \C$ is $\ctrue \cand \square \cand \cunif \tva \tint$. In this constraint $\tvb$ is unconstrained, so for example $\cerase {\C {\cunif \tvb \tint}}$ and $\cerase {\C {\cunif \tvb \tbool}}$ are both satisfiable: unicity does not hold and \Rule{Susp-Ctx} cannot be applied.

Now consider instead applying \Rule{Susp-Ctx} to the suspension on $\tva$
first. To do, so, we must show that $\tva$ has a uniquely determined shape in
the context $\C \is \square \cand \cmatch \tvb {\cbranch \cpatwild \ctrue}
\cand \cunif \tva \tint$. The erasure $\cerase \C$ is
$(\square \cand \ctrue \cand \cunif \tva \tint)$. In this context $\tva$
is uniquely determined, so we have $\Cshape \C \tva \tint$.
%
We may now resolve the suspension on $\tva$: the discharged constraint $(\cmatched \tva \tint {\cbranch \cpatwild {\cunif \tvb \tbool}})$ is $(\tva = \tint \cand \cunif \tvb \tbool)$, so we are left to satisfy the constraint $\C \where {\tva = \tint \cand \cunif \tvb \tbool}$, that is,
$(\cunif \tva \tint \cand \cunif \tvb \tbool \cand \cmatch \tvb
{\cbranch \cpatwild \ctrue} \cand \cunif \tva \tint)$. At this point, we can
safely apply \Rule{Susp-Ctx} to the remaining match constraint on $\tvb$.
The unicity condition holds now that the erasure of the context includes the
discharged match constraint $\cunif \tvb \tbool$, so we can discharge
eliminating the second suspended match.

This demonstrates that suspended match constraints must be resolved in a
dependency-respecting order: attempting to resolve a match
constraint too early may result in unsatisfiability.
\end{example}

\begin{example}
Let us consider a constraint with a cyclic dependency between match
constraints:
\begin{mathpar}
  \cexists {\tva \tvb}
  \left(\begin{array}{l}
    \quad \cmatch \tva {\cbranch \cwild {\cunif \tvb \tbool}} \\
    {} \cand \cmatch \tvb {\cbranch \cwild {\cunif \tva \tint}}
  \end{array}\right)
\end{mathpar}
This constraint can be proved satisfiable under the ``natural semantics'' introduced
earlier: by guessing the assignment $\tva \is \tint, \tvb \is
\tbool$, the two matches suceed and the constraint holds. However,
our solver fails to solve it, as does our contextual semantics.

By symmetry we can try to apply \Rule{Susp-Ctx} on $\tva$ first: we
must show $\Cshape \C \tva \tint$ for the context
$\C \eqdef (\square \cand \cmatch \tvb {\cbranch \cwild {\cunif \tva \tint}})$. But
this unicity condition does not hold as the erasure
$\cerase \C = \square \cand \ctrue$ does not constrain $\tva$, so we cannot
apply the \Rule{Susp-Ctx} rule.
\end{example}

\begin{example}
Considering the example \code{e_7} from \cref{sec/introduction}:
\begin{program}[input]
let e_7 r = let x = r.x in x + r.y
\end{program}
The constraint generated when typing \code{e_7} contains the following, where $\tv$ stands for the type of \code{r}:
\begin{mathpar}
  \cexists \tv
    \clet x \tvb
      {(\cmatch \tva \dots)}
      {\cinst x \tint \cand \cunif \tv {\mathsf{two}}}
\end{mathpar}
The suspended constraint can be discharged under our semantics, as intended. We apply the
\Rule{Susp-Ctx} rule with context $\C \is \clet \x \tvb \square \capp \x
\tint \cand \cunif \tv {\mathsf{two}}$. Although the context includes a
\code{let}-binding --- which in practice involves let-generalization --- we
can still deduce $\Cshape \C \tv {\mathsf{two}}$, since the erased context $\cerase \C$ contains the unification $\cunif \tv
{\mathsf{two}}$.

This example illustrates that our formulation of suspended constraints
interacts nicely with \code{let}-polymorphism. Although the two features are
specified in a modular fashion, they are carefully crafted to work together,
as we will further show in our next example.
\end{example}

\begin{example}\label{ex:backprop}
A subtle yet crucial feature of our semantics is its support for
\emph{backpropagation}:
\begin{program}[input]
let e_8 = let getx r = r.x in getx { x = 1 }
\end{program}
As in the previous example the type of \code{r} cannot be disambiguated in the \code{let}-definition alone. In the previous example, this type was unified to a known shape in the \code{let}-body. Here this is more subtle: an \emph{instance} of the type scheme is taken, which is only well-typed if \code{r} has a variable type or a type of the form \lstinline[mathescape=true]{$\tva$ one}. The projection \code{r.x} would be forbidden if \code{r} had a variable type, so \lstinline[mathescape=true]{$\tva$ one} is the unique solution. We call \emph{backpropagation} the information flow from instances to definitions.

The constraint generated when typing
\code{e_8} is:
\begin{mathpar}
\begin{tabular}{L.L}
  \cexists \tv {}
  &\clet {getx} \tvd
     {\cexists {\tvb, \tvc} \Parens {\strut
        \cunif \tvd {\tvb \to \tvc} \cand
	\cmatch \tvb \dots
        }}{}
    \cinst {getx} {(\tint \ \mathsf{one} \to \tv)}
\end{tabular}
\end{mathpar}
With the context $\C$ equal to $\clet {getx}
\tvd {\cexists {\tvb, \tvc} \cunif \tvd {\tvb \to \tvc} \cand \square}
{\capp {getx} {(\tint \ \mathsf{one} \to \tv)}}$, we can show the unicity predicate $\Cshape \C \tvb \sh$ for the shape $\sh \eqdef (\any \tvc {\tvc~\mathsf{one}})$. :
\begin{mathpar}
  \all {\semenv, \gt} \uad
    \semenv \th \cerase {\C\where{\cunif \tvb \gt}} \implies \shape \gt = \sh
\end{mathpar}
For any $\semenv, \gt$, the erasure $\cerase {\C \where {\cunif \tvb \gt}}$ is
$\clet {getx}
\tvd {\cexists {\tvb, \tvc} \cunif \tvd {\tvb \to \tvc} \cand \cunif \tvb \gt}
{\capp {getx} {(\tint \ \mathsf{one} \to \tv)}}$. $getx$ is bound to the constraint abstraction $\cabs \tvd \exists \tvc.\uad \cunif \delta {(\gt \to \gamma)}$, so the instantiation $getx (\tint \ \mathsf{one} \to \tv)$ can only be satisfied when $\gt = \tint\ \mathsf{one}$. This proves unicity, so \code{e_7} is accepted by our semantics.
\end{example}

\section{The \OML calculus}
\label{sec:language}

\parcomment {Running example: tuple projection disambiguation}

\parcomment {We need a spec, but this itself is hard}

In order to show our approach is sound and complete with respect to our
constraint generation translation, we must first define a formal
specification in the form of a calculus and accompanying type
system. Surprisngly, identifying an appropriate declarative type system to
use as a specification is itself an interesting problem!

\parcomment {Why do naive approaches not guarantee principal types.}

Na\"ive specifications, though accessible, often lack principal types. Take
overloaded tuple projects \ala \SML: an expression $\efield e j$
governed by the rule:
\begin{mathpar}
  \infer
    {\G \th \e : \Pi\iton \ti \and 1 \leq j \leq n}
    {\G \th \efield \e j : \tj}
\end{mathpar}
admits many typings, as any tuple of at least size $j$ satisfies the
premise. This multiplicity undermines principality.

\parcomment {Partial annotations (shapes) to the rescue!}

Our first insight is that partial type annotations often suffices to
recover principality. For
example, explicitly annotating the tuple projection with its arity, as in
$\exfield \e n j$, disambiguates the expression:
\begin{mathpar}
   \inferrule*
      {\G \th \e : \Pi\iton \ti \\ 1 \leq j \leq n }
      {\G \th \exfield \e n j : \tj}
\end{mathpar}
Yet, requiring users to write such annotations manually is impractical.

\parcomment {Inference of shapes}

Our solution is to permit \textit{inference} of such information, provided
that the inferred annotation is \textit{uniquely determined by context}.  This
approach ensures principality by construction and yields a clear specification
with predictable annotability requirements for the everyday programmer.

\parcomment {How do we specify our intuition?}

Specifying the notion of unique inferrability from context is
tricky. Luckily, we can leverage many of the formal methods developed in the
previous section---in particular, the manipulation of contexts within the
derivation---to prove uniqueness of these partial annotations. Allowing us to
systematically replace the ambiguous constructs with their disambiguated
counterparts:
\begin{mathpar}
  \inferrule*
    {\eshape \E \e {\any \tvcs \Pi\iton \tvcs} \\ \G \th E[\exfield \e n j] : \t}
    {\G \th \E[\efield \e j] : \t}
\end{mathpar}
The intuition here is that the context $\E$ is sufficiently large to
constrain the shape of $\e$'s type as a tuple of size $n$.

\parcomment {Limitations}

The chief limitation of our approach is that disambiguation can still
require explicit annotations in inherently ambiguous scenarios.  This is
fine. Another option would be the use of sensible ``defaults'', such as
resolving $\efield e j$ by assuming the tuple has arity $j$. While
appealing, such defaults compromise principality. We return to this
pragmatic tradeoff in \cref{sec:discussion}.

\subsection{Syntax}

\begin{bnffig}{fig/syntax}{Syntax of \OML}
\entry[Terms]{\e}{
  x \and
  () \and
  \efun x e \and
  \eapp \eone \etwo \and
  \elet x \eone \etwo \and
  \eannot \e \tvs \t \andcr
  \erecord {\overline{\el = \e} } \and
  \efield e \el \andcr
   (\eone, \ldots, \en) \and
   \efield e j \and
   \exfield e n j \andcr
   \epoly e \and
   \epoly[\exi \tvs \ts] e  \and
   \einst e \and
   \exinst e \tvs \ts
}\\
\entry[Labels]{l}{
  \elab \and \elab / \T
}\\[1ex]
\entry[Types]{\t}{
   \tv \and
   1 \and
   \tya \to \tyb \and
   \tys \T \and
   \Pi \iton \ti \and
   \tpoly \ts
}\\
\entry[Type schemes]{\ts}{
     \t \and
     \tfor \tv \ts
}\\
\entry[Contexts]{\G}{
   \eset \and
   \G, x : \ts
}\\
\end{bnffig}


In \cref {fig/syntax}, we give the grammar for our calculus. Terms include
all of the \ML calculus: variables $x$, the unit literal $\eunit$,
lambda-abstractions $\efun x e$, applications $\eapp \eone \etwo$,
annotations $\eannot \e \tvs \t$ and let-bindings $\elet x \eone \etwo$.
Our extensions include:
\begin{enumerate}
\item
  Constructor and record label disambiguation, modelled using record
  literals $\erecord { \ela = \eone; \ldots; \el_n = \en }$ and field
  projections $\efield e l$.

\item
  Tuples $(\eone, \ldots, \en)$ with overloaded tuple projections
  $\efield e j$.

\item
  For semi-explicit first-class polymorphism, we have the explicit and
    implicit boxing constructs
    $\expoly \e {\exi \tvs \ts}$, $\epoly \e$  and the unboxing construct $\einst e$.

\end{enumerate}
Each construct that endangers principality (written $\e^i$ or $\el^i$) has an
explicitly annotated counterpart (written $\e^x$ or $\el^x$) that is
$\eproj [n] \e j$, $\epoly [\exi \tvs \ts]
\e$, $\einst [\exi \tvs \ts] \e$, or $\elab / \T$ for record labels.

Typing contexts $\G$ are an ordered sequence of expression variable
typings $x : \ts$.

\subsection{Typing rules}
\label{sec/language/typing-rules}

As usual, the main typing judgment $\G \th \e : \ts$ states that in context
$\G$, expression $\e$ has type scheme $\ts$.  Typing rules are given on \cref
{fig/typing}.  They use auxiliary typing judgemnents $\G \th \elab = \e : \t$
and $\G \th \el : \t \to \tp$ for the typing of record assignments and
label instantiations respectively.

\begin{mathparfig}{fig/typing}{Typing rules of \OML}
  \inferrule[Var]
    {x : \sigma \in \G}
    {\G \th x : \sigma}

  \inferrule[Fun]
    {\G, x : \tone \th e : \ttwo }
    {\G \th \efun x e : \tone \to \ttwo}

  \inferrule[App]
    {\G \th \eone : \tone \to \ttwo \\
     \G \th \etwo : \tone}
    {\G \th \eapp \eone \etwo : \ttwo}

  \inferrule[Unit]
    {}
    {\G \th () : 1}

  \inferrule[Annot]
    {\G \th e : \t\where {\tvs \is \tys}}
    {\G \th (e : \exi \tvs \t) : \t\where {\tvs \is \tys}}

  \inferrule[Gen]
    {\G \th e : \sigma \\ \tv \disjoint \G}
    {\G \th e : \tfor \tv \sigma}

  \inferrule[Inst]
    {\G \th e : \tfor \tv \ts}
    {\G \th e : \tv \where{\tv \is \t}}

  \inferrule[Let]
    {\G \th \eone : \sigma \\
     \G, x : \sigma \th \etwo : \t}
    {\G \th \elet x \eone \etwo : \t}

  \inferrule[Tuple]
    {\parens{\G \th \ei : \ti}\iton}
    {\G \th (\eone, \ldots, \en) : \Pi\iton \ti}

  \inferrule[Proj-X]
    {\G \th \e : \Pi\iton \ti \\
     1 \leq j \leq n}
    {\G \th \exfield \e n j : \tj}

  \inferrule[Proj-I]
    {\eshape \E \e {\any \tvcs \Pi\iton \tvcs} \\
     \G \th \E\where{\exfield \e n j} : \t}
    {\G \th \E\where{\efield \e j} : \t}

  \inferrule [Poly-X]
    {\G \th \e : \ts\where {\tvs \is \tys}}
    {\G \th \epoly[\exi \tvs \ts] \e : \tpoly {\ts \where {\tvs \is \tys}}}

  \inferrule [Poly-I]
    {\Eshape \E \e {{\any \tvcs \tpoly \ts}} \\
     \G \th \E \where{\epoly[\exi \tvcs \ts] \e} : \t}
    {\G \th \E \where{\epoly \e} : \t}

  \inferrule [Use-X]
    {\G \th \e : \tpoly \ts \where {\tvs \is \tys}}
    {\G \th \exinst e \tvs \ts : \ts \where {\tvs \is \tys}}

  \inferrule [Use-I]
    {\eshape \E  \e {\any \tvcs \tpoly \ts} \\
     \G \th \E\where{\exinst \e \tvcs \ts} : \t}
    {\G \th \E\where{\einst \e} : \t}

  \inferrule[Rcd-Assn]
    {\G \th \e : \t \\
     \G \th \el : \t \to \tp}
    {\G \th \el = \e : \tp}

  \inferrule[Rcd]
    {\parens{\G \th \eli = \ei : \t}\iton \\
     \G \th \bar \el \uni \t}
    {\G \th \erecord {\ela = \eone; \ldots; \el_n = \en} : \t}

  \inferrule[Rcd-Proj]
    {\G \th \e : \tp \\
     \G \th \el : \t \to \tp \\
     \G \th \el \uni \t }
    {\G \th \efield \e \el : \t}

  \inferrule[Lab-X]
    {\Omega(\elab / \T) = \tfor \tvs \t \to \tvs \T }
    {\G \th \elab / \T : \tys\where{\tvs \is \tys} \to \tys \T}

  \inferrule[Lab-I]
    {\Lshape \Lab \elab \T \\
      \G \th \Lab[\elab / \T] : \t}
    {\G \th \Lab[\elab] : \t}

  \inferrule[Lab-!]
    {\bar \el \uni \T \in \labenv}
    {\G \th \bar \el \uni \tys \T}

  \inferrule[Lab-?]
    {\G \th \t}
    {\G \th \bar \el \uni \t}
\end{mathparfig}

\TODO{We should consider handling the three features (tuples, records, polytypes) separately, one in each subsection.
If we later run out of space budget it will be easier to move only some of those subsections to an appendix.
Concrete proposal:
\begin{itemize}
\item one subsection at the beginning that only hints at the standard ML stuff:
  show the syntax, hint at typing rules, show a short bit of constraint generation
\item then each advanced feature (from the simplest to the more complex: tuples, polytypes, records)
  show the typing rules, and examples, and its dedicated patterns, and its constraint generation
\end{itemize}}

\begin{version}{}
\begin{bnffig}{fig/types/bnf}{aaa}
\entry[Types]{\t}{
    \tv \and
    \tunit \and
    \t \to \t \and
    \Pi \parens \t\iton \and
    \tys \T \and
    \tpoly \ts
}
\end{bnffig}

\begin{mathparfig}
  {fig/types/wf}
  {Well-formedness of types}
  \inferrule[Var-Wf]
    {\tv \in \G}
    {\G \th \tv}

  \inferrule[Unit-Wf]
    {}
    {\G \th \tunit}

  \inferrule[Arr-Wf]
    {\G \th \t \\ \G \th \tp}
    {\G \th \t \to \tp}

  \inferrule[Prod-Wf]
    {(\G \th \ti)\iton}
    {\G \th \Pi\iton \ti}

  \inferrule[Rcd-Wf]
    {(\G \th \ti)\iton \\
     \T \in \dom \Omega}
    {\G \th \tys \T}

  \inferrule[Poly-Wf]
    {\G \th \ts}
    {\G \th \tpoly \ts}

  \inferrule[Forall-Wf]
    {\G, \tv \th \ts}
    {\G \th \tfor \tv \ts}
\end{mathparfig}
\end{version}


\parcomment {Simple typing rules explained}

Rule \Rule{Var} retrieves the type scheme $\x : \ts$ from the context $\G$.
Function types are introduced via lambda abstractions: in Rule \Rule{Fun}, the
system guesses a well-formed type $\tone$ for the type of $x$, typechecks the
body $e$ is under the extended context $\G, \x : \tya$ producing the return
type $\tyb$, and assigns the abstraction the function type $\tya \to \tyb$.
Conversely, function types are eliminated by applications; in Rule \Rule{App},
the type of the argument must match the function's parameter type $\tya$ and
application returns the type $\tyb$. Rule \Rule{Unit} asserts that $\eunit$ has
the unit type $\tunit$.

\parcomment {Gen/Inst explained}

Rules \Rule{Gen} and \Rule{Inst} correspond to implicit
\textit{generalization} and \textit{instantiation} respectively.
Generalization universally quantifies a type variable $\tv$, introducing it
as a fresh polymorphic variable in the typing context. In \Rule{Inst}, we
specialize a type scheme $\tfor \tv \ts$ to $\ts \where{\tv \is \t}$,
substituting $\tv$ for an arbitrary monotype $\t$.

\parcomment {Let rule}

Let-polymorphism is handled by the \Rule{Let} rule, where a
\textit{polymorphic} term can be bound. This allows a single definition to be
instantiated differently at each use site---an essential feature of \ML. In
this rule, the term $\eone$ has a polymorphic type scheme $\ts$, adds $\x :
\ts$ into the context $\G$ to typecheck $\etwo$.

\parcomment {Annotations}

Annotations $(e : \exi \tvs \t)$ ensures that the type of $e$ is (an instance
of) the type $\t$. The type variables $\tvs$ are \emph{flexibly} (or
existentially) bound in $\t$, meaning that $\tvs$ may be unified with some
types $\tys$ to produce a well-typed term. For instance, the term $(\efun x x
+ 1 : \exi \tv \tv \to \tv)$ is well-typed with $\tv := \tint$ in
\Rule{Annot}.

\parcomment {Contextual rules (Poly-*, Use-*, Proj-*)}

\paragraph{Polytypes and overloaded tuples}
The typing rules for fully annotated terms ($\e^x$) are unsurprising.
However, typing rules for terms with omitted type annotations are
non-compositional as they depend on a surrounding one-hole context
$\E$. Hence, they assert that the typability of the expression $\G \th \E
\where {\e^i}: \t$ where $\e^i$ is an expression with an implicit type
annotation.
%
We first request a typing for the expression with an explicit annotation $\G
\th \E \where {\e^x}: \t$ where $\e^x$ is a fully annotated variant of $\e^i$.
We then request that (the shape of) the annotation is fully determined from
context, either from the type of the expression, which we write $\eshape \E
\e \sh$, or from the type of the hole, which we write $\Eshape \E \e \sh$.

In order to describe the judgments $\eshape \E \e \sh$ and $\Eshape \E \e \sh$,
we introduce a \emph{typed hole} construct $\emagic \e$ that allows any
well-typed expression $\e$ to be treated as if it had any type. That is the
typing rule for holes is:
\begin{mathpar}
  \inferrule[Magic]
    {\G \th \e : \t}
    {\G \th \emagic \e : \tp}
\end{mathpar}
\TODO{This rule should go in a figure dedicated to the various magic rules.}
Typed holes are not allowed on source terms and are just a device for
the definition of non-ambiguous shapes.  Finally, we define what it means for a
shape to be determined from the type of a context or an expression:
\begin{mathpar}
\def \Eqdef {&\eqdef&}
{\begin{tabular}{RCL}
\eshape E \e \sh \Eqdef
  \forall \G, \t, \gt, \uad
  \G \th \eerase {\E \where {\emagic {\eannot \e {} \gt }}} : \t
      \wide\implies \shape \gt = \sh
\\[1ex]
\Eshape E \e \sh \Eqdef
  \forall \G, \t, \gt, \uad
      \G \th \eerase {\E\where{\eannot {\emagic \e} {} \gt}} : \t
      \wide\implies \shape \gt = \sh
\end{tabular}}
\end{mathpar}
These states that the shape $\sh$ of expression $\e$ in context $\E$ is
determined by the expression $\e$, in the former case, or by the context
$\E$ in the latter case. Just like constraints, we must erase implicit constructs
in the term that have not yet been elaborated, written $\eerase \e$ (defined in
??).

%% Shapes are equal modulo alpha equivalence and the removal of useless
%% polymorphic type variables. They do not have useless existential variables.

% Expression-based implicit rules

The implicit rule \Rule{Proj-I} types the projection $\eproj \e j$ provided the
context $\E$ \emph{infers} that the shape of $\e$ must be a tuple with arity $n$.
Similarly, \Rule{Use-I} permits instantiating a polytype in $\einst e$ if
the context $\E$ infers that the type of $\e$ must be a polytype with shape
$\any \tvcs \tpoly \ts$. The rule \Rule{Poly-I} types the implicit boxing
construct $\epoly \e$ by \emph{checking} the expected type of $\epoly \e$ in the
context $\E$ is a polytype with the shape $\any \tvcs \tpoly \ts$. This rule
differs from the previous two as the shape is determined by the expected type
within the context as opposed to the inferred type of $\e$.


% Labels

\paragraph{Overloaded record labels}
% Contextual rules
We adopt a similar non-compositional approach for elaborating overloaded
labels, whether in record projection ($\efield e \elab$) or record
construction ($\erecord {\overline{\elab = \e}}$), although the definitions
are slighly more involved.  Here, a one-hole label context $\Lab$ provides the
surrounding context in which a label $\elab$ may appear:
\begin{mathpar}
\begin{bnfgrammar}
\entry [Label contexts]{\Lab}{
    \E \where {\e.\square} \and
    \E \where
         {\erecord
            {\ela = \eone; \ldots; \square = \ei ; \ldots; \el_n = \en}
         }
}
\end{bnfgrammar}
\end{mathpar}

As with our contextual rules for expressions, we define two rules for labels.
\Rule{Lab-X} handles explicitly annotated labels $\elab / \T$ by instantiating
the type scheme $\tfor \tvs \t \to \tvs \T$ associated with $\elab$ in label
environment $\labenv$. \Rule{Lab-I} handles unannotated labels by elaborating
$\elab$ to $\elab / \T$ if the context $\Lab$ uniquely infers the record type
$\T$ for $\elab$ and the resulting elaboration is well-typed.

The unicity of the inferred record type is captured by the judgement $\Lshape
\Lab \elab \T$. The definition fits into the framework we established for
expressions above by introducing a label cast operator $\elcast \elab \t$,
analogous to the combination of typed holes and annotations in expressions,
which asserts that the label is instantiated to the type $\tp \to \t$ for any
$\tp$.
\begin{mathpar}
  \infer[Lab-Magic]
    {}
    {\G \th \elcast \elab \t : \tp \to \t}
\\
\Eshape \Lab \elab \T \Wide\eqdef
   \forall \G, \t, \gt , \uad
     \G \th \eerase {\Lab[\elcast \elab \gt]} : \t
	\implies \shape \gt= \any \tvcs \tvcs \T
\end{mathpar}

% Rules
\Rule{Rcd} types a record $\erecord {\overline{\el = \e}}$ as a record type
$\t$ provided that each field assignment $\el = \e$ can be assigned the record
type $\t$.
\Rule{Rcd-Assn} checks that $\e$ has the appropriate field type in $\el = \e$
and returns the instantiated record type $\tp$ for the label $\el$.
\Rule{Rcd-Proj} types the projection $\efield \e \el$ by checking that the
type of $\e$ matches the record type associated with label $\el$, returning
the field type $\t$.

% Closed world reasoning
Both \Rule{Rcd} and \Rule{Rcd-Proj} impose additional constraints on their
record types to support \emph{closed-world} reasoning. These constraints
exploit the uniqueness of type definitions in the global label environment
$\labenv$ to resolve overloaded labels:
\begin{enumerate*}
\item
  in a record projection $\efield \e \elab$, if the label $\elab$ is not
  overloaded, then the global record typing context $\labenv$ assigns a
  unqiue record type $\T$ to $\elab$;

\item
  in a record expression $\erecord {\ela = \eone; \ldots; \el_n =
  \en}$, if the set of labels ${ \ela, \ldots, \el_n }$ uniquely
  identifies a record type $\T$ in the typing context $\labenv$, then
  we can assign this type to the record expression.
\end{enumerate*}

We formalize this with the judgement $\G \th \bar \el \uni \t$, which
either:
\begin{enumerate*}
  \item enforces $\t$ to be of the form $\tys \T$ if the labels $\bar \el$
    uniquely identify a nominal record type $\T$ in $\labenv$ (\Rule{Lab-!}),
    or
  \item imposes no constraint on $\t$ in the ambiguous case
    (\Rule{Lab-?}).
\end{enumerate*}

Label declarations in $\labenv$ have the form $\elab : \tfor \tvs \tp \to \tvs
\T$, assigning labels to field types and record types\footnote{For a given
record type $\T$, we assume each label associated with it is unique.}. We
write $\elab / \T \in \labenv$ if such a declaration of $\elab$ exists for the
record type $\T$. This membership relation extends to explicitly annotated and
casted labels:
\begin{mathpar}
  \infer[Lab-$\in$X]
    {\elab / \T \in \labenv}
    {(\elab / \T) / \T \in \labenv}

  \infer[Lab-$\in$Cast]
    {\elab / \T \in \labenv}
    {\elcast \elab \t / \T \in \labenv}
\end{mathpar}
We then define the uniqueness predicate $\bar \el \uni \T \in \labenv$ as:
\begin{mathpar}
  \infer[Lab-U]
    {\bar \el / \T \in \labenv \\
     \all {\T'} \uad\bar \el / \T' \in \labenv \implies \T = ~\T'}
    {\bar \el \uni \T \in \labenv}
\end{mathpar}
This states that the set of labels $\bar \el$ determines a unique nominal type
$\T$ in $\labenv$ if no other type $\T'$ can be associated with the same label
set.
\TODO{Should the rules above be in a figure as well?}

\subsection {Examples of typings}

The following lemma shows that we can always take a larger context
$\E$ or $\Lab$ for implicit rules \Rule {Proj-I}, \Rule {Use-I}, \Rule {Poly-I}
and \Rule{Lab-I}.
That is, there is always a derivation using only toplevel contexts.
\begin{lemma}
\label{lem/context/largest}
\newcommand {\Eab}{\parens{\Ea\where \Eb}}
If $\eshape \Eb \e \sh$, then $\eshape \Eab \e \sh$. Similarly, for label
contexts, if $\Lshape \Lab \elab \T$, then \\$\Lshape {(\E[\Lab])} \elab \T$.
\end{lemma}

% Examples
We now illustrate the typing of implicit constructs with a few examples.
\begin{example}
To illustrate a simple case of non-typability, we show that the expression $e$
equal to $\efun \x {\eproj \x k}$ is ambiguous, \ie that it does not
typecheck.
%
%% Let $e_n$ be the explicitly annotated version $\efun r
%% {\eproj[n] \x i}$. for $n \le k$.
If there is a derivation of $\efun \x
{\eproj \x k}$ then there must be one of the form:
\begin{mathpar}
\infer*[Right=Proj-I]{
                  \eshape \E \x {\any \tvcs \Pi\iton \tvcs} \\
                \eset \th \E \where {\eproj[n] \x k} : \t
}{%             -------------------------------------------
                  \eset \th \E \where {\eproj \x k} : \t
}
\end{mathpar}
where $E$ is the term $\efun \x \ehole$, which is the largest possible
context, thanks to~\cref {lem/context/largest}.
%
Let $\t$ be $\Pi\iton \ti \to \t_k$ for some $n \geq k$.  We have the
following derivation:
\begin{mathpar}
\infer* [Right=Fun]{
          \infer*[Right=Proj-X]
                {\x : \Pi\iton \ti \th \x : \Pi\iton \ti}
                {\x : \Pi\iton \ti \th \eproj[n] \x k : \t_k}
}{%      ----------------------------------------------------------
         \eset \th \E \where{\eproj[n] \x k} : \t
}
\end{mathpar}
Unfortunately, $\eshape \E  \x {\any \tvcs \Pi\iton \tvcs}$ does not hold.
  Indeed, we have $\eset \th \E \where {\emagic {\eannot \x {} {\gt}}} : \t$
for any $\gt$ assuming $\t$ is of the form $\gt \to \tp$.
Hence, $\any \tvcs \Pi\iton[n] \tvcs$ and $\any \tvcs \Pi\iton[n+1] \tvcs$
are two possible shapes for the type of $\x$.
\end{example}

\begin{example}
\locallabelreset
We now illustrate a non-ambiguous example, showing that the
expression $e$ equal to $\th \eapp {(\efun \x {\eproj
\x  1})} {(1, 2)} : \tint$.
%
%
Let $\E$ be the context $\eapp {(\efun \x \ehole)} {(1, 2)}$.  We
have the derivation:
\begin{mathpar}
\infer* [Right=Proj-I]{
		\eshape \E \x {\any {\tvca, \tvcb} \tvca \tprod \tvcb} \\
                \eset \th \E \where {\eproj[2] \x 1} : \tint
}{%             -------------------------------------------
                  \eset \th \E \where {\eproj \x 1} : \tint
}
\end{mathpar}
We have $\eset \th \E \where {\eproj[2] \x 1} : \tint$, indeed. Therefore, it
just remains to show $\eshape \E \x {\any {\tvca, \tvcb} \tvca \tprod \tvcb}$~\llabel C.
Assume $\eset \th \E \where{\emagic {\eannot \x {} \gt}} : \t$. Since $\x : \tint \tprod \tint$
is bound in the context at the hole in $\E$,
there is no other choice but take $\gt$ equal to $\tint \tprod \tint$,
hence $\shape \gt = \any {\tvca, \tvcb} \tvca \tprod \tvcb$, which proves~\lref C.
\end{example}

The following example of non-typability illustrates how the typing rules
still forces to reject typing of some expressions whose elaboration would be
unambiguous. This is intended, to prevent us from having to focus at several
terms simultaneously. Our typing rules enforce the resolution of
shape inference, locally, one at a time.

\begin{example}
\newcommand{\tyid}{\ty_{\kwd{id}}}
  \newcommand{\eid}{\efun z z}
\newcommand {\epid}[1][]{\epoly[#1]{\eid}}
Let $\tyid$ be $\tpoly{\all \tv \tv \to \tv}$.
%
We show that the expression $e$ equal to $\elet \x {\epoly {\efun z z}}
{(\eapp {\einst \x} 1, \eapp {\einst \x} \eunit)}$ is rejected as ambiguous.
Let $\tyid$ be $\tpoly {\all \tv \tv \to \tv}$.  Clearly, we have $\elet \x
{\epoly [\tyid] {\efun z z}} {(\eapp {\einst[\tyid] \x} 1, \eapp
{\einst[\tyid] \x} \eunit)}$.  This is actually the only possible fully
annotated derivation.
%
To show that $e$ is typable, we must be able to make all annotations
optional, sequentially.  Therefore, the final step, which will eliminate the
last annotation has a single point of focus of the form $\E\where{e^i}$,
where $\e^i$ can be any of the three positions with a missing annotation.  We
consider each case independently, and show that it is actually not typable.
  \begin{itemize}
\proofcase
{$\E$ is $\elet \x \ehole (\eapp {\einst \x} 1, \eapp {\einst \x}\eunit)$}
%
If this holds, we should have a derivation that ends with
\begin{mathpar}
\infer*[Right=Poly-I]{
		  \Eshape \E \eid {\tpoly \tyid} \\
                  \eset \th \E \where {\epid [\tyid]}: \t
}{%               ---------------------------------------
                       \eset \th \E \where \epid : \t
}
\end{mathpar}
However, $\Eshape \E \eid {{\tpoly \tyid}}$ does not hold.
Indeed, the following judgment
$\eset \th \E \where {\eannot {\emagic \eid} {} {\tpoly \ts}} : \t$ holds, where
$\ts$ is either $\tfor \tv \tv \to \tv$ or $\tfor \tv \tv \to
\tv\to\tv$. Hence, the shape of the type of $\eid$ is not uniquely
determined and this case cannot occur.

\proofcase
{$\E$ is
    $\elet \x {\epid} {\eapp {\einst \ehole} 1, \eapp {\einst \x} \eunit}$}
%
The derivation must end with:
\begin{mathpar}
\infer*[Right=Proj-X]{
		  \eshape \E \x {\tpoly \tyid} \\
                \eset \th \E \where {\einst[\tyid] \x} : \t
}{%             -------------------------------------------
                    \eset \th \E \where {\einst \x} : \t
}
\end{mathpar}
However, $\eshape \E \x \tyid$ does not hold (the proof is similar to the
previous case).

\proofcase {$\E$ is  $\elet \x \epid {(\eapp {\einst \x} 1, \eapp
  {\einst \ehole} \eunit)}$} This is symmetric to the previous case, which cannot
hold either.
  \end{itemize}
\end{example}

\begin{example}
Let $\e$ be $\elet f {\efun \x {\eproj \x 1}} {\eapp f (1, 2)}$.
$\e$ is well-typed using \emph{backpropagation}.
$\e$ is of the form $\E \where {\x}$ where  $\E$ is the context $\elet f
{\efun \x \ehole} {\eapp f (1, 2)}$.
We have $\eset \th \E \where {\eproj[2] \x 1} : \tint$.
Let us show that $\eshape \E \x {\any {\tvca, \tvcb} \tvca \tprod \tvcb}$.
%
Assume $\eset \th \E \where {\emagic {\eannot \x {} \gt} } : \t$. As $\gt$ is a ground
type, the type $\gt$ of $\x$ is not a variable.  Then, $\gt$ cannot be that
of an arbitrary sized tuple, since there is no such type for a tuple of
arbitrary size. Hence, $\gt$ must be a tuple $\Pi\iton \tys$ for some size
$n$. Since the codomain of $f$ must be a tuple of size~$2$ (for $\eapp f (1,
2)$ to be well-typed), then $n$ must also be $2$. This shows that $\eshape \E
\x {\any {\tvca, \tvcb} \tvca \tprod \tvcb}$.
%% \begin{mathpar}
%%   \infer
%%     {
%%     \infer
%%       {
%% 	\infer
%% 	  {
%% 	    \infer
%% 	      {
%% 		\infer
%% 		  {}
%% 		  {\tva, \tvb, x : \tva \th x : \tva}}
%% 	      {\tva, \tvb, x : \tva \th \ecast x \tva \tvb : \tvb}}
%% 	  {\tva, \tvb \th \efun x {\ecast x \tva \tvb : \tva \to \tvb}}}
%%       {\emptyset \th \efun \x \ecast \x \tva \tvb : \tfor {\tva, \tvb} \tva \to \tvb} \\
%%     \infer
%%       {\ldots}
%%       {f : \tfor {\tva, \tvb} \tva \to \tvb \th \eapp f (1, 2) : \tunit}}
%%     {\emptyset \th \elet f {\efun \x {\ecast \x \tva \tvb}} {\eapp f (1, 2)} : \tunit}
%% \end{mathpar}
\end{example}


\subsection{Constraint generation}
\label{sec:constraint-gen}

% Intro
We now present the formal translation from terms $\e$ to constraints $\c$,
such that the resulting constraint is satisfiable if and only if the term is
well typed. The translation is defined as a function $\cinfer \e \t$, where $\e$
is the term to be translated and $\t$ is the expected type of $\e$.

% Explanation of expected type
The expected type $\t$ is permitted to contain type variables, which can be
existentially bound in order to perform type inference. The models of constraint
$\cinfer \e \t$ interpret the free variables of $\t$ such that
$\t$ becomes a valid type of $\e$. For example, to infer the entire type of $\e$
we may pick a fresh type variable $\tv$ for $\t$.
\paragraph{Pattern constraints}

Thus far, our formal presentation of constraint patterns has remained
abstract, deliberately leaving the syntax and semantics of patterns unspecified to
accommodate a range of language features. We now concretize this by specifying
the patterns used in \OML (See \cref{fig:patterns-oml}), and introducing the
corresponding constraints for the variables they bind.
%
Patterns include:
\begin{enumerate*}

  \item Tuple patterns $\cpatprod \tv j$, matching a tuple type $\Pi\iton
    \tys$ of arity $n \geq j$, and binding the $j$-th component to $\tv$.

  \item Nominal patterns $\cpatrcd \ct$, binding the name of a nominal type
    $\tys \Tapp$ to the nominal variable $\ct$.

  \item Polytype patterns $\cpatpoly \cscm$ matching a polytype $\tpoly \ts$ and
    binding the resulting scheme to the variable $\cscm$.

\end{enumerate*}

Each new constraint has an unsubstituted form ($\cscm \leq \t, \x \leq \cscm$
\etc), whose semantics is defined via substitution into a sugared form ($\ts
\leq \t, \x \leq \ts,$ \etc). Semantic environments $\semenv$ are extended to
interpret nominal variables $\ct$ as names $\T$ and scheme variables $\cscm$ as
ground type schemes $\gscm$, that is type schemes with no unbound variables
(\ie $\tfor {\fvs \t} \t$).

\begin{mathparfig}
  {fig:patterns-oml}
  {Patterns for \OML}
  \begin{bnfgrammar}
   \entry[Patterns]{\cpat}{
      \cpatprod \tv j
      \and \cpatrcd \ct
      \and \cpatpoly \cscm
    } \\
    \entry[Constraints]{\c}{
      \dots
      \and \labenv(\elab/\ct) \leq \tone \to \ttwo
      \and \labenv(\elab/\T) \leq \tone \to \ttwo
      \andcr \cscm \leq \t
      \and \ts \leq \t
      \andcr \x \leq \cscm
      \and \x \leq \ts
    }
 \end{bnfgrammar}
  \\
  \newcommand{\Mrule}[4][]{{#2} \Matches {#3} &\eqdef& {#4} & #1}
  \begin{tabular}{RCLL}
    \Mrule[\text{if } n \geq j]
      {\cpatprod \tv j}
      {\shapp[\any \tvcs \Pi\iton \tvcs] \tys}
      {[\tv \is \ty_j]}
    \\[1ex]
    \Mrule
      {\cpatrcd \ct}
      {\shapp[\any \tvcs \tvcs \Tapp] \tys}
      {[\ct \is ~ \T]}
    \\[1ex]
    \Mrule
      {\cpatpoly \cscm}
      {\shapp[\any \tvcs \tpoly \ts] \tys}
      {[\cscm \is \ts \where{\tvcs \is \tys}]}
  \end{tabular}
  \\
  \inferrule[Lab-Inst]
    {\semenv \th \labenv(\elab/\semenv(\ct)) \leq \tone \to \ttwo}
    {\semenv \th \labenv(\elab/\ct) \leq \tone \to \ttwo}

  \inferrule[Scm-Inst]
    {\semenv \th \semenv(\cscm) \leq \t}
    {\semenv \th \cscm \leq \t}

  \inferrule[Abs-Inst]
    {\semenv \th \x \leq \semenv(\cscm)}
    {\semenv \th \x \leq \cscm}
  \\
  \newcommand{\Srule}[3][]{{#2} &\eqdef& {#3} & {#1}}
  \begin{tabular}{RCLL}
    \Srule[\text{if } \labenv(\elab/\T) = \tfor \tvs \t \to \tvs \Tapp]
      {\labenv(\elab/\T) \leq \tone \to \ttwo}
      {\cexists \tvs \cunif \tone \t \cand \cunif \ttwo {\tvs \Tapp}}
    \\[1ex]
    \Srule
      {(\tfor \tvs \tp) \leq \t}
      {\cexists \tvs \cunif \tp \t}
    \\[1ex]
    \Srule
      {\x \leq (\tfor \tvs \t)}
      {\cfor \tvs \capp \x \t}
  \end{tabular}

\end{mathparfig}


% Explanation of constraint gen cases
% Simple constraint gen

\paragraph{Constraint generation}

\begin{mathparfig}
  {fig:constraint-gen}
  {The constraint generation translation for \OML}
\newcommand {\Crule}[2]{#1 &\eqdef& #2}
\def \arraystretch{1.2}%4
\begin{tabular}{LCL}
\Crule
   {\cinfer x \t}
   {\cinst x \t}
\\
\Crule
  {\cinfer {()} \t}
  {\cunif \t \tunit}
\\
\Crule
  {\cinfer {\efun \x \e} \t}
  {\cexists {\tva, \tvb} \cunif \t {\tva \to \tvb}
    \cand \clet \x \tvc {\cunif \tvc \tva} {\cinfer \e \tvb}}
\\
\Crule
  {\cinfer {\eapp \eone \etwo} \t}
  {\cexists {\tva} \cinfer \eone {\tva \to \t} \cand \cinfer \etwo \tva}
\\
\Crule
  {\cinfer {\elet \x \eone \etwo} \t}
  {\clet \x \tva {\cinfer \eone \tva} {\cinfer \etwo \t}}
\\
\Crule
  {\cinfer {\eannot \e \tvs \tp} \t}
  {\cexists \tvs \cunif \t \tp \cand \cinfer \e \tp}
\\
\Crule
  {\cinfer {\etuple {\eone, \ldots, \en}} \t}
  {\cexists \tvs \cunif \t {\Pi\iton \tvs}
    \cand \cAnd \iton \cinfer \ei {\tv_i}}
\\
\Crule
  {\cinfer {\exfield \e j n} \t}
  {\cexists {\tvbs}
    \cinfer \e {\Pi\iton \tvbs}
    \cand \cunif \t {\tvb_j}}
\\
\Crule
  {\cinfer {\efield \e j} \t}
  {\cexists \tv \cinfer \e \tv
    \cand \cmatch \tv {\cbranch {\cpatprod \tvb j} {\cunif \t \tvb}}}
\\
\Crule
  {\cinfer {\expoly \e {\exi \tvs \ts}} \t}
  {\cexists {\tvs}
    \cinfer \e \ts
    \cand \cunif \t {\tpoly \ts}}
\\
\Crule
  {\cinfer {\exinst \e \tvs \ts} \t}
  {\cexists {\tvs}
    \cinfer \e {\tpoly \ts}
    \cand \ts \leq \t}
\\
\Crule
  {\cinfer {\einst \e} \t}
  {\cexists \tva
    \cinfer \e \tva
    \cand \cmatch \tva {\cbranch {\cpatpoly \cscm} \cscm \leq \t}}
\\
\Crule
  {\cinfer {\epoly \e} \t}
  {\clet \x \tv {\cinfer \e \tv}
    {\cmatch \t {\cbranch {\cpatpoly \cscm} {\x \leq \cscm}}}}
\\
\Crule
  {\cinfer {\efield \e \el} \t}
  {\cexists \tv \cinfer \e \tv
    \cand \cinferlabuni \el \tv
    \cand \cinferlab \elab \t \tv}
\\
\Crule
  {\cinfer {\erecord {\overline{\el = \e}}} \t}
  {\cinferlabuni {\bar \el} \t
    \cand \cAnd \iton \cinferassn \eli \ei \t}
\\
\Crule
  {\cinfer {\emagic \e} \t}
  {\cexists \tv \cinfer \e \tv}
\\
\Crule
  {\cinfer \e {\tfor \tvs \t}}
  {\cfor \tvs \cinfer \e \t}
\\ \\
\Crule
  {\cinferassn \el \e \t}
  {\cexists \tv \cinfer \e \tv
    \cand \cinferlab \el \tv \t}
\\
\Crule
  {\cinferlab \elab \tone \ttwo}
  {\cmatch \ttwo {\cbranch {\cpatrcd \ct} {\labenv(\elab/\ct) \leq \tone \to \ttwo}}}
\\
\Crule
  {\cinferlab {\elab/\T} \tone \ttwo}
  {\labenv(\elab/\T) \leq \tone \to \ttwo}
\\
\Crule
  {\cinferlab {\elcast \elab \t} \tone \ttwo}
  {\cunif \t \ttwo}
\\
\Crule
  {\cinferlabuni {\bar \el} \t}
  {\begin{cases}
    \cexists \tvs \cunif \t {\tvs \Tapp} &\text{if } \bar \el \uni \T \in \labenv \\
    \ctrue &\text{otherwise}
   \end{cases}}
\\ \\
\Crule
  {\csem {\enil \th \e : \t}}
  {\cinfer \e \t}
\\
\Crule
  {\csem {\tv, \G \th \e : \t}}
  {\call \tv {\csem {\G \th \e : \t}}}
\\
\Crule
  {\csem {x : \ts, \G \th \e : \t}}
  {\clet \x \tv {\ts \le \tv} {\csem {\G \th \e : \t}}}
\\
\end{tabular}
\end{mathparfig}


The function $\cinfer - {\mathop{=}}$ is defined in \cref{fig:constraint-gen}.
All generated type variables are fresh with respect to the expected type $\t$,
ensuring capture-avoidance.
%
Unsurprisingly, variables generate an instantiation constraint. Unit $()$
requires the type $\t$ to be $\tunit$. A function generates a constraint that
binds two fresh flexible type variables for the argument and return types.  We
use a $\Let$ constraint to bind the argument in the constraint generated for
the body of the function. The $\Let$ constraint is monomorphic since $\tvc$ is
fully constrained by type variables defined outside the abstraction's scope
and therefore cannot be generalized. Applications introduce a fresh flexible
type variable for the argument type and ensures $\t$ is the return type of the
function. Let-bindings generates a polymorphic let constraint; $\cabs \tv
{\cinfer \e \tv}$ is a principal constraint abstraction for $\e$: its intended
interpretation is the set of all types that $\e$ admits.

% New constraint gen cases
% Annotations, tuples
Annotations bind their flexible variables and enforce the equility of
the annotated type $\tp$ and the expected type $\t$. Tuples introduce
fresh variables for each component and unify their product with $\t$.
Explicit projections ensure $\e$ has a tuple type $\Pi\iton \tvbs$
and extract the $j$-th component $\tvb_j$, unifying it with $\t$.
Implicit projections defer this via a suspended match constraint, until
the shape of $\e$'s expected type is known to be a tuple, extracting the
$j$-th component with the pattern $\cpatprod \tvb j$,

% Polytypes
For polytypes, boxing asserts that $\e$ has the polymorphic type $\ts$ (using
universal quantification) and that the expected type is the polytype $\tpoly
\ts$. Unboxing suspends until the inferred type of $\e$ is known to be a
polytype, captured by the pattern $\cpatpoly \cscm$, at which point we require
$\t$ to be an instance of $\cscm$. Explicit unboxing is analogous, but uses an
explicit scheme $\ts$ and therefore does not require a suspended match
constraint. Implicit boxing infers the principal type for $\e$ using a $\Let$
constraint and suspends until the expected type of the entire term is known to
be a polytype, bound to $\cscm$. We then assert that the principal type of $\e$
is at least as general as $\cscm$, via the constraint $\x \leq \cscm$.


% Records
Record projections generate a fresh variable for the nominal record type,
constraining $\e$ to this type, and use the auxiliary function $\cinferlab \el
\tone \ttwo$ to instantiate the label. The function $\cinferlabuni {\bar \el}
\t$ checks whether a label sequence $\bar \el$ uniquely determines a record
type, unifying $\t$ with $\tvs \Tapp$ if so, or leaving it unconstrained if
ambiguous. This function enables closed-world reasoning for both projections
and constructions, and corresponds to the judgement $\G \th \bar \el \uni \t$
judgement defined in \cref{sec/language/typing-rules}.

Record construction checks label uniqueness and generates a per-field
constraint $\eli = \ei$, introducing a fresh variable $\tv$ for each
field's type and ensuring that $\e$ has this type and the label $\el$
instantiates to $\tv \to \t$.

Label instantiation constraints $\cinferlab \elab \tone \ttwo$ suspend
until $\ttwo$ is known to be a record type; once resolved, the label type is
looked up in $\labenv$ and instantiated. Explicit instantiations bypass
suspension and directly instantiate the label's type.

% Other cases
The remaining (greyed) cases \TODO{Make the cases grey} correspond to internal
constructs used in \OML's typing rules and are included for completeness.

% Soundness/completeness of constraint gen

The translation is sound and complete with respect to the typing judgment.
That is to say, the term $e$ is typable with $\ts$ if and only if
$\cinfer \e \ts$ is satisfiable.
%
\begin{theorem}{(Constraint generation is sound and complete)}
$\G \th \e : \ts$ iff\/
$\centails \csem {\G \th \e : \ts}$.
\end{theorem}
\TODO{This result should go in the Metatheory section.}

\subsection{Metatheory}
\label{sec:constraint-prop}


\paragraph{Stability by program transformations}

A well-known property of \ML is to admit principal types---and the design of
fragile \ML extensions is all about the preservation of principal types.
However, core \ML has several other key properties:
\SetLabelAlign{mydesc}{#1:\hfill}
\begin{description}[font=\it,align=mydesc,topsep=1ex,itemsep=1ex,leftmargin=0ex]
\newcommand {\eswap}{\mathprefix  {swap}}

\item [Compositionality]
  If $\G \th \E \where \e : \t$ then there exists $\Gp$ and $\tp$ such that
  $\Gp \th \e : \tp$ and for all term $\ep$ such that $\Gp \th \ep : \tp$,
  then $\G \th \E \where \ep : \t$.

\item [Factorization]
  If $\G \th \e \where {\x \is \ez} : \t$ and $\G \th \ez : \ts$ then
  $\G \th \elet \x \ez \e : \t$.

%% \item [Inlining]
%%   If $\G \th \elet \x \ez \e : \t$ then $\G \th \e \where {\x \is \ez} : \t$

\item [Application equitypability]
  The expressions $\eappp f \ea \eb$ and $\eappp {\eswap f} \eb \ea$ are
  equitypable where $\eswap$ is $\efun f {\efun \xa {\efun \xb {\eappp f \xb
  \ea}}}$.  Thanks to a few other properties, this implies in turn that
  $\eapp f \ea$ and $\eApp f \ea$ and $\eb \ePipe \ea$ are themselves
  equi-typable where are the application function $\efun f \efun \x \eapp f
  \x$ and the reverse application function $\efun \x \efun f \eapp f \x$,
  respectively.

\end{description}
It is well-known that bidirectional types inference breaks application
equi-typability. However, both \Geninst-directional and omnidirectional type
inference preserve it.

In core \ML, Factorization is a consequence of principal types and
compositionality.  It is definitely broken by overloading, since the inlined
version may has several overloading sites that may be resolved differently.
However, we may wonder about linear factorization, \ie when $\x$ occurs
exactly once in~$\e$.  This is again broken by bidirectional type inference,
but preserved by both \Geninst-directional and omnidirectional type inference,
although for different reasons: for \Geninst-directional type inference, it
follows from the limited use of contextual type information; for
omnidirectional type inference, it follows by backpropagation\Xdidier{I
think this is not the right term}.

\Xdidier {How formal should we be? Should we justify these claims?}

\section{Solving constraints}
\label{sec:solving}

% Intro
We now present a machine for solving constraints in our language. The solver
operates as a rewriting system on constraints $\c \csolve \cp$. Once no further
transitions are applicable, \ie $\c \cnsolve$, the constraint $\c$ is either in
solved form---from which we can read off a most general solution---or the
solver becomes stuck, indicating that $\c$ is unsatisfiable.

% Contexts
% Treatment of $alpha$-equivalence in contexts
% Membership of constraints

% Unification
\subsection{Unification}
%
Our constraints ultimately reduce to equations between types, which we solve
using first-order unification. Like our solver, we specify unification as a
non-deterministic rewriting relation between \emph{unification problems} $\upa
\unif \upb$.

\begin{bnffig}
  {fig:unification-syntax}
  {Syntax of unification problems. Constraints are also extended with the
  administrative multi-equation construct.}

  \entry[Unification problems]{\up}{
    \ctrue \and \cfalse \and \upa \cand \upb \and \cexists \tv \up \and \ueq
  } \\
  \entry[Multi-equations]{\ueq}{
    \eset \mid \cunif \t \ueq
  } \\
  \entry[Constraints]{\c}{
    \dots \and \ueq
  }
\end{bnffig}

% Multi-equations

Unification problems $\up$ are a restricted subset of constraints.
To enable an efficient presentation of unification, we adopt
\emph{multi-equations} \citep{Pottier-Remy/emlti}---a multi-set of types
considered equal. These generalize binary equalities and their semantic
interpretation is given by:
\begin{mathpar}
  \infer[Multi-Unif]
    {\all {\t \in \ueq}\, \semenv(\t) = \gt}
    {\semenv \th \ueq}
\end{mathpar}
That is, $\semenv$ satisfies $\ueq$ if all members of $\ueq$ are mapped to a
single ground type $\gt$ by $\semenv$. We consider multi-equations equal
modulo permutation of their members.

% The spec

The unification rules are listed in
\cref{fig:omni-unification-algorithm}. Rewriting proceeds modulo
$\alpha$-equivalence, as well as associativity and commutativity of
conjuctions, and takes place under an arbitrary unification problem context
$\Up$.
%
Our algorithm is largely standard \cite{Pottier-Remy/emlti} but replaces
type constructors with \emph{canonical principal shapes}, enabling a uniform
treatment of monotypes and polytypes within unification, simplifying the handling of polytypes, as found
in \citep{Garrigue-Remy/poly-ml}.

% Explaination of rules

We briefly summarize the purpose of each rule below: Ruke
\Rule{U-Exists} lifts existential quantifiers to the top of the
unification problem, enabling applications of rules \Rule{U-Merge} and
\Rule{U-Cycle} as all multi-equations eventually part of a single
conjunction. Rule \Rule{U-Merge} combines mutli-equations sharing
a common variable. Rule \Rule{U-Stutter} remxoves duplicate
occurrences of a variable. Rule \Rule{U-Decomp} decomposes equal types
with matching shapes into equalities between their subcomponents. Rule
\Rule{U-Clash} complements Rule \Rule{U-Decomp}, handling shape
mistmatches that result in failure.  Rule \Rule{U-Name} introduces
a fresh variable to expose a subcomponent of a shape, ensuring
unification over \emph{shallow terms}, making sharing of type
variables explicit and avoids the need for copying types in rules such
as \Rule{U-Decomp}. Rules \Rule{U-True}, \Rule{U-False} simplify conjunctions.
Rule \Rule{U-Trivial} occurs when a multi-equation $\ueq$ is trivial, that is, when it is either empty
($\eset$) or contains only a single type ($\cunif \t \eset$).

\TODO{Replace $\pshapp \tys[\ti]$ by $\pshapp (\tys, \t_i, \tys')$.}

\TODO{Replace $\trivial \ueq$ with $|\ueq| \leq 1$.}

Rule \Rule{U-Cycle} implements the \emph{occurs check}, ensuring that a type
variable does not occur in the type it is being unified with. This is a
necessary condition for unification, as it would otherwise lead to infinite
types\footnote{We discuss relaxing this constraint in \cref{sec/rec-types}.}
is defined by the relation $\tv \prec_\up \tvb$ indicating that $\tv$ occurs
in a type assigned to $\tvb$ in $\up$. A unification problem is cyclic,
written $\cyclic \up$, if $\tv \prec_\up^* \tv$ for some $\tv$.

\begin{mathparfig}[t]
  {fig:omni-unification-algorithm}
  {Unification algorithm as a series of rewriting rules
   $\upa \unif \upb$. All shapes are principal.}
   \rewrite[U-Exists]
      {(\cexists \alpha \upa) \cand \upb \\ \tv \disjoint \upb}
      {\cexists \tv {\upa \cand \upb}}

    \rewrite[U-Cycle]
      {\up \\ \cyclic \up}
      {\cfalse}

    \rewrite[U-True]
      {\up \cand \ctrue}
      {\up}

    \rewrite[U-False]
      {\up \cand \cfalse}
      {\cfalse}

    \rewrite[U-Merge]
      {\cunif \tv \ueqa \cand \cunif \tv \ueqb}
      {\cunif \tv {\cunif \ueqa \ueqb}}

    \rewrite[U-Stutter]
      {\cunif \tv {\cunif \tv \ueq}}
      {\cunif \tv \ueq}

    \rewrite[U-Name]
      {\cunif {\pshapp \tys[\ti]} \ueq \\ \tv \disjoint \tys, \ueq }
      {\cexists \tv {\cunif \tv \ti \cand \cunif {\pshapp \tys[\tv] } \ueq}}

    \rewrite[U-Decomp]
      {\cunif {\pshapp \tvs} {\cunif {\pshapp \tvbs} \ueq}}
      {\cunif {\pshapp \tvs} \ueq \cand \cunif \tvs \tvbs}

    \rewrite[U-Clash]
      {\cunif {\pshapp \tvs} {\cunif {\pshapp[\shp]\tvbs } \ueq }\\
       \sh \neq \shp}
      {\cfalse}

    \rewrite[U-Trivial]
      {\trivial \ueq}
      {\ctrue}
\end{mathparfig}


\subsection{Solving rules}

% What we do (introduce / explain the solver)

We now gradually introduce the rules of the constraint solver itself
\TODO{List all individual figures, and also point at the big picture in the appendix}.
These rules define a non-deterministic rewriting
system, operating modulo $\alpha$-equivalence and the associativity and
commutativity of conjunction. Rewriting is performed under an arbitrary
one-hole constraint context $\C$.
% Solved forms
A constraint $C$ is satisfiable if it can be rewritten into a \emph{solved
form}:
\begin{mathpar}
  \hat{\up} \uad\triangleq\uad \cexists \tvs \cAnd \iton \ueqi
  \quad
  \text{where:}\quad\left\{
  \begin{array}{l}
    \text{each $\ueqi$ contains at most one non-variable type,}\\
    \text{head variables do not occur in multiple equations,}\\
    \text{the constraint is acyclic}
  \end{array}
  \quad\right\}
\end{mathpar}
From such a form, a most general solution (\ie a unifier $\vartheta$) can be
directly read off.

% Common rule descriptions

\paragraph{Basic rules}

Rule \Rule{S-Unif} invokes the unification algorithm to the
current unification problem. The unification algorithm itself is treated as a
black box by the solver, allowing the solver to be parameterized over the
equational theory of types implemented by the unifier.

\begin{bnffig}[t]
  {fig:constraint-syntax-extension}
  {Syntax of region-based $\Let$ constraints and partial applications}
  \entry[Constraints]{\c}{
    \dots \and \cletr \x \tv \tvs \ca \cb \and \cpapp \x \ren \ueqs \t
  } \\
  \entry[Semantic environment]{\semenv}{
    \dots \and \semenv, \x := \gabsr \tv \tvs
  } \\
  \entry[Renaming]{\ren}{
    \eset \and \rho[\tv := \tvb]
  }
\end{bnffig}

In general, existential quantifiers $\cexists \tv \c$ are lifted to the
nearest enclosing $\Let$, if any, or otherwise to the top of the constraint.
We refer to the resulting existential prefix $\exists \tvs$ as a
\emph{region}. To make regions explicit, we introduce the syntax $\cletr \x
\tv \tvs \ca \cb$, where $\tv$ is the \emph{root} of the region and $\tvs$ are
auxiliary existential variables. Both $\tv, \tvs$ are bound in $\ca$ but they
may also appear in $\cb$ but only in a partial application of $\x$.  In
particular, $\tvs$ cannot appear in $\cb$ when $\x$ does not appear in $\cb$.
The order of $\tvs$ is immaterial, and regions are considered equal up to
permutation of these variables.

The satisfiability of regional $\Let$-constraints is defined by:
\begin{mathpar}
  \infer[LetR]
    {\semenv \th \cexists {\tv, \tvs} \ca \\
     \semenv, \x \is \semenv(\cabsr \tv \tvs \ca) \th \cb}
    {\semenv \th \cletr \x \tv \tvs \ca \cb}

  \infer[AppR]
    {\semenv(\x) = \gabsr \tv \tvs \\
     \semenv(\t) \in \glam}
    {\semenv \th \capp \x \t}
\\
  \semenv(\cabsr \tv \tvs \c) \uad\eqdef\uad \gabsr[\set {\gt \in \Ground :
  \semenv, \tv \is \gt, \tvs \is \bar \gt \th \c}] \tv \tvs
\end{mathpar}
The semantic interpretation of an abstraction with a region, written
$\semenv(\cabsr \tv \tvs \c)$, is a set of ground types $\glam \subseteq
\Ground$ for which the applications are satisfiable with the region
$\tv\;\where\tvs$. Region-based $\Let$-constraints
strictly generalize ordinary constraint abstractions:
\begin{mathpar}
  \clet \x \tv \ca \cb \cequiv \cletr \x \tv \eset \ca \cb
\end{mathpar}
In particular, Rule \Rule{S-Let} rewrites the usual let constraints $\clet
\x \tv \ca \cb$ into region form.

\begingroup
\sloppy Rule \Rule{S-Exists-Conj} lifts existentials across
conjunctions, while rules \Rule{S-Let-ExistsLeft} and \Rule{S-Let-ExistsRight}
lift existentials into or across let-binders.
%
Rules \Rule{S-Let-ConjLeft} and
\Rule{S-Let-ConjRight} hoist constraints out of let-binders when they do not
depend on locally bound variables. Collectively, these lifting rules
normalize the structure of each region into a block of existentially bound
variables, comprising set of solved multi-equations (from unification) and
residual unsolved constraints (\ie applications, let-bindings and suspended
constraints).
\par
\endgroup

\begin{mathparfig}[t]
  {fig:solver-basic}
  {Basic rewriting rules $\ca \csolve \cb$}
  \rewrite[S-Unif]
    {\upa \\ \upa \unif \upb}
    {\upb}

  \rewrite[S-Let]
    {\clet \x \tv \ca \cb}
    {\cletr \x \tv \eset \ca \cb}

  \rewrite[S-Exists-Conj]
    {(\cexists \alpha \ca) \cand \cb \\
     \tv \disjoint \cb}
    {\cexists \tv {\ca \cand \cb}}

  \rewrite[S-Let-ExistsLeft]
    {\cletr \x \tv \tvs {\cexists \tvb \ca} \cb \\
     \tvb \disjoint \tv, \tvs}
    {\cletr \x \tv {\tvs, \tvb} \ca \cb}

  \rewrite[S-Let-ExistsRight]
    {\cletr \x \tv \tvs \ca {\cexists \tvb \cb} \\
     \tvb \disjoint \tv, \tvs, \cb}
    {\cexists \tvb {\clet \x \tvs \ca \cb}}

  \rewrite[S-Let-ConjLeft]
    {\cletr \x \tv \tvs {\ca \cand \cb} \cc \\
     \ca \disjoint \tv, \tvs}
    {\ca \cand \cletr \x \tv \tvs \cb \cc}

  \rewrite[S-Let-ConjRight]
    {\cletr \x \tv \tvs \ca (\cb \cand \cc) \\
     \x \disjoint \cc}
    {\cc \cand \Clet \x \tv \ca \cb}
\end{mathparfig}


Remaining basic constraint-formers, such as label and polytype instantiation
constraints, introduced in \cref{sec/TODO}, are handled by rewriting them once
their left-hand side becomes concrete. For instance, an instantiation
constraint $s \leq \t$ as introduced in \cref{TODO} is handled once $s$ is subsitituted by a match constraint,
yielding $\ts \leq \t$, which is then solved using the syntactic sugar
introduced in \cref{sec:constraints}. Similarly, label instantiation
constraints of the form $\Omega(\elab/t) \leq \t$ are resolved once $t$ is
substituted for some concrete $\T$.

\paragraph{Suspended match constraints}

\parcomment{S-Susp-Use}

Rule \Rule{S-Susp-Use} solves suspended match constraints whose scruintee
has a locally known shape---either statically, or as a result of appplying
Rule \Rule{S-Susp-Ctx}. The rule selects a unique matching branch based on
the shape of the scruitinee and applies the generated substitution to the
branch. If no branch matches, the rule is stuck, indicating that the
constraint is unsatisfiable.

\begin{mathparfig}
  {fig:solver-susp}
  {Rewriting rules for suspended match constraints.}
  \rewrite[S-Susp-Use]
    {\cmatch \t {\cbranch {\bar \cpat} {\bar \c}} \\
     \shape \t \Defined \\
     \cmatches \cpati {\decomp \t} \theta}
    {\theta(\ci)}

  \rewrite[S-Susp-Ctx]
    {\C\where{\cmatch \tv \cbrs} \\
     \cunif \tv {\cunif \t \ueq} \in \C \\
     \shape \t = \any \tvcs \tp}
    {\C\where{\cexists \tvcs  \cunif \tv \tp \cand \cmatch \tp \cbrs}}
\end{mathparfig}

\parcomment {S-Susp-Ctx}

When the solver encounters a suspended constraint with an unknown shape, it
cannot proceed immediately. Instead, it must wait until the surrounding
context proves that the scruintee $\tv$ is unified with a type $\t$ whose
shape is defined. This enables the suspended match to be rewritten, according
to \Rule{Susp-Ctx}, thereby exposing the shape required to continue solving in
\Rule{S-Susp-Use}.

We decide whether the context $\C$ proves an equality (or more generally),
a multi-equation $\ueq$, by finding contexts $\Ca, \Cb$ such that
$\C = \Ca[\ueq \cand \Cb]$ and $\fvs \ueq \disjoint \Cb$ \ie $\Cb$ does not
bind the free variables of $\ueq$.

\paragraph{Let constraints}

% Let-constraint solving is generalization

Application constraints can be solved by copying constraints:
\begin{mathpar}
  \rewrite[S-Let-App-Beta]
    {\cletr \x \tv \tvs \ca {\C\where{\capp \x \t}} \\ \tv, \tvs \disjoint \t \\ x \disjoint \C}
    {\clet \x \tv \ca {\C\where{\cexists {\tv, \tvs} \cunif \tv \t \cand \ca}}}
\end{mathpar}
This is similarly to $\beta$-reduction, except that the original constraint
is kept around.
While obviously correct, this na\"ive strategy may duplicate constraint solving
work across applications of the same abstraction.

\parcomment{Background on efficient solving of applications (aka generalization)}

A more efficient approach first solves the abstraction once---\eg reducing it
to $\cabsr \tv \tvs \ueqs$, where $\tvs$ are generalizable variables---and
then reuses the result by copying only the solved constraint body $\ueqs$ at
each application site, only copying the solved constraints. Concretly, this
corresponds to solving $\ca$ to $\hat\up$ and identifying the generalizable
variables $\tvs$ and equations $\ueqs$ in $\hat\up$.

\parcomment{Where is this formalized?}

This connection between constraint abstractions and \ML type schemes is
formalized by Pottier and R\'emy \citet{Pottier-Remy/emlti}, who show that the
solved $\cabsr \tv \tvs \ueqs$ is equivalent to $\tfor {\tvs} \theta(\tv)$
where $\theta$ is the most general unifier of $\ueqs$. Consequently, efficient
implementations of \HM inference, such as that of \OCaml, implement
instantiation by copying the generalized constraint body $\ueqs$ with fresh
variables for $\tvs$.

\parcomment{Generalization with suspended constraints is hard}

However, suspended constraints complicate generalization.
To illustrate this, let us examine:
\begin{program}[input]
  type three = {x : int; y : int} °\ocamlcomment {OCaml - typechecks}°
  let e$_{10}$ r = let y = r.y in y + r.x
\end{program}
The generated constraint\footnote{Simplified for readability.} is of the
form:
\begin{mathpar}
  \cexists \tv
    \clet y \tvb
      %% {\cmatchdots \tva}
      {\cmatch \tvb {\cbranch {(\wild \Tapp[t])} {\C\where {t, \tva,\tvb}}}}
      {\cinst y \tint \cand \cunif \tv {\mathsf{three}}}
\end{mathpar}
\begin{version}{}
where $\C\where{t,\tv,\tvb}cbr$ is ${\Omega(\elab / t) \leq \tva \to \tvb}$
\end{version}
Here, $\tv$ stands for \code{r}'s type. The constraint remains
suspended until \code{r.y} forces \code{r}'s type to be
\code{three}.\Xdidier {I don't understand the discourse. First, we do not
know what are the overloaded labels. Then the suspended constaints if for
$r.y$, so the expression that could unfreeze it is $r.x$ not $r.y$.}
Cruicially, the variable $\tvb$ (introduced inside the abstraction for the
type of \code{y}) is captured by the suspended match constraint that is
not yet resolved at the point of generalizing \code{y}.

\parcomment {Partial type schemes}

Nonetheless, to continue solving the let-body, we must assign a scheme to
\code{y}. We na\"ively pick $\tfor \tvb \tvb$. This appears unsound, since
$\tvb$ will later unify with $\tint$ once the match constraint is discharged.
But it would be incomplete to lower $\tvb$ as a monomorphic variable.
%
This motivates, \emph{partial type schemes}, our second novel mechanism for
omnidirectional inference, Partial type shemes are type schemes that delay
commitment to certain quantifications (\eg $\tvb$). Such \emph{partially
generalized} variables are treated as generalized, but can still be refined
in future as suspended constraints are discharged.

\parcomment {Intro partial applications}

To support this, we extend the language of constraints with \emph{partial
application constraints} $\cpapp \x \ren \ueqs \t$, representing a partially
solved application $\capp \x \t$. The renaming $\ren$ map $\x$'s regional
variables to fresh copies, and $\ueqs$ accumulates the solved equations copied
so far. This enables efficient, incremental instantiation of constraint
abstractions: solved parts are reused immediately, while unresolved
constraints (\ie suspended constraints) can be solved later, further refining
the abstraction and propagation additional equations to the application sites.

\begin{mathparfig}[t]
  {fig:solver-schemes}
  {Rewriting rules for let-bindings and applications.}
  \rewrite[S-Let-App]
    {\cletr \x \tv \tvs \ca \C\where{\capp \x \t} \\
     \tvc \disjoint \t \\
     \x \disjoint \C}
    {\cletr \x \tv \tvs \ca \C\where{\cexists \tvc \cunif \tvc \t
                               \cand \cpapp \x \eset \eset \tvc }}

  \rewrite[S-Papp-Exists]
    {\cletr \x \tv {\tvs, \tvb} \c \C\where{\cpapp \x \ren \ueqs \tvc} \\
     \tvb \notin \dom \ren \\
     \tvbp \disjoint \ren, \ueqs, \tvc \\
     \x \disjoint \C}
    {\cletr \x \tv {\tvs, \tvb} \c
      \C\where{\cexists \tvbp \cpapp \x {\ren \where {\tvb \is \tvbp}} \ueqs \tvc}}

  \rewrite[S-Papp-Unif]
    {\cletr \x \tv \tvs {\c \cand \ueq} {\C\where{\cpapp \x \ren \ueqs \tvc}} \\
     \dom \ren = \tvs \\
     \ueqs \nvDash \ueq \\
     \x \disjoint \C}
    {\cletr \x \tv \tvs {\c \cand \ueq}
      \C\where{\cpapp \x \ren {\ueqs, \ueq} \tvc
	 \cand \ueq[\ren\where{\tv \is \tvc}}]}

  \rewrite[S-Papp-Solve]
    {\cletr \x \tv \tvs {\ueqs'} {\C\where{\cpapp \x \ren \ueqs \tvc}} \\
      \dom \ren = \tvs \\
     \ueqs \vDash \ueqs' \\
     \x \disjoint \C}
    {\cletr \x \tv \tvs {\ueqs'} {\C\where{\ctrue}}}

  \rewrite[S-Let-Solve]
    {\cletr \x \tv \tvs \ueqs \c \\ \x \disjoint \c \\
     \cexists {\tv, \tvs} \ueqs \cequiv \ctrue}
    {\c}

  \rewrite[S-Exists-Lower]
    {\cletr \x \tv {\tvas, \tvbs} \c
     {\C\where{\overline{\cpapp \x \ren \ueqs \tvc}}} \\
     \cdetermines {\cexists {\tv, \tvas} \c} \tvbs \\
     \tvbs \subseteq \dom \bar \ren \\
     \x, \tvbs \disjoint \C
     }
    {\cexists \tvbs
      \cletr \x \tv \tvas \c
	{\C\where{{\overline{\cpapp \x {\ren \setminus \tvbs} \ueqs
			 \cand \cunif {\ren(\tvbs)} \tvbs}}}}}

  \rewrite[S-BackProp]
    {\C\where
       {\cletr \x \tv {\tvs,\tvp} {\Ca\where{\cmatch \tvp \cbrs}}
                           {\Cb\where{\cpapp \x \ren \ueqs \tvc}}} \\
     \cunif {\ren(\tv')} {\cunif \t \ueq} \in \C\where\Cb \\
     \x \disjoint \Cb}
    {\C\where{\cletr \x \tv {\tvs,\tvp} {\Ca\where{\cmatched \tvp {\shape \t} \cbrs}}
		      {\Cb\where{\cpapp \x \ren \ueqs \tvc}}}}
\end{mathparfig}


% Semantics to partial applications
We assign the following semantics to partial application constraints:
\begin{mathpar}
\infer[Papp]{
   \semenv(\x) = \gabsr \tv \tvs \\
   \dom \ren \subseteq \tvs \\
   \semenv, \tv \is \semenv(\t) \th \ueqs [\ren] \implies
   \semenv(\t) \in \glam                             \\
}{% -------------------------------------------------------
   \semenv \th \cpapp \x \ren \ueqs \t
}
\end{mathpar}
Informally, this states that a solution $\semenv$ to a partial application
constraint $\cpapp \x \ren \ueqs \t$ must also be a solution to the full
application $\capp \x \t$, provided the copied equations $\ueqs$ are
satisfiable under the renaming.
\parcomment{Treatment of alpha-renaming}
The semantics enforces that the domain of the renaming $\ren$ is a subset of
the regional variables $\tvs$ of $\x$---a cruicial property for treating these
variables modulo $\alpha$-equivalence. As mentioned above, $\tvs$ are bound not
only in the body of the abstraction $\ca$, but also in the
constraint $\cb$, where they may appear in partial applications of $\x$ via
renamings---and only there. Hence, when they cannot appear in $\ca$ when
(the corresponding variable) $\x$ does not itself appear in $\ca$.

\begin{remark}
  The satisfiability judgement in \Rule{Papp} occurs in a negative position.
  This does not threaten well-foundness, as the conclusion of this rule is
  strictly larger than the negatively occuring judgement. See Appendix ??.
\end{remark}

Partial application constraints are solved using the following rules:
\begin{enumerate}

\item
  Rule \Rule{S-Papp-Exists} allocates fresh existential variables at the
  call site that are mapped to uninstantiated variables within the
  abstraction.  We write $\ren : \tvs$ for $\dom \ren = \tvs$.

  \TODO{Remove $\ren : \tvs$ and use $\dom \ren = \tvs$ instead.}

\item
  Rule \Rule{S-Papp-Unif} copies multi-equations from the abstraction to the
  call site, provided we have already introduced the renamings by Rule \Rule
  {S-Papp-Exits} and we have not copied them previously.

\end{enumerate}
\parcomment{Cleaning up partial applications and let constraints}
Rule \Rule{S-Papp-Solve} removes partial applications once all necessary
constraints have been copied---that is, when the copied equations entail
the abstraction's body. \Rule{S-Let-Solve} remove a $\Let$ constraint when
the bound term variable is unused and the abstraction is satisfiable.

\parcomment{Lowering}

Rule \Rule{S-Exists-Lower} implements the non-trivial case of lowering
existentials across $\Let$-binders. It identifies a subset of variables in
the region of a $\Let$ constraint that are unified with variables from
outside the region. Such variables are considered monomorphic and thus
cannot be generalized; they can instead be safely lowered to the outer
scope.

\parcomment {Determines}

This is the case when the types of $\tvbs$ are \emph{determined} in a unique
way. In short, $\c$ determines $\tvbs$ if and only if the solutions for
$\tvbs$ are uniquely fixed by the solutions to other variables in $\c$.

\begin{definition}
  $\cdetermines \c \tvbs$ if and only if every ground assignments
  $\semenv$ and $\semenvp$ that satisfy $\c$ and coincide outside of $\tvb$
  concide on $\tvbs$ as well.
  \begin{mathpar}
    \cdetermines \c \tvb \uad\eqdef\uad \all {\semenv, \semenvp} \uad
      \semenv \th \c
      \wedge \semenvp \th \c
      \wedge \semenv =_{\setminus \tvbs} \semenvp
      \implies
      \semenv = \semenvp
  \end{mathpar}
\end{definition}

\parcomment {How the determines relation corresponds to ML}
Conceptually, this corresponds to the negation of the generalization condition
in \ML: a type variable cannot be generalized if it appears in the typing
context. In the constraint setting, it cannot be generalized if it depends on
variables from outside the region.

% How to decide the relation
To decide when $\cdetermines {\cexists \tvs \c} \tvbs$ holds for $\tvbs
\disjoint \tvs$, we search for a multi-equation $\ueq$ in $\c$ of the form:
\begin{enumerate*}
  \item $\cunif \tvc \ueq'$ where $\tvc \disjoint \tvs, \tvbs$ and
    $\tvbs \subseteq \fvs {\ueq'}$, or
  \item $\cunif \tvbs {\cunif \t \ueq'}$ where $\fvs \t \disjoint
    \tvs, \tvbs$.
\end{enumerate*}
For instance, $\cexists \tvba \cunif \tv {\tvba \to \tvbb}$ determines
$\tvbb$, as $\tvbb$ is free.
\parcomment{Why we lower?}

Lowering such variables improves solver efficiency. It avoids unnecessary
duplication of work that would otherwise occur via \Rule{S-Papp-Exists}, which
allocates fresh copies of all regional variables at each application site. By
reducing the number of variables that need to be copied, lowering directly
reduces instantiation overhead.

\parcomment{Updating partial applications}

However, sometimes \Rule{S-Papp-Exists} must copy a variable before it can be
lowered, for instance, in the constraint $\cexists \tvc \cletr \x \tva \tvb
{\cunif \tva \tvb \cand \cmatch \tvb {\cbranch \cwild {\cunif \tvb \tvc}}}
{\capp \x \tint}$ requires us to instantiate $\x$ before we can make any
progress on lowering $\tvb$. In this case, when we eventually lower the
variable, we must also update any partial applications that previously copied
it, ensuring their local copies are unified with the new, lowered variables
$\tvbs$.


\TODO{Compression}

\TODO{Should we mention backpropagation prior to here?}

\TODO{Forall constraints -- this should be the same from EMLTI}

\paragraph{Backpropagation}

Finally, Rule \Rule{S-BackProp} expresses \emph{backpropagation}, which we explained in \cref{ex:backprop}.

Consider:
\begin{program}[input]
  let e$_{10}$ = let f r = r.y in f {x = 1; y = 1}
\end{program}
The (simplified) constraint generated when typing \code{e}$_{10}$ is:
\begin{align*}
  \cexists \tv
  &\clet f \tvd {\cexists {\tvb, \tvc}
    \cunif \tvd {\tvb \to \tvc} \cand
    \cmatch \tvb
       {(\cbranch {\wild \Tapp[t]} {\Omega(y / t) \leq \gamma \to \beta})}} \\
  &\capp f {(\mathsf{three} \to \tv)}
\end{align*}
At the use site, the abstraction is applied with a type of known shape
(\code{three}).  Rule \Rule{S-BackProp} propagates this shape back into the
suspended match inside the abstraction by unifying it with the variable
$\tvb$. This is sound: the unicity predicate $\Cshape \C \tv \sh$ ensures
that $\tvb$'s shape is \code{three} provided the context $\C$ include the
application constraint.
\TODO{Maybe we want to rephrase the explanation of the rule to not refer to the example anymore,
and get rid of the example. But then the rule is hard to explain abstractly.}

\begin{mathpar}
\end{mathpar}


\subsection{Metatheory}
% Properties

\begin{theorem}[Progress]
  If $\cdot \th \c$ and $\c$ is not solved, then there exists a $\cp$
  such that $\c \csolve \cp$.
\end{theorem}

\begin{theorem}[Termination]
  The constraint solver terminates on all inputs.
\end{theorem}

\begin{theorem}[Preservation]
  If $\ca \csolve \cb$, then $\ca \cequiv \cb$.
\end{theorem}



\section{Discussion}
\label{sec:discussion}


\subsection{Prototype implementation}

\parcomment{We have a solver, but it isn't an algorithm}
We've formalized our constraint solver, but it is not immediately clear how to
turn our specification into an executable algorithm. We outline some of the key
engineering work necessary for this, but omit a full description due to space
constraints.

\paragraph{Unification and schedulers}

To support suepended match constraints, we must allow a match constraint to
wait on the resolution of a unification variable. Inspired by
lightweight-threading implementations \cite{TODO}, we treat unification
variables as \emph{write-once immutable cells} (or \emph{ivars})
that, once instantiated, may unblock further constraint solving.

Each unsolved unification variable maintains a \emph{wait list} of suspended
constraints that are blocked until the variable is unified with a concrete
type.
%
Once unified, the wait list is flushed: the suspended constraints are scheduled
on the global constraint scheduler, which is responsible for eventually solving
them.


\paragraph{Rank-based generalization}
% Ranks
To implement generalization (the \Rule{S-Lower-Exists} rule) efficiently, we
follow the classic rank-based approach to generalization \citep{TODO}. Each
$\Let$ constraint and type variable is allocated an integer \emph{rank}, which
informs us the depth of the region within the constraint. Type variables of
rank $0$ are bound at the top-level region, and type variables of rank $r \geq
1$ are bound in the region of $\Let$ constraint at depth $r$.

% Lowering / generalization
As inference progresses, unification may widen the scope of variables,
thereby lowering their rank. The set of variables eligible for
generalization at a given region consists precisely of those
whose rank remains equal to that of the region.

\TODO{Cool tikz diagram of this -- displaying ranks as a stack against a constraint}

% Ranks are no-longer enough
While this rank-based stack of regions suffices for \ML generalization,
it is too inflexible for \emph{partial generalization}. If generalization
at some region is suspended (\eg due to a suspended match constraint),
the region must remain alive. Later parts of the constraint may
introduce a new $\Let$-region at the same rank that is unrelated --
neither an ancestor nor a descandant -- breaking the linear assumption
of ranks.

% Generalization trees
To address this, our implementation replaces the rank stack
with a \emph{region tree}, where nested $\Let$-regions form
a proper tree structure. In the specification, this structure
is encoded syntactically by $\Let$ constraints themselves.

\TODO{Cool tikz diagram of generalization trees with paths for ranks.
      Using the same example as above.}

Under this scheme, ranks no longer uniquely determine a variable's
region. Instead, we interpret a rank relative to a path $\pi$ in the
region tree from the root. When two variables are unified, they must
always lie on some shared path---by scoping invariants---so computing
their minimum rank \emph{along $\pi$} sufficies to determine the
lowered region.


\paragraph{Partial generalization}

Partial generalization arises when a region cannot be fully
generalized due to suspended constraints that may still update
its variables. To manage this, we classify type variables
into four categories:

\begin{itemize}
  \item[\textbf{I}] Variables are yet to be generalized. \\
    \emph{Introduced by applications or source types in constraints} \\

  \item[\textbf{G}] Variables that are generalized. \\
    \emph{Not accessible from any instance type. Treated polymorphically.} \\

  \item[\textbf{PG}] Variables that are generalizable, but may be
    updated in the future. \\
    \emph{Variables mentioned by suspended match constraint or partial applications.} \\

  \item[\textbf{PI}] Variables that were previously partially generalized
    but have since been updated.  \\
    \emph{Awaiting re-generalization. Introduced by unification of partial generics.}
\end{itemize}


It is possible to transition between these states, triggered by generalization
and unification.

At generalization time, we conservatively approximate whether a variable may be
updated in the future using \emph{guards}. A guard is a mark on a variable that
indicates the variable is captured by some suspended constraint that has not
yet been solved. Guarded variables are generalized as partial generics (PG);
unguarded ones are fully generalized (G).

When an instance is taken from a partial generic, we retain a forward-reference
from the partial generic (PG) to the instance. This enables the generic to
notify the instance that it has been updated, propagating the updated type
structure to all instances. This is a reversed analogue of how partial
application constraints track renamings. In addition, the instance remains
guarded by the partial generic until the latter is either lowered or fully
generalized.

Once a suspended match constraint is solved, it removes the guard it
introduced. This may enable previously partial generics to become fully
generalizable. Conversely, if a partially generalized variable is unified with
an instance and lowered (\eg by \Rule{S-Lower-Exists}), it must unify all
instances with itself (once lowered).

\TODO{Draw cool state diagram figure (with colours)}

\paragraph{Lazy generalization}

Repeatedly generalzing a region after every update is expensive.
Instead, we adopt a lazy approach: we track updates to the region using
a \emph{generalization tree}, marking regions as ``visited'' when they may
require re-generalization. Only when an instance is taken do we
force generalization of a region and all of its descendants---the descendants
must be fully generalized before generalizing the parent region, since
generalization may update ancestor regions when lowering variables.


\subsection{Static overloading and variational types}

We applied omnidirectional type inference to three features that have
already been used in some dialect of \ML.  But applications are not limited
to those.

In particular, overloading of nominal record fields (or data-consttructors)
is a very specific case of overloading.  Omnidirectional type inference can
also be used for a general forms of static overloading.

Following~\cite{Leijen-Ye/prefix@pldi2025},
we may allow identifiers $\M.\x$ to be prefixed by a
disambiguation name $\M$ and let the prefix be optional in expressions.
When absent, such expressions are then overloaded and the prefix must be
resolved before the program may be compiled.

In this setting an overlaoded symbol $\x$ may have two (or more) definitions
$\Ma.\x$ an $\Mb.\x$ of respective types $\tya$, $\tyb$, which may have very
litte in common. In particular, $\tya$ and $\tyb$ may share the same prefix
an only differ from their leaves.

The framework we presented for \OML may only suspend a constraint on the
shape of a variable, that is roughtly on the toplevel symbol.  While it
would be possible to extend the framework to alow more general shapes, we
can actually encode general overloading without such an extension.
We present a solution below based on variational types, but other variants
are possible.

\paragraph{Variational types}

\begin{version}{}
\color{red}
Variational types are a mean to factor a tuple of types into a single one so
as to shape the common prefixes.

The syntax of variational type just extends simple types with a special form
of n-ary products $\varpi (\ti)\iton$, which however distributes other type
symbols. For example, $\vty {\tint \to \tint \to \tbool \vor \tbool \to
\tbool \to \tbool}$ is a 2-variational type $\t$ that represents both the type
$\tint \to \tint \to \tbool$ and $\tbool \to \tbool \to \tbool$.  However,
the interest of variational types is to be able to push them down the leaves
and equivalently write $\vty {\tint \vor \tbool} \to \vty {\tint \vor
\tbool} \to \tbool$ for $\t$ so as to share the common prefix of all types.
We may thus maintained variational types in such canonical form.  If $\vty
{\ti}\iton$ is in canonical form, then at least two $\tyi$ and $\tyj$ have
different shapes.  The arity of avariational typ must be known from context
since in the pathological case where all types agrees, its canonical form is
$\t$.

An n-variational type may be projected on any index of its domain: $\t/1$
means $\tint \to
\tint \to \tbool$.

For overloading of two defintions $\Ma.\x$ and $\Mb.\x$ of types $\tya$ and
$\tyb$, we may assign $\x$ the variational type $\vty {\tya \vor \tyb}$.
\end{version}

\paragraph {Opaqueness of suspended constraints}

Suspended constraints are opaque, \ie the solve may not learn anything form
suspended constraints until uniqueness of the shape the control them allows
to discharge them.

For example, this prevents typing of example \ocaml{e_9}, which we repeat
below:
\begin{program}
let e9 r = (r.x : bool)
\end{program}
Hence, the constraint type of \ocaml{r.x} is frozen until the record type
$r$ can be resolved. Hence, the overloading resolution ignores that the
return type of \ocaml{r.x} is \ocaml{bool} and fails. While, if we were
able to take it into account, then the only possibility would be that
\ocaml{r} has type \ocaml{bool one}---in the closed world view.


constaint that the return type of
\ocaml{r.x} is not taken into account in the overloading resolution.


While sometimes there is obvious information that we could learn from them,
especially by merging constraints on the same variable.

For example,


\paragraph {Variational types}

\begin{version}{\color{blue}\Draft}
Title to adjust.
This section is to explain
\begin{itemize}

\item
  that there are possible variants on static overloading.

\item
  The main idea of omnidirectional type inference is to collect as
  many constraints that can help resolve overloading, statically.
  Hence, we may be more aggressive, in some cases,

\item
  Variational types in general---but the have a cost

\item
  Using rows in nomimal records to merge contraints on the same record.
\end{itemize}


\end{version}

\Xdidier [From 2.1]{Then, it seems that we cannot solve
$\efun r {r.x + r.y}$, since the two accesses $r.x$ and $r.y$ will both wait
for the type of $r$ to be known independently of the other.
Since we could collect the information that $r$ must have two fields $x$
and $y$ and there is a unique solution. In fact, an approach based on
qualified types would solve this example statically.
Nominal records are an example where we should allow to combine information
from two different use points---of the same type. That is several matches on
the same variable should be combined. I cannot either for the moment, but at
some point I wanted to merge overloaded constraint on the same variable to
take take the intersection of their head constructors. Still planing to do
that, which would solve this example}



\subsection{Default rules}

Type inference may failed either because of a typing error (the constraint
if of the form $\C \where \cfalse$), or because some choice point cannot be
resolved---a stuck constraint contains a suspended match constraint.
%
In the later case, we may point the user to at least one choice point that
could not be solved and invite him to provide additional type annotations.

In fact \OCaml uses default rules in case an overloaded constructor or a record
field could not be solved, then selecting the most recently defined type.

While, we could also introduce similar default rules in our proposal, they
would likely break the properties of the type system.

One option is not to integrate default rules in omnidirectional type
inference, and instead combine omnidirectional type inference with another
external external algorithm that can suggest extra type annotations and
calls back type inference, or rather, resume type inference from its
previous state.  In this view, we preserve the properties in
in the absence of uses of default rules, but give up any property
as default rule have been used.

Hence, it is also appealing to attend to formalize default rules.  In order
to avoid choices, one option may be to fire all default rules at once and
independently of one another.  This however may loose opportunities to learn
useful information after firing one default rule so as to not have to used
other default rules.

Another track is to computer dependencies between unsolved constraints---and
solve them in inverse order of dependencies.

One the strategy of when firing default rules, the behavior of default rules
is quite dependent on the feature.  While, the reasonable default for
overloading of record fields and data constructors is to giving priority to
the most recent definition, as in \OCaml, there is not obvious choice for
structural tuples, for which the default should probably remain a failure.

For polytypes, there are actually to opposite meaningful choices, but
opposite of one another. One is to fallback to a monomorphic polytype.  The
other one is to pick the principal type scheme of the expression with an
unresolved polytype.  Of course, there are examples that favor one choice
and other examples that favor the other.

While default rules may be convenient in practice, they remain unprincipled
and may break properties of omnidirectional type infernce.  The strategy of
when firing them should be precautionary to limit what may look like
arbitrary or unpredictable choices.

\subsection{Equi-recursive types}
\label {sec/rec-types}

For sake of simplicity, we have taken finite trees for ground types, that
is, for the semantics of constraints.  Iso-recursive types can easily be
added as in \ML, using a new datatype definition where  a record label
or datatype constructor is used to fold or unfold the recursive definition.

\OCaml also allows equi-recursive types, which can be modeled by taking
regular trees instead of tree for ground types.  Doing this change would
preserve the main metatheoretical properties, as in~\cite
{Pottier-Remy/emlti}, as we never rely on finiteness of ground types.  This
would allow type inference with equi-recursive types.

Technically, we need to extend the syntax of types with
a new introduction form $\trec \tv \t$ with the condition that $\t$ is
constructive, \ie a nonvariable type.  Internally, we never manipulate
recursive types but equations. The change, is a remove
\Rule {U-Cycle} and consistently allow canonical forms to contain cycles,
as they now admit regular tree solutions.

Similarly, shapes may be recursive, but only minimal shapes of polytypes may
be recursive. All types still have unique minimal shapes where equality of
shapes is taken up to equi-recursion, indeed.  While equality of shapes is
syntactical in the absence of recursive types, this is no longer the
case---but this does not raise any issue.

\Xdidier{There is an inline note here.}
\begin{version}{}
I did not see any difficult with equi-recursive types.
I am not even sure we need to mention~\cite
{Gauthier-Pottier/numbering@icfp04}.
Ths provides a cannonical representation, generalizing de Bruin indicies so
that second-order equality of terms amount to fisrt-order equality.
The canonical representation useing shapes is simiular.
\Xdidier {How much more should we say?}

\Xdidier {Should we give the change in unification rules?}
\end{version}

\section{Related work}
\label{sec:related-work}

Should we cite \cite{Leijen-Ye/prefix@pldi2025} ?

\subsection{Qualified types}

Qualified types~\citep*{TODO} represent additional knowledge on a type. They
are used in particular in Haskell type-classes, where a constraint
$\mathsf{Show}~\alpha$ represents the fact that a part of the inferred term
needs to print values of type $\alpha$. A constraint on a ground type such
as $\mathsf{Show}~\mathsf{Int}$ can be resolved to a known printer. But if
the undetermined variable $\alpha$ becomes generalizable in a type $\tau$,
we get a type-scheme $\tfor{(\alpha \mid \mathsf{Show}~\alpha)}{\tau}$ that
also includes the type-class constraint.

DRAFT: Type-classes let each use-site choose a different typeclass instance
-- but this implies dictionary-passing or specialization. Sometimes we want
the choice to be shared by all use-sites:


\subsection{Suspended constraints in \textsc{OutsideIn}}

DRAFT: They first solve simple constraints (existentials,
unification). Then they solve ``... constraints'', with implication
constraints, which corresponds to the case of GADT matches. Crucially, they
abandon local let generalization.

\subsection{Suspended constraints in dependent-type systems}


\subsection{Bi-directional type inference}

\Xgabriel{Our approach should have better properties for disambiguation, but
which ones?}

In the simply-typed case our system works better. But with generalization
you can have more issues.

\Xdidier{Should we try to extend what we did to predicative polymorphism?}

\Xalistair{Not ready yet and would probably need too much space.}

\Xgabriel{CoreML + bidirectional disambiguation of constructors?}

\TODO
{do we understand what to say precisely about bidirectional type inference?}

\subsection{Principality tracking in \OCaml}

\Xdidier{The point of that is to enforce a directional type inference that
is based on let-bindings. When we check that the level is generic, we check
that we already ``closed'' this thing, it is an earlier 'let'
binding. Because we are omni-directional, we don't have principality issues
anymore -- except with default rules. We are principal by construction, we
never make any choice.}

\Xdidier{Principality tracking is making the choice that we are going to
make a directional let-binding-based type inference. We get rid of that, we
are omni-directional, and don't have any principality issues.}

\Xgabriel{We should not claim too much if we don't understand default
clauses well enough. It could people the impression that we hide the issue
under the carpet.}

\Xdidier{First a declarative/principal system, and then non-principal
heuristics to refine it, a two-phase process. It's fine.}

\section{Future work}
\label{sec:future-work}

% \begin{acks}
% \end{acks}


%% \bibliographystyle{ACM-Reference-Format}
\bibliography{suspended}

\newpage
\appendix

\section{Figures}

This section contains supplementary material to the main paper, including a
list of figures and definitions. Some of them are repeated from the main paper.

\TODO{Write these here}

\judgbox{\c \simple}{The constraint $\c$ is simple.}
\begin{mathparfig}
  {fig:defn-simple}
  {Simple constraints.}

  \inferrule[Simple-True]
    {}
    {\ctrue \simple}

  \inferrule[Simple-False]
    {}
    {\cfalse \simple}

  \inferrule[Simple-Conj]
    {\cone \simple \\ \ctwo \simple}
    {\cone \cand \ctwo \simple}

  \inferrule[Simple-Exists]
    {\c \simple}
    {\cexists \tv \c \simple}

  \inferrule[Simple-Forall]
    {\c \simple}
    {\cfor \tv \c \simple}

  \inferrule[Simple-Unif]
    {}
    {\cunif \tone \ttwo \simple}

  \inferrule[Simple-Let]
    {\cone \simple \\ \ctwo \simple}
    {\clet \x \tv \cone \ctwo \simple}

  \inferrule[Simple-App]
    {}
    {\capp \x \t \simple}

  \inferrule[Simple-LetR]
    {\cone \simple \\ \ctwo \simple}
    {\cletr \x \tv \tvs \cone \ctwo \simple}

  \inferrule[Simple-Papp]
    {}
    {\cpapp \x \ren \ueqs \simple}
\end{mathparfig}

\judgbox{\C \simple}{The constraint context $\C$ is simple.}
\begin{mathparfig}
  {fig:defn-ctx-simple}
  {Simple constraint contexts.}
  \inferrule[Simple-Ctx-Hole]
    {}
    {\square \simple}

  \inferrule[Simple-Ctx-Conj-Left]
    {\C \simple \\ \c \simple}
    {\C \cand \c \simple}

  \inferrule[Simple-Ctx-Conj-Right]
    {\C \simple \\ \c simple}
    {\c \cand \C \simple}

  \inferrule[Simple-Ctx-Exists]
    {\C \simple}
    {\cexists \tv \C \simple}

  \inferrule[Simple-Ctx-Forall]
    {\C \simple}
    {\cfor \tv \C \simple}

  \inferrule[Simple-Ctx-Let-Abs]
    {\C \simple \\ \c \simple}
    {\clet \x \tv \C \c \simple}

  \inferrule[Simple-Ctx-Let-In]
    {\c \simple \\ \C \simple}
    {\clet \x \tv \c \C \simple}
\end{mathparfig}

\begin{mathparfig}
  {fig:constraint-erasure}
  {The erasure of suspended match constraints}
\newcommand{\Erule}[2]{\cerase {#1} &\eqdef& {#2}}
  \begin{tabular}{RCL}
  \Erule{\ctrue}{\ctrue} \\
  \Erule{\cfalse}{\cfalse} \\
  \Erule{\cone \cand \ctwo}{\cerase \cone \cand \cerase \ctwo} \\
  \Erule{\cexists \tv \c}{\cexists \tv \cerase \c} \\
  \Erule{\cfor \tv \c}{\cfor \tv \cerase \c} \\
  \Erule{\cunif \tone \ttwo}{\cunif \tone \ttwo} \\
  \Erule{\clet \x \tv \cone \ctwo}{\clet \x \tv {\cerase \cone} {\cerase \ctwo}} \\
  \Erule{\capp \x \t}{\capp \x \t} \\
    \Erule{\cmatch \t {\cbranch {\bar \cpat} {\bar \c}}}{\ctrue} \\
\end{tabular}
\end{mathparfig}

\judgbox{\semenv \Th \c}
  {Under the semantic environment $\semenv$,
   the constraint $\c$ is canonically satisfiable.}
\begin{mathparfig}
  {fig:defn-canonical-sat-deriv}
  {Canonical satisfiability derivations}

  \inferrule[Can-Simple]
    {\semenv \th \c \\ \c \simple}
    {\semenv \Th \c}

  \inferrule[Can-Susp-Ctx]
    {\Cshape \C \t \sh \\ \semenv \Th \C\where{\cmatched \t \sh \cbrs}}
    {\semenv \Th \C\where{\cmatch \t \cbrs}}
\end{mathparfig}

\judgbox{\up \unif \upp}
  {The unifier rewrites $\up$ to $\upp$.}
\begin{mathparfig}[htpb!]
  {fig:defn-unification-algorithm}
  {Unification algorithm.}
   \rewrite[U-Exists]
      {(\cexists \alpha \upa) \cand \upb \\ \tv \disjoint \upb}
      {\cexists \tv {\upa \cand \upb}}

    \rewrite[U-Cycle]
      {\up \\ \cyclic \up}
      {\cfalse}

    \rewrite[U-True]
      {\up \cand \ctrue}
      {\up}

    \rewrite[U-False]
      {\up \cand \cfalse}
      {\cfalse}

    \rewrite[U-Merge]
      {\cunif \tv \ueqa \cand \cunif \tv \ueqb}
      {\cunif \tv {\cunif \ueqa \ueqb}}

    \rewrite[U-Stutter]
      {\cunif \tv {\cunif \tv \ueq}}
      {\cunif \tv \ueq}

    \rewrite[U-Name]
      {\cunif {\pshapp \tys[\ti]} \ueq \\ \tv \disjoint \tys, \ueq }
      {\cexists \tv {\cunif \tv \ti \cand \cunif {\pshapp \tys[\tv] } \ueq}}

    \rewrite[U-Decomp]
      {\cunif {\pshapp \tvs} {\cunif {\pshapp \tvbs} \ueq}}
      {\cunif {\pshapp \tvs} \ueq \cand \cunif \tvs \tvbs}

    \rewrite[U-Clash]
      {\cunif {\pshapp \tvs} {\cunif {\pshapp[\shp]\tvbs } \ueq }\\
       \sh \neq \shp}
      {\cfalse}

    \rewrite[U-Trivial]
      {\trivial \ueq}
      {\ctrue}
\end{mathparfig}
\FloatBarrier

\judgbox{\c \csolve \cp}
  {The constraint solver rewrites $\c$ to $\cp$.}
\begin{mathparfig}[htpb!]
  {fig:defn-solver-basic}
  {}
  \rewrite[S-Unif]
    {\upa \\ \upa \unif \upb}
    {\upb}

  \rewrite[S-Let]
    {\clet \x \tv \ca \cb}
    {\cletr \x \tv \eset \ca \cb}

  \rewrite[S-Exists-Conj]
    {(\cexists \alpha \ca) \cand \cb \\
     \tv \disjoint \cb}
    {\cexists \tv {\ca \cand \cb}}

  \rewrite[S-Let-ExistsLeft]
    {\cletr \x \tv \tvs {\cexists \tvb \ca} \cb \\
     \tvb \disjoint \tv, \tvs}
    {\cletr \x \tv {\tvs, \tvb} \ca \cb}

  \rewrite[S-Let-ExistsRight]
    {\cletr \x \tv \tvs \ca {\cexists \tvb \cb} \\
     \tvb \disjoint \tv, \tvs, \cb}
    {\cexists \tvb {\clet \x \tvs \ca \cb}}

  \rewrite[S-Let-ConjLeft]
    {\cletr \x \tv \tvs {\ca \cand \cb} \cc \\
     \ca \disjoint \tv, \tvs}
    {\ca \cand \cletr \x \tv \tvs \cb \cc}

  \rewrite[S-Let-ConjRight]
    {\cletr \x \tv \tvs \ca (\cb \cand \cc) \\
     \x \disjoint \cc}
    {\cc \cand \Clet \x \tv \ca \cb}

  \rewrite[S-Susp-CtxType]
    {\C\where{\cmatch \t \cbrs} \\ \t \notin \TyVars}
    {\C\where{\cmatched \t {\shape \t} \cbrs}}

  \rewrite[S-Susp-CtxVar]
    {\C\where{\cmatch \tv \cbrs} \\
     \cunif \tv {\cunif \t \ueq} \in \C}
    {\C\where{\cmatched \tv {\shape \t} \cbrs}}

  \rewrite[S-Let-App]
    {\cletr \x \tv \tvs \ca \C\where{\capp \x \t} \\
     \tvc \disjoint \t \\
     \x \disjoint \C}
    {\cletr \x \tv \tvs \ca \C\where{\cexists \tvc \cunif \tvc \t
                               \cand \cpapp \x \eset \eset \tvc }}

  \rewrite[S-Papp-Type]
    {\cpapp \x \ren \ueqs \t}
    {\cexists \tvc \cunif \tvc \t \cand \cpapp \x \ren \ueqs \tvc}

  \rewrite[S-Papp-Exists]
    {\cletr \x \tv {\tvs, \tvb} \c \C\where{\cpapp \x \ren \ueqs \tvc} \\
     \tvb \notin \dom \ren \\
     \tvbp \disjoint \ren, \ueqs, \tvc \\
     \x \disjoint \C}
    {\cletr \x \tv {\tvs, \tvb} \c
      \C\where{\cexists \tvbp \cpapp \x {\ren \where {\tvb \is \tvbp}} \ueqs \tvc}}

  \rewrite[S-Papp-Unif]
    {\cletr \x \tv \tvs {\c \cand \ueq} {\C\where{\cpapp \x \ren \ueqs \tvc}} \\
     \ren : \tvs \\
     \ueqs \nvDash \ueq \\
     \x \disjoint \C}
    {\cletr \x \tv \tvs {\c \cand \ueq}
      \C\where{\cpapp \x \ren {\ueqs, \ueq} \tvc
	 \cand \ueq[\ren\where{\tv \is \tvc}}]}

  \rewrite[S-Papp-Solve]
    {\cletr \x \tv \tvs {\ueqs'} {\C\where{\cpapp \x \ren \ueqs \tvc}} \\
     \ren : \tvs \\
     \ueqs \vDash \ueqs' \\
     \x \disjoint \C}
    {\cletr \x \tv \tvs {\ueqs'} {\C\where{\ctrue}}}

  \rewrite[S-Let-Solve]
    {\cletr \x \tv \tvs \ueqs \c \\ \x \disjoint \c \\
     \cexists {\tv, \tvs} \ueqs \cequiv \ctrue}
    {\c}

  \rewrite[S-Exists-Lower]
    {\cletr \x \tv {\tvas, \tvbs} \c
     {\C\where{\overline{\cpapp \x \ren \ueqs \tvc}}} \\
     \cdetermines {\cexists {\tv, \tvas} \c} \tvbs \\
     \tvbs \subseteq \dom \bar \ren \\
     \x, \tvbs \disjoint \C
     }
    {\cexists \tvbs
      \cletr \x \tv \tvas \c
	{\C\where{{\overline{\cpapp \x {\ren \setminus \tvbs} \ueqs
			 \cand \cunif {\ren(\tvbs)} \tvbs}}}}}

  \rewrite[S-BackProp]
    {\C\where
       {\cletr \x \tv \tvs {\Ca\where{\cmatch \tvp \cbrs}}
                           {\Cb\where{\cpapp \x \ren \ueqs \tvc}}} \\
     \tvp \in \dom \ren \\
     \cunif {\ren(\tv')} {\cunif \t \ueq} \in \C\where\Cb \\
     \x \disjoint \Cb}
    {\C\where{\cletr \x \tv \tvs {\Ca\where{\cmatched \tvp {\shape \t} \cbrs}}
		      {\Cb\where{\cpapp \x \ren \ueqs \tvc}}}}
\end{mathparfig}
\FloatBarrier

\section{Properties of the constraint language}

\subsection{Principality of shapes}
\TODO{Make the styling between this section and other proofs more consistent}

\Cref {th/shapes/principal} states that any non variable type $\t$ has a
principal shape. We show that this is indeed the case.

\medskip

When $\t$ is not a polytype, it has a toplevel constructor $c$ of
arity $n$, which in our setting may be the nullary $\tunit$, the binary
arrow, the n-ary product, or a $n$-ary nominal type. In all these
cases, the shape of $\t$ is $\any \tvcs \tc \tvcs$ where $\tvcs$ is a
sequence of $n$ distinct type variables.

\medskip

It only remains to check that a polytype $\tpoly {\all \tvs \t}$ also has a
unique shape. We may assume \Wlog that each variable of $\tvs$ occurs free in
$\t$.
%
Let $(\pi_i)\iton$ be the sequence of shortest paths in $\t$ that cannot be
extended to reach a (polymorphic) variable in $\tvas$, in lexicographic
order and $\tvcs$ be a sequence $(\tvci)\iton$ of distinct variables that do
not appear in~$\t$.
%
Let $\tyz$ be $\t \where {\pi_i \is \tvci}\iton$, \ie the term $\t$ where each
path $\pi_i$ has been substituted by the variable $\tvci$.  Let $\Sh$ be the
shape $\any \tvcs {\tpoly {\all \tvs \tyz}}$.
We claim that $\Sh$ is actually the principal shape of $\tpoly {\all \tvs
\t}$.

\medskip
\locallabelreset

By construction, $\t$ is equal to $\shapp[\Sh] \tys$~\llabel 1.
where $\tys$ is the sequence composed of $\ti$ equal to $\t/\pi_i$
for $i$ ranging from $1$ to $n$.
%
Indeed, by
definition, $\shapp[\Sh] \tys$ is equal to $(\t\where {\pi_i \is \tvci}\iton)
\where {\tvci \is \ti}$ which is obviously equal to $\t$.
The remaining of the proof checks that $\Sh$ is minimal~\llabel 2, that is,
we assume that $\Sh'$ is another shape such that $\tpoly {\all\tvs\t}$ is
equal to $\shapp [\Shp] \typs$ for some $\typs$~\llabel H and show that $\Sh
\preceq \Shp$~\llabel C.

\medskip

It follows from~\lref H that
  $\Shp$ must be a polytype shape, \ie of the form $\any \tvcps {\tpoly
  {\all \tvbs \typ}}$ and
  $\tpoly {\all \tvs \t}$ is equal to $\tpoly {\all\tvbs \tp} \where {\tvcps
  \is \typs}$~\llabel{P}.
\relax
We may assume \Wlog that $\tvbs$ and $\tvcps$ are disjoint, that
$\tvcps$ does not contain useless variables, \ie
that they all appear in $\tp$ and that they actually appear in lexicographic
order.
\relax
Now that never term contains useless variables, \lref P implies that the
sequences $\\tvas$ and $\tvbs$ can be put in one-to-one correspondances.
Besides, since they all ordered in the order of appearance in terms, they
the correspondance respects the ordering. Hence, the subsitution $\where
{\tvbs \is \tvas}$ is a renaming. Therefore, we can assume \Wlog that
$\tvbs$ is $\tvas$,
\relax
That is, \lref P becomes that $\tpoly {\all \tvs \t}$ is equal to $\tpoly
{\all \tvs \typ \where {\tvcps \is \typs}}$, which given that they $\tvs$
apprear in the same order in both terms, implies that $\t$ is equal to $\typ
\where {\tvcps \is
\typs}$~\llabel T.

\relax

\medskip

Since $\typs$ does not contain any variable in $\tvs$, every path $\pi_i$
is a path in $\typ$. Thus, we may write $\typ$ as
\relax $\typ \where {\pi_i \is \tyi''}\iton$ where $\tyi''$ is $\typ/\pi_i$.
This is also equal to
\relax $(\typ \where {\pi_i \is \tvci}\iton) \where {\tvci \is \tyi''}\iton$,
that is $\tyz\where {\tvci \is \tyi''}\iton$.
%
In summary, we have $\typ$ is equal to
\relax $\tyz \where {\tvci \is \tyi''}\iton$,
which implies that
\relax  $\tpoly {\all \tvs \typ}$ is equal to
\relax  $\tpoly {\all \tvs {\tyz \where {\tvci \is \tyi''}\iton}}$, \ie
\relax  $\tpoly {\all \tvs \tyz} \where {\tvci \is \tyi''}\iton$~\llabel E.
%
By Rule \Rule {Inst-Shape}, we have
\begin{mathpar}[inline]
\any \tvcs  \tpoly {\all \tvs \tyz} \preceq
\any \tvcps\tpoly {\all \tvs \tyz} \where {\tvci \is \tyi''}\iton,
\end{mathpar}
which, given~\lref E, is exactly~\lref C.



\subsection{Canonicalization of satisfiability}

This section aims at proving the canonicalization of satisfiability, that is,
for any $\semenv \th \c$, there is a canonical derivation $\semenv \Th \c$.
%
We begin by proving some auxillary lemmas regarding inversion and
entailment in our system.

\begin{lemma}
  \label{lem:match-is-not-simple}
  For any constraint context $\C\where{\square}$,
  the constraint $\C\where{\cmatch \t \cbrs}$ is not simple.
  \begin{proof}
    Structural induction on $\C$.
  \end{proof}
\end{lemma}

\newcommand{\simplePf}[2]{\Pf{}{}{#1 \simple}{#2}}
\newcommand{\nsimplePf}[2]{\Pf{}{\neg}{#1 \simple}{#2}}
\newcommand{\shapePf}[4]{\Pf{}{}{\Cshape {#1} {#2} {#3}}{#4}}

\begin{lemma}[Simple inversion]~
  \label{lem:simple-inversion}
  \begin{enumerate}[(\roman*)]
    \item If $\semenv \th \cunif \tone \ttwo$, then $\semenv(\tone) = \semenv(\ttwo)$.
    \item If $\semenv \th \ueq$, then $\semenv(\t) = \gt$ for all $\t \in \ueq$ and some ground type $\gt$.
    \item If $\semenv \th \capp \x \t$, then $\semenv(\t) \in \semenv(\x)$.
    \item If $\semenv \th \cpapp \x \ren \ueqs$, then $\semenv(\x) = \gabsr \tv \tvs$, $\dom \ren \subseteq \tvs$, and \\$\semenv\where{\tv \is \semenv(\t)} \th \ueqs\where\ren \implies \semenv(\t) \in \glam$.

    \item If $\semenv \th \cone \cand \ctwo$ and $\cone \cand \ctwo \simple$, then $\semenv \th \cone$ and $\semenv \th \ctwo$.
    \item If $\semenv \th \cexists \tv \c$ and $\cexists \tv \c \simple$, then $\semenv\where{\tv \is \gt} \th \c$ for some $\gt$.
    \item If $\semenv \th \cfor \tv \c$ and $\cfor \tv \c \simple$, then $\semenv\where{\tv \is \gt} \th \c$ for all ground types $\gt$.
    \item If $\semenv \th \clet \x \tv \cone \ctwo$ and $\clet \x \tv \cone \ctwo \simple$, then $\semenv \th \cexists \tv \cone$ and \\$\semenv\where{\x \is \semenv(\cabs \tv \cone)} \th \ctwo$.
    \item If $\semenv \th \cletr \x \tv \tvs \cone \ctwo$ and $\cletr \x \tv \tvs \cone \ctwo \simple$, then
      $\semenv \th \cexists {\tv, \tvs} \cone$ and \\$\semenv\where{\x \is \semenv(\cabsr \tv \tvs \cone)} \th \ctwo$.
  \end{enumerate}
  \begin{proof}~
    \begin{enumerate}[(\roman*)]
      \item Case analysis on the given derivation $\semenv \th \cunif \tone \ttwo$.
	There is a unique case for the atomic constraint $\cunif \tone \ttwo$:
	\begin{itemize}
	    \proofcasederivation
	      {Unif}
	      {\semenv(\tone) = \semenv(\ttwo)}
	      {\semenv \th \cunif \tone \ttwo}

	    \begin{llproof}
\Hand 		\eqPf{\semenv(\tone)}{\semenv(\ttwo)}  {Premise}
	    \end{llproof}
	\end{itemize}

      \item Similar to \Rule{Unif} case.
      \item Similar to \Rule{Unif} case.
      \item Similar to \Rule{Unif} case.

      \item Case analysis on the given derivation $\semenv \th \cone \cand \ctwo$.
      \begin{itemize}
	\proofcasederivation
	  {Conj}
	  {\semenv \th \cone \\ \semenv \th \ctwo}
	  {\semenv \th \cone \cand \ctwo}

	\begin{llproof}
\Hand 	  \vdashPf{\semenv}{\cone} {Premise}
\Hand     \vdashPf{\semenv}{\ctwo} {Premise}
	\end{llproof}

	\proofcasederivation
	  {Susp-Ctx}
	  {\Cshape \C \t \sh \\ \semenv \th \C\where{\cmatched \t \sh \cbrs}}
	  {\semenv \th \underbrace{\C\where{\cmatch \t \cbrs}}_{\cone \cand \ctwo}}

	\begin{llproof}
	  \simplePf{\C\where{\cmatch \t \cbrs}} {Given}
	  \nsimplePf{\C\where{\cmatch \t \cbrs}} {By \cref{lem:match-is-not-simple}}
\Hand 	  \contraPf{\semenv \th \cone, \semenv \th \ctwo}
	\end{llproof}

      \end{itemize}

      \item Similar to \Rule{Conj} case.
      \item Similar to \Rule{Conj} case.
      \item Similar to \Rule{Conj} case.
      \item Similar to \Rule{Conj} case.
    \end{enumerate}
  \end{proof}
\end{lemma}

\begin{lemma}[Composability of simplicity]
  \label{lem:compose-simple}
  If $\c \simple$ and $\C \simple$, then $\C\where\c \simple$.
  Additionally, if $\Ca \simple$ and $\Cb \simple$, then $\Ca\where\Cb \simple$.
  \begin{proof}
    Induction on the derivation of $\C \simple$ and $\Ca \simple$, respectively.
  \end{proof}
\end{lemma}

\begin{lemma}[Simple congruence]
  \label{lem:cong-simple}
  Given simple constraints $\cone, \ctwo$ and simple context $\C$.
  If \\$\cone \centails \ctwo$, then $\C\where{\cone} \centails \C\where{\ctwo}$.
  \begin{proof}
    Induction on the derivation of $\C \simple$.
  \end{proof}
\end{lemma}

\begin{lemma}[Erasure is simple]
  \label{lem:erase-simple}
  For all constraints $\c$, $\cerase \c \simple$.
  \begin{proof}
    Induction on the structure of $\c$.
  \end{proof}
\end{lemma}



\begin{lemma}[Composability of unicity]
  \label{lem:compose-unicity}
  If $\Cshape \Ca \t \sh$, then $\Cshape {\Cb\where\Ca} \t \sh$.
  \begin{proof}
    Induction on the structure of $\Cb$.
    \begin{itemize}
      \proofcase{$\square$}

	\begin{llproof}
	  \shapePf{\Ca}{\t}{\sh}{Premise}
	  \eqPf{\Cb\where\Ca}{\square\where\Ca}{$\Cb$ is $\square$}
	  \continueeqPf{\Ca}{By definition}
\Hand	  \shapePf{\square\where\Ca}{\t}{\sh}{From premise}
	\end{llproof}

      \proofcase{$\Cc \cand \c$}

	\begin{llproof}
	  \shapePf{\Ca}{\t}{\sh}{Premise}
	  \shapePf{\Cc\where\Ca}{\t}{\sh}{By \ih}
	  \ForallPf{\semenv, \gt}{}{Definition of $\Cshape {\parens{\Cc\where\Ca \cand \c}} \t \sh$}
	  \vdashPf{\semenv}{\cerase {\Cc\where\Ca\where{\cunif \t \gt} \cand \c}}{$\implies$I}
	  \eqPf{\cerase {\Cc\where\Ca\where{\cunif \t \gt} \cand \c}}{\cerase {\Cc\where\Ca\where{\cunif \t \gt}} \cand \cerase \c}{By definition}
	  \simplePf{\cerase {\Cc\where\Ca\where{\cunif \t \gt} \cand \c}}{\cref{lem:erase-simple}}
	  \vdashPf{\semenv}{\cerase {\Cc\where\Ca\where{\cunif \t \gt}}}{\cref{lem:simple-inversion}}
	  \eqPf{\shape \gt}{\sh}{$\implies$E on $\Cshape {\Cc\where\Ca} \t \sh$}
\Hand 	  \shapePf{\parens{\Cc\where\Ca \cand \c}}{\t}{\sh}{Above}
	\end{llproof}

      \proofcase{$\c \cand \Cc$}

	\begin{llproof}
	  Similar to the $\Cc \cand \c$ case.
	\end{llproof}

      \proofcase{$\cexists \tv \Cc$}

	\begin{llproof}
	  \shapePf{\Ca}{\t}{\sh}{Premise}
	  \shapePf{\Cc\where\Ca}{\t}{\sh}{By \ih}
	  \ForallPf{\semenv, \gt}{}{Definition of $\Cshape {\parens{\cexists \tv \Cc\where\Ca}} \t \sh$}
	  \vdashPf{\semenv}{\cerase {\cexists \tv \Cc\where\Ca\where{\cunif \t \gt}}}{$\implies$I}
	  \eqPf{\cerase {\cexists \tv \Cc\where\Ca\where{\cunif \t \gt}}}{\cexists \tv \cerase {\Cc\where\Ca\where{\cunif \t \gt}}}{By definition}
	  \simplePf{\cerase {\cexists \tv \Cc\where\Ca\where{\cunif \t \gt}}}{\cref{lem:erase-simple}}
	  \vdashPf{\semenv\where{\tv \is \gtp}}{\cerase {\Cc\where\Ca\where{\cunif \t \gt}}}{\cref{lem:simple-inversion}}
	  \eqPf{\shape \gt}{\sh}{$\implies$E on $\Cshape {\Cc\where\Ca} \t \sh$}
\Hand 	  \shapePf{\parens{\cexists \tv \Cc\where\Ca}}{\t}{\sh}{Above}
	\end{llproof}

      \proofcase{$\cfor \tv \Cc$}

	\begin{llproof}
	  Similar to $\cexists \tv \Cc$ case.
	\end{llproof}

      \proofcase{$\clet \x \tv \Cc \c$}

	\begin{llproof}
	  \shapePf{\Ca}{\t}{\sh}{Premise}
	  \shapePf{\Cc\where\Ca}{\t}{\sh}{By \ih}
	  \ForallPf{\semenv, \gt}{}{Definition of $\Cshape {\parens {\Let \x \ldots}} \t \sh$}
	  \vdashPf{\semenv}{\cerase {\clet \x \tv {\Cc\where\Ca\where{\cunif \t \gt}} \c}}{$\implies$I}
	  &&$\cerase {\clet \x \tv {\Cc\where\Ca\where{\cunif \t \gt}} \c}$ & \\
	  &&$=\clet \x \tv {\cerase {\Cc\where\Ca\where{\cunif \t \gt}}} {\cerase \c}$ & {By definition} \\
	  \simplePf{\cerase {\clet \x \tv {\Cc\where\Ca\where{\cunif \t \gt}} \c}}{\cref{lem:erase-simple}}
	  \vdashPf{\semenv}{\cexists \tv \cerase {\Cc\where\Ca\where{\cunif \t \gt}}}{\cref{lem:simple-inversion}}
	  \simplePf{\cerase {\Cc\where\Ca\where{\cunif \t \gt}}}{\cref{lem:erase-simple}}
	  \vdashPf{\semenv\where{\tv \is \gtp}}{\cerase {\Cc\where\Ca\where{\cunif \t \gt}}}{\cref{lem:simple-inversion}}
	  \eqPf{\shape \gt}{\sh}{$\implies$E on $\Cshape {\Cc\where\Ca} \t \sh$}
\Hand 	  \shapePf{\parens{\clet \x \tv {\Cc\where\Ca} \c}}{\t}{\sh}{Above}
	\end{llproof}

      \proofcase{$\clet \x \tv \c \Cc$}

	\begin{llproof}
	  Similar to $\clet \x \tv \Cc \c$ case.
	\end{llproof}

      \proofcase{$\cletr \x \tv \tvs \Cc \c$}

	\begin{llproof}
	  Similar to $\clet \x \tv \Cc \c$ case.
	\end{llproof}

      \proofcase{$\cletr \x \tv \tvs \c \Cc$}

	\begin{llproof}
	  Similar to $\clet \x \tv \c \Cc$ case.
	\end{llproof}


    \end{itemize}
  \end{proof}
\end{lemma}

\begin{lemma}[Inversion of unicity]
  \label{lem:unicity-inversion}~
  \begin{enumerate}[(\roman*)]
    \item If $\Cshape {\parens{\cexists \tv \C}} \t \sh$, then $\Cshape \C \t \sh$.
    \item If $\Cshape {\parens{\cfor \tv \C}} \t \sh$, then $\Cshape \C \t \sh$.
  \end{enumerate}
  \begin{proof}
    Follows from \cref{lem:simple-inversion}.
  \end{proof}
\end{lemma}

\begin{lemma}[Decanonicalization]
  \label{lem:decanonicalization}
  If $\semenv \Th \c$, then $\semenv \th \c$.
  \begin{proof}
    Induction on the given derivation $\semenv \Th \c$
  \end{proof}
\end{lemma}

\Xgabriel{Note: I put the 'Measure' back here after removing the well-foundedness section, it may not be the right place to have it.}
\newcommand{\cnmatches}[1]{\#\Match #1}
\newcommand{\csize}[1]{|#1|}
\newcommand{\cmeasure}[1]{{\| #1 \|}}

\begin{definition}[Measure]
  For the relation $\semenv \th \c$, the following measure enables a useful
  induction principle:
    \begin{mathpar}
    \cmeasure \c \uad\eqdef\uad \angles{\cnmatches \c, \csize \c}
  \end{mathpar}
  where $\angles \ldots$ denotes a pair with lexicographic ordering, and:
  \begin{enumerate}

    \item $\cnmatches \c$ is the number of $\cmatch \t \cbrs$ constraints in
      $\c$.

    \item the last component $\csize \c$ is a structural measure of constraints \ie a
      conjunction $\cone \cand \ctwo$ is larger than the two conjuncts $\cone,
      \ctwo$.

  \end{enumerate}
\end{definition}

\newcommand{\VdashPf}[3]{\Pf{#1}{\Vdash}{#2}{#3}}
\begin{theorem}[Canonicalization]
  \label{thm:canonicalization}
  If $\semenv \th \c$, then $\semenv \Th \c$.
  \begin{proof}
  We proceed by induction on $\semenv \th \c$ with the measure $\cmeasure \c$.
  \begin{itemize}
    \proofcasederivation
      {True}
      {}
      {\semenv \th \ctrue}

      \begin{llproof}
	\simplePf{\ctrue}{By definition}
\Hand 	\VdashPf{\semenv}{\ctrue}{By \Rule{Can-Base}}
      \end{llproof}
    \proofcasederivation
      {Unif}
      {\semenv(\tone) = \semenv(\ttwo)}
      {\semenv \th \cunif \tone \ttwo}

      \begin{llproof}
	Similar to the \Rule{True} case.
      \end{llproof}
    \proofcasederivation
      {Conj}
      {\semenv \th \cone \\ \semenv \th \ctwo}
      {\semenv \th \cone \cand \ctwo}

      \begin{llproof}
	\vdashPf{\semenv}{\cone} {Premise}
	\vdashPf{\semenv}{\ctwo} {Premise}
	\VdashPf{\semenv}{\cone} {By \ih}
	\VdashPf{\semenv}{\ctwo} {By \ih}
	\decolumnizePf
	\casesPf{\semenv \Th \cone, \semenv \Th \ctwo}
      \end{llproof}

      \begin{itemize}
	\proofcasederivationdouble
	  {Can-Base}
	  {\semenv \th \cone \\ \cone \simple}
	  {\semenv \Th \cone}
	  {Can-Base}
	  {\semenv \th \ctwo \\ \ctwo \simple}
	  {\semenv \Th \ctwo}

	  \begin{llproof}
	    \simplePf{\cone}{Premise}
	    \simplePf{\ctwo}{Premise}
	    \simplePf{\cone \cand \ctwo}{By \Rule{Simple-Conj}}
\Hand 	    \VdashPf{\semenv}{\cone \cand \ctwo}{By \Rule{Can-Base}}

	  \end{llproof}

	\proofcasederivationdouble
	  {Can-Susp-Ctx}
	  {\Cshape \C \t \sh \\ \semenv \Th \C\where{\cmatched \t \sh \cbrs}}
	  {\semenv \Th \underbrace{\C\where{\cmatch \t \cbrs}}_\ca}
	  {}
	  {}
	  {\semenv \Th \cb}

	  \begin{llproof}

	    \VdashPf{\semenv}{\C\where{\cmatched \t \sh \cbrs}} {Premise}
	    \vdashPf{\semenv}{\C\where{\cmatched \t \sh \cbrs}} {\cref{lem:decanonicalization}}
	    \vdashPf{\semenv}{\C\where{\cmatched \t \sh \cbrs} \cand \cb}{By \Rule{Conj}}
 	    \VdashPf{\semenv}{\C\where{\cmatched \t \sh \cbrs} \cand \cb}{By \ih}
	    \Pf{}{}{\Cshape \C \tv \sh}{Premise}
	    \Pf{}{}{\Cshape {\parens {\C \cand \cb}} \tv \sh}{\cref{lem:compose-unicity}}
\Hand 	    \VdashPf{\semenv}{\C\where{\cmatch \t \cbrs}}{By \Rule{Can-Susp-Ctx}}
	  \end{llproof}

	\proofcasederivationdouble
	  {}
	  {}
	  {\semenv \Th \ca}
	  {Can-Susp-Ctx}
	  {\Cshape \C \t \sh \\ \semenv \Th \C\where{\cmatched \t \sh \cbrs}}
	  {\semenv \Th \underbrace{\C\where{\cmatch \t \cbrs}}_\cb}

	  \begin{llproof}
	    \Pf{}{}{}{Symmetric to the above case.}
	  \end{llproof}
      \end{itemize}

      \proofcasederivation
	{Exists}
	{\semenv\where{\tv \is \gt} \th \c}
	{\semenv \th \cexists \tv \c}

	\begin{llproof}
	  \vdashPf{\semenv\where{\tv \is \gt}}{\c}{Premise}
	  \VdashPf{\semenv\where{\tv \is \gt}}{\c}{By \ih}
	  \casesPf{\semenv\where{\tv \is \gt} \Th \c}
	\end{llproof}

	\begin{itemize}

	    \proofcasederivation
	      {Can-Base}
	      {\semenv\where{\tv \is \gt} \th \c \\ \c \simple}
	      {\semenv\where{\tv \is \gt} \Th \c}

	      \begin{llproof}
		\simplePf{\c}{Premise}
		\simplePf{\cexists \tv \c}{By \Rule{Simple-Exists}}
\Hand 		\VdashPf{\semenv}{\cexists \tv \c}{By \Rule{Can-Base}}
	      \end{llproof}


	      \proofcasederivation
		{Can-Susp-Ctx}
		{\Cshape \C \t \sh \\ \semenv\where{\tv \is \gt} \Th \C\where{\cmatched \t \sh \cbrs}}
		{\semenv \Th \underbrace{\C\where{\cmatch \t \cbrs}}_\c}

		\begin{llproof}
		  \VdashPf{\semenv\where{\tv \is \gt}}{\C\where{\cmatched \t \sh \cbrs}}{Premise}
		  \vdashPf{\semenv\where{\tv \is \gt}}{\C\where{\cmatched \t \sh \cbrs}}{\cref{lem:decanonicalization}}
		  \vdashPf{\semenv}{\cexists \tv \C\where{\cmatched \t \sh \cbrs}}{By \Rule{Exists}}
		  \VdashPf{\semenv}{\cexists \tv \C\where{\cmatched \t \sh \cbrs}}{By \ih}
		  \Pf{}{}{\Cshape \C \t \sh}{Premise}
		  \Pf{}{}{\Cshape {\parens {\cexists \tv \C}} \t \sh}{\cref{lem:compose-unicity}}
\Hand             \VdashPf{\semenv}{\cexists \tv \C\where{\cmatch \t \cbrs}}{By \Rule{Can-Susp-Ctx}}
		\end{llproof}
	\end{itemize}

	\proofcasederivation
	  {Forall}
	  {\forall \gt,~ \semenv\where{\tv \is \gt} \th \c}
	  {\semenv \th \cfor \tv \c}

	  \begin{llproof}
	    Similar to the \Rule{Exists} case.
	  \end{llproof}

	\proofcasederivation
	  {Let}
	  {\semenv \th \cexists \tv \cone \\ \semenv\where{\x \is \semenv(\cabs \tv \cone)} \th \ctwo}
	  {\semenv \th \clet \x \tv \cone \ctwo}

	  \begin{llproof}
	    \vdashPf{\semenv}{\cexists \tv \cone}{Premise}
	    \VdashPf{\semenv}{\cexists \tv \cone}{By \ih}
	    \vdashPf{\semenv\where{\x \is \semenv(\cabs \tv \cone)}}{\ctwo}{Premise}
	    \VdashPf{\semenv\where{\x \is \semenv(\cabs \tv \cone)}}{\ctwo}{By \ih}
	    \decolumnizePf
	    \casesPf{\semenv \Th \cexists \tv \cone, \semenv\where{\x \is \semenv(\cabs \tv \cone)} \Th \ctwo}
	  \end{llproof}

	  \begin{itemize}
	    \proofcasederivationdouble
	      {Can-Base}
	      {\semenv \th \cexists \tv \cone \\ \cexists \tv \cone \simple}
	      {\semenv \Th \cexists \tv \cone}
	      {Can-Base}
	      {\semenv\where{\x \is \semenv(\cabs \tv \cone)} \th \ctwo \\ \ctwo \simple}
	      {\semenv\where{\x \is \semenv(\cabs \tv \cone)} \Th \ctwo}

	      \begin{llproof}
		\simplePf{\cexists \tv \cone}{Premise}
		\simplePf{\cone}{Inversion of \Rule{Simple-Exists}}
		\simplePf{\ctwo}{Premise}
		\simplePf{\clet \x \tv \cone \ctwo}{By \Rule{Simple-Let}}
\Hand		\VdashPf{\semenv}{\clet \x \tv \cone \ctwo}{By \Rule{Can-Base}}
	      \end{llproof}

	    \proofcasederivationdouble
	      {Can-Susp-Ctx}
	      {\Cshape {\parens {\cexists \tv \cone}} \t \sh \\ \semenv \Th \cexists \tv \C\where{\cmatched \t \sh \cbrs}}
	      {\semenv \Th \cexists \tv \underbrace{\C\where{\cmatch \t \cbrs}}_\ca}
	      {}
	      {}
	      {\semenv\where{\x \is \semenv(\cabs \tv \cone)} \Th \ctwo}

	      \begin{llproof}
		\shapePf{\parens {\cexists \tv \C}}{\t}{\sh}{Premise}
		\shapePf{\C}{\t}{\sh}{\cref{lem:unicity-inversion}}
		\VdashPf{\semenv}{\cexists \tv \C\where{\cmatched \t \sh \cbrs}}{Premise}
		\vdashPf{\semenv}{\cexists \tv \C\where{\cmatched \t \sh \cbrs}}{\cref{lem:decanonicalization}}
		\eqPf{\semenv(\cabs \tv \cone)}
		  {\semenv(\cabs \tv \C\where{\cmatched \t \sh \cbrs})}
		  {\cref{corollary:matched-abstractions}}
		\vdashPf{\semenv}{\clet \x \tv {\C\where{\cmatched \t \sh \cbrs}} \ctwo}{By \Rule{Let}}
		\VdashPf{\semenv}{\clet \x \tv {\C\where{\cmatched \t \sh \cbrs}} \ctwo}{By \ih}
		\shapePf{\parens{\clet \x \tv \C \ctwo}}{\t}{\sh}{\cref{lem:compose-unicity}}
\Hand 		\VdashPf{\semenv}{\clet \x \tv {\C\where{\cmatch \t \cbrs}} \ctwo}{By \Rule{Can-Susp-Ctx}}
	      \end{llproof}

	    \proofcasederivationdouble
		{}
		{}
		{\semenv \Th \cexists \tv \cone}
		{Can-Susp-Ctx}
		{\Cshape \C \t \sh \\ \semenv\where{\x \is \semenv(\cabs \tv \cone)} \Th \C\where{\cmatched \t \sh \cbrs}}
		{\semenv\where{\x \is \semenv(\cabs \tv \cone)} \Th \underbrace{\C\where{\cmatch \t \cbrs}}_\cb}

		\begin{llproof}
		  \shapePf{\C}{\t}{\sh}{Premise}
		  \shapePf{\parens{\clet \x \tv \cone \C}}{\t}{\sh}{\cref{lem:compose-unicity}}
		  \VdashPf{\semenv\where{\x \is \semenv(\cabs \tv \cone)}}{\C\where{\cmatched \t \sh \cbrs}}{Premise}
		  \vdashPf{\semenv\where{\x \is \semenv(\cabs \tv \cone)}}{\C\where{\cmatched \t \sh \cbrs}}{\cref{lem:decanonicalization}}
		  \vdashPf{\semenv}{\clet \x \tv \cone {\C\where{\cmatched \t \sh \cbrs}}}{By \Rule{Let}}
		  \VdashPf{\semenv}{\clet \x \tv \cone {\C\where{\cmatched \t \sh \cbrs}}}{By \ih}
\Hand 		  \VdashPf{\semenv}{\clet \x \tv \cone {\C\where{\cmatch \t \sh}}}{By \Rule{Can-Susp-Ctx}}
		\end{llproof}
	  \end{itemize}

      \proofcasederivation
	{App}
	{\semenv(\t) \in \semenv(\x)}
	{\semenv \th \capp \x \t}

      \begin{llproof}
	Similar to the \Rule{True} case.
      \end{llproof}

      \proofcasederivation
	{LetR}
	{\semenv \th \cexists {\tv, \tvs} \cone \\
	 \semenv\where{\x \is \semenv(\cabsr \tv \tvs \cone)} \th \ctwo}
	{\semenv \th \cletr \x \tv \tvs \cone \ctwo}

      \begin{llproof}
	Similar to the \Rule{Let} case.
      \end{llproof}

      \proofcasederivation
	{AppR}
	{\semenv(\x) = \gabsr \tv \tvs \\
	 \semenv(\t) \in \glam}
	{\semenv \th \capp \x \t}

      \begin{llproof}
	Similar to the \Rule{App} case.
      \end{llproof}

    \proofcasederivation
      {Multi-Unif}
      {\forall \t \in \ueq,~ \semenv(\t) = \gt}
      {\semenv \th \ueq}

      \begin{llproof}
	Similar to the \Rule{Unif} case.
      \end{llproof}

    \proofcasederivation
      {Papp}
      {\semenv(\x) = \gabsr \tv \tvs \\ \dom \ren \subseteq \tvs \\
       \semenv\where{\tv \is \semenv(\t)} \th \ueqs\where\ren \implies \semenv(\t) \in \glam}
      {\semenv \th \cpapp \x \ren \ueqs \t}


      \begin{llproof}
	Similar to the \Rule{App} case.
      \end{llproof}


  \end{itemize}
  \end{proof}
\end{theorem}

\newcommand{\entailsPf}[3]{\Pf{#1}{\centails}{#2}{#3}}
\begin{lemma}[Inversion of suspension]
  \label{lem:susp-inversion}
  If $\semenv \th \C\where{\cmatch \t \cbrs}$ and $\Cshape \C \t \sh$,
  then\\$\semenv \th \C\where{\cmatched \t \sh \cbrs}$.

  \begin{proof}
    We use canonicalization (\cref{thm:canonicalization}) to induct on $\semenv \Th
    \C\where{\cmatch \t \cbrs}$ instead of $\semenv \th \C\where{\cmatch \t
    \cbrs}$.

    This simplifies the proof, but introduces a circular dependency between
    \cref{thm:canonicalization} and \cref{lem:susp-inversion}.
    %
    However, this does not compromise the well-foundedness of induction, as the
    application of \cref{lem:susp-inversion} (via
    \cref{corollary:matched-abstractions}) within the proof of
    \cref{thm:canonicalization} is restricted to strictly smaller constriaints.

    \begin{itemize}
      \proofcasederivation
	{Can-Base}
	{\semenv \th \C\where{\cmatch \t \cbrs} \\ \C\where{\cmatch \t \cbrs} \simple}
	{\semenv \Th \C\where{\cmatch \t \cbrs}}

	\begin{llproof}
	  \simplePf{\C\where{\cmatch \t \cbrs}}{Premise}
	  \nsimplePf{\C\where{\cmatch \t \cbrs}}{\cref{lem:match-is-not-simple}}
\Hand 	  \contraPf{\semenv \th \C\where{\cmatched \t \sh \cbrs}}{}
	\end{llproof}

      \proofcasederivation
	{Can-Susp-Ctx}
	{\Cshape \Cp \tp \shp \\ \semenv \Th \Cp\where{\cmatched \tp \shp \cbrs'}}
	{\semenv \Th \underbrace{\Cp\where{\cmatch \tp \cbrs'}}_{\C\where{\cmatch {~\t~} {~\cbrs}}}}

	\begin{llproof}
	  \casesPf{\C = \Cp}
	\end{llproof}
	\begin{itemize}
	  \proofcase{$\C = \Cp$}

	    \begin{llproof}
	      \eqPf{\C}{\Cp}{Premise}
	      \eqPf{\tp}{\t}{}
	      \eqPf{\shp}{\sh}{}
	      \eqPf{\cbrs'}{\cbrs}{}
\Hand         \VdashPf{\semenv}{\C\where{\cmatched \t \sh \cbrs}}{Premise}
	    \end{llproof}

	  \proofcase{$\C \neq \Cp$}

	    \TODO{The formatting here is really off. Mainly due to the janky match notation
	          we have. }

	    \newcommand{\Ctwo}{\C_2}
	    \begin{llproof}
	      \eqPf{\Ctwo\where{\cmatch \t \cbrs, \cmatch \tp \cbrs'}}{\C\where{\cmatch \t \cbrs}}{For some 2-hole context $\Ctwo$}
	      \continueeqPf{\Cp\where{\cmatch \tp \cbrs'}}{}
	      \decolumnizePf
	      \VdashPf{\semenv}{\Ctwo\where{\cmatch \t \cbrs, \cmatched \tp \shp \cbrs'}}{Premise}
	      \decolumnizePf
	      \ForallPf{\semenvp, \gtp}{}{\hspace{32.5ex}Defn. of $\Cshape {\Ctwo\where{\square, \cmatched \tp \shp \cbrs'}} \t \sh$}
	      \decolumnizePf
	      \vdashPf{\semenvp}{\cerase {\Ctwo\where{\cunif \t \gtp, \cmatched \tp \shp \cbrs'}}}{$\implies$I}
	      \vdashPf{\semenvp}{\cerase {\Ctwo\where{\cunif \t \gtp, \ctrue}}}{\cref{lem:cong-simple}}
	      \eqPf{\cerase {\Ctwo\where{\cunif \t \gtp, \ctrue}}}{\cerase {\Ctwo\where{\cunif \t \gtp, \cerase {\cmatch \tp \cbrs'}}}}{By definition}
	      \continueeqPf{\cerase {\C\where{\cunif \t \gtp}}}{By definition}
	      \vdashPf{\semenvp}{\cerase {\C\where{\cunif \t \gtp}}}{Above}
	      \eqPf{\shape \gtp}{\sh}{$\implies$E on $\Cshape \C \t \sh$}
	      \shapePf{\Ctwo\where{\square, \cmatched \tp \shp \cbrs'}}{\t}{\sh}{Above}
	      \VdashPf{\semenv}{\Ctwo\where{\cmatched \t \sh \cbrs, \cmatched \tp \shp \cbrs'}}{By \ih}
	      \decolumnizePf
	      \ForallPf{\semenvp, \gtp}{}{\hspace{32.5ex}Defn. of $\Cshape {\Ctwo\where{\cmatched \t \sh \cbrs, \square}} \tp \shp$}
	      \decolumnizePf
	      \vdashPf{\semenvp}{\cerase {\Ctwo\where{\cmatched \t \sh \cbrs, \cunif \tp \gtp}}}{$\implies$I}
	      \vdashPf{\semenvp}{\cerase {\Ctwo\where{\ctrue, \cunif \tp \gtp}}}{\cref{lem:cong-simple}}
	      \eqPf{\cerase {\Ctwo\where{\ctrue, \cunif \tp \gtp}}}{\cerase {\Ctwo\where{\cerase {\cmatch \t \cbrs}, \cunif \tp \gtp}}}{By definition}
	      \continueeqPf{\cerase {\Cp\where{\cunif \tp \gtp}}}{By definition}
	      \vdashPf{\semenvp}{\cerase {\C\where{\cunif \t \gtp}}}{Above}
	      \shapePf{\Cp}{\tp}{\shp}{Premise}
	      \eqPf{\shape \gtp}{\shp}{$\implies$E on $\Cshape \Cp \tp \shp$}
	      \shapePf{\Ctwo\where{\cmatched \t \sh \cbrs, \square}}{\tp}{\shp}{Above}
\Hand 	      \VdashPf{\semenv}{\Ctwo\where{\cmatched \t \sh \cbrs, \cmatch \tp \cbrs'}}{By \Rule{Con-Susp-Ctx}}
	    \end{llproof}
	\end{itemize}
    \end{itemize}
  \end{proof}
\end{lemma}

\begin{corollary}
  \label{corollary:matched-abstractions}
  If $\Cshape \C \t \sh$, then $\semenv(\cabs \tv \C\where{\cmatch \t \cbrs}) = \semenv(\cabs \tv \C\where{\cmatched \t \sh \cbrs})$.
  Similarly, $\semenv(\cabsr \tv \tvs \C\where{\cmatch \t \cbrs}) = \semenv(\cabsr \tv \tvs \C\where{\cmatched \t \sh \cbrs})$.
  \begin{proof}
    It is sufficient to show that $\semenv\where{\tv \is \gt} \th \C\where{\cmatch \t \cbrs}$ if and only if
    $\semenv \th \C\where{\cmatched \t \sh \cbrs}$.

    \begin{itemize}
      \proofcase{$\implies$}

	\begin{llproof}
	  \shapePf{\C}{\t}{\sh}{Premise}
	  \vdashPf{\semenv\where{\tv \is \gt}}{\C\where{\cmatch \t \cbrs}}{Premise}
\Hand 	  \vdashPf{\semenv\where{\tv \is \gt}}{\C\where{\cmatched \t \sh \cbrs}}{\cref{lem:susp-inversion}}
	\end{llproof}
      \proofcase{$\impliedby$}

	\begin{llproof}
	  \shapePf{\C}{\t}{\sh}{Premise}
	  \vdashPf{\semenv\where{\tv \is \gt}}{\C\where{\cmatched \t \sh \cbrs}}{Premise}
\Hand 	  \vdashPf{\semenv\where{\tv \is \gt}}{\C\where{\cmatch \t \cbrs}}{By \Rule{Susp-Ctx}}
	\end{llproof}
    \end{itemize}

    For $\semenv(\cabsr \tv \tvs \C\where{\cmatch \t \cbrs}) = \semenv(\cabsr \tv \tvs \C\where{\cmatched \t \sh \cbrs})$, the proof is identical.
  \end{proof}
\end{corollary}


\section{Examples of suspended match constraints}

\Xgabriel{We have a problem in this definition, due to the fact that $\phi$ also contains polymorphic schemes, which are not necessarily the same as the one at the point where we use the \Rule{Susp-Ctx} rule. For example when the current environment $\phi$ has $x : \forall \alpha. \tint$, one would expect that $\phi \vdash \capp x \alpha \cand \cmatch \alpha \dots$ is satisfiable as $\capp x \alpha$ is morally equivalent to $\alpha = \tint$, but technically it is not because we can pick $\phi' := [x := \forall \alpha. \tbool]$.}
\Xalistair{Actually it's okay if this example fails, the ``fix'' is to use a wider $\C$ that contains the $\mathsf{let}~ x$ definition as part of the constraint.}
\Xgabriel{This example/discussion should go in the Unicity Testsuite Appendix.}

%
%\section{Well-foundedness of satisfiability}

%\section{Well-foundedness of \OML typing rules}
%
%\Xalistair{The same problem of well-foundedness comes up
%with the \OML typing rules. Should we have a separate
%section? Or deal with it in the above section (A single
%section for well-foundedness concerns)}
%
%\section{Ordered substitutions and unifiers}
%
%An ordered substitution $\sub$ is a refinement on substitutions, which
%defines the scope of existential variables in solutions and controls the free
%variables of their solutions.
%
%\begin{mathpar}
%  \begin{bnfgrammar}
%    \entry[Ordered substitution]{\sub}{
%      \cdot
%      \and \sub, \tv
%      \and \sub, \tv := \t
%      \and \sub, \x := \cabs \tv \c
%    }
%  \end{bnfgrammar}
%\end{mathpar
%Like typing contexts $\G$, substitutions $\sub$ are ordered and contain
%delcarations of polymorphic variables ($\tv$), they also provide term variable
%bindings ($\x \is \cabs \tv \c$). Unlike typing contexts, ordered substitutions
%$\sub$ also contain declarations of existential (flexible) type variables with
%solutions ($\tv \is \t$). We write $\dom \sub$ for the sequence of type
%variables and term variables defined in $\sub$.
%
%The well-formedness of substitutions $\sub$, written $\th \sub$ and defined
%below, enforces an order: if $\sub = \sub_L, \tv \is \t, \sub_R$, the solution
%$\t$ must be well-formed under $\sub_L$.
%
%\begin{mathpar}
%  \infer[Sub-Emp-Wf]
%    {}
%    {\th \eset}
%
%  \infer[Sub-TyVar-Wf]
%    {\th \sub}
%    {\th \sub, \tv}
%
%  \infer[Sub-TyAssn-Wf]
%    {\th \sub \\ \sub \th \t}
%    {\th \sub, \tv \is \t}
%
%  \infer[Sub-Var-Wf]
%    {\th \sub \\ \dom \sub \th \cabs \tv \c}
%    {\th \sub, \x \is \cabs \tv \c}
%\\
%  \infer[S-Var-Wf]
%    {\tv \in \dom \sub}
%    {\sub \th \tv}
%
%  \infer[S-Unit-Wf]
%    {}
%    {\sub \th \tunit}
%
%  \inferrule[S-Arr-Wf]
%    {\sub \th \t \\ \sub \th \tp}
%    {\sub \th \t \to \tp}
%
%  \inferrule[S-Prod-Wf]
%    {(\sub \th \ti)\iton}
%    {\sub \th \Pi\iton \ti}
%
%  \inferrule[S-Rcd-Wf]
%    {(\sub \th \ti)\iton \\
%     \T \in \dom \Omega}
%    {\sub \th \tys \T}
%
%  \inferrule[S-Poly-Wf]
%    {\sub \th \ts}
%    {\sub \th \tpoly \ts}
%
%  \inferrule[S-Forall-Wf]
%    {\sub, \tv \th \ts}
%    {\sub \th \tfor \tv \ts}
%
%\end{mathpar}
%Ordered substitutions $\sub$ can be viewed as parallel substitutions on types,
%substitutiong solved existential variables. We write $\sub(\t)$ for $\sub$
%applied as a substitution to type $\t$:
%\begin{mathpar}
%  \begin{tabular}{RCLL}
%    \sub(\tv) &\eqdef&
%      \begin{cases}
%	\theta(\t) &\text{if } \tv \is \t \in \theta \\
%	\tv &\text{otherwise}
%      \end{cases}\\
%    \sub(\tunit) &\eqdef& \tunit \\
%    \sub(\t \to \tp) &\eqdef& \sub(\t) \to \sub(\tp) \\
%    \sub(\Pi\iton \ti) &\eqdef& \Pi\iton \sub(\ti) \\
%    \sub(\tys \Tapp) &\eqdef& \sub(\tys) \Tapp \\
%    \sub(\tpoly \ts) &\eqdef& \tpoly {\sub(\ts)} \\
%    \sub(\tfor \tv \ts) &\eqdef& \tfor \tv \sub(\ts) &\text{if } \tv \disjoint
%    \sub
%  \end{tabular}
%\end{mathpar}
%
%We can interpret $\sub$ as a set of semantic environments, written $\csem
%\sub$.
%\TODO{Semantic environments should not be ordered}
%\begin{mathpar}
%  \begin{tabular}{RCL}
%      \csem \cdot &\eqdef& \set {\cdot} \\
%      \csem {\sub, \tv} &\eqdef& \set{\semenv[\tv \is \gt] : \semenv \in \csem
%      \sub} \\
%      \csem {\sub, \tv \is \t} &\eqdef& \set{\semenv[\tv \is \semenv(\t)] :
%	\semenv \in \csem \sub} \\
%      \csem {\sub, \x \is \cabs \tv \c} &\eqdef& \set{\semenv[\x \is
%      \semenv(\cabs \tv \c)] : \semenv \in \csem \sub}
%  \end{tabular}
%\end{mathpar}
%
%We say that $\sub$ is a \emph{unifier} of $\c$ iff $\all {\semenv } \semenv
%\in \csem \sub \implies \semenv \th \c$.
%
%\begin{lemma}
%  If $\sub$ is unifier of $\cexists \tv \c$, there exists type $\t$ such that
%  $\sub \th \t$ and $\sub, \tv \is \t$ is a unifier of $\c$. Conversely,
%  if $\sub, \tv \is \t$ is a unifier of $\c$, then $\sub$ is a unifier of
%  $\cexists \tv \c$.
%
%
%\end{lemma}
%
%\TODO{Add some more stuff on ordered substitutions and unifiers (probably on
%an as need basis)}
%
%\section{Proofs for \cref{sec:constraint-gen}}
%
%In this appendix, we give the details for proving the soundness and completeness
%of the constraint generator, that is $\G \th \e : \ts$ if and only if
%$\cdot \th \cinfer {\G \th \e} \ts$. We begin by reformulating this statement using
%unifiers.
%%
%We write $\csem \G$ for the set of semantic environments induced by $\G$:
%\begin{mathpar}
%  \begin{tabular}{RCL}
%    \csem \cdot &\eqdef& \set \cdot \\
%    \csem {\G, \tv} &\eqdef& \set{\semenv[\tv \is \gt] : \semenv \in \csem \G} \\
%    \csem {\G, \x : \ts} &\eqdef& \set{\semenv[\x \is \semenv(\cabs \tv \ts \leq \tv)] : \semenv \in \csem \G}
%  \end{tabular}
%\end{mathpar}
%
%\begin{lemma}
%  $\cdot \th \csem{\G \th \e : \ts}$ if and only if $\csem \G \Vdash \cinfer \e \ts$
%  \begin{proof}
%    Induction on $\G$.
%  \end{proof}
%\end{lemma}
%
%We also relate $\G$ and ordered substitutions with the translation $\pparens \G$:
%\begin{mathpar}
%  \begin{tabular}{RCL}
%    \pparens \cdot &\eqdef& \cdot \\
%    \pparens {\G, \tv} &\eqdef& \pparens \G, \tv \\
%    \pparens {\G, \x : \ts} &\eqdef& \pparens \G, \x \is \cabs \tv \ts \leq \tv
%  \end{tabular}
%\end{mathpar}
%
%\begin{lemma}
%  $\csem \G = \csem {\pparens \G}$.
%  \begin{proof}
%    Induction on $\G$.
%  \end{proof}
%\end{lemma}
%
%We can now reformulate soundness and completeness as:
%$\G \th \e : \ts$ if and only if $\pparens \G$ is a
%unifier of $\cinfer \e \ts$.
%
%
%
%\begin{lemma}[X-soundness and completeness]
%  For the explicit term $\e$, we have
%  $\G \th \e : \ts$ if and only if $\pparens \G$ is a unifier of $\cinfer \e \ts$.
%  \begin{proof}
%    Structural induction on $\e$.
%  \end{proof}
%\end{lemma}
%
%\pagebreak
%\section{Unicity properties}
%
%\newcommand{\cscmmatch}[2]{\angles{#1}#2}
%\newcommand{\cscmmatchsub}[3]{\angles{#1 \is #2}#3}
%\newcommand{\Csshape}[4]{{#1}\where{\cscmmatch{#2}{#3} \uni #4}}
%\newcommand{\Cscshape}[7]{{#1}\where{\cscmmatch{#2}{#3} \uni {#4} \mid \csmatch{#5}{#6} \uni {#7}}}
%\newcommand{\cbrsp}{\cbrs'}
%
%In this section, we give details of the unicity predicate along with our main result: \emph{inversion of suspension}.
%We recall the following definition of unicity:
%
%\begin{definition}[Unicity]
%  The type variable $\tv$ has a uniquely known principal shape $\sh$ in the context $\C$,
%  written $\Cshape \C \tv \sh$ (or $\Csshape \C \tv \cbrs \sh$), iff for all assignments
%  $\semenv$ and ground types $\gt$, then $\C\where{\cunif \tv \gt}$ implies that the
%  principal shape of $\gt$ is equal to $\sh$:
%  \begin{mathpar}
%    \Csshape \C \tv \cbrs \sh \uad\eqdef\uad \all {\semenv, \gt} \; \semenv \th \C\where{\cunif \tv \gt} \implies \shape \gt = \sh
%  \end{mathpar}
%\end{definition}
%
%We introduce the notation $\Csshape \C \tv \cbrs \sh$ to be more consistent with our later notation on
%\emph{joint} and \emph{conditional} unicity.
%
%\begin{lemma}
%  If $\Csshape \Cb \tv \cbrs \sh$, then $\Csshape {\parens {\Ca\where\Cb}} \tv \cbrs \sh$.
%  \begin{proof}
%    Induction on $\Ca$.
%  \end{proof}
%\end{lemma}
%
%\begin{definition}[Conditional unicity] For two type variables $\tv, \tvp$, $\tv$ has the conditionally
%  known shape $\sh$ given that $\tvp$ has the shape $\shp$
%  within the match $\cscmmatch \tvp \cbrsp$, written
%  $\Cscshape \C \tv \cbrs \sh \tvp \cbrsp \shp$, is defined as:
%  \begin{mathpar}
%    \Cscshape \C \tv \cbrs \sh \tvp \cbrsp \shp \uad\eqdef\uad \Csshape {\parens{\C\where{\square, \cscmmatchsub \tvp \shp \cbrsp}}} \tv \cbrs \sh
%  \end{mathpar}
%
%  Here $\C$ is a two-hole context and $\C\where{\square, \cscmmatchsub \tvp \shp \cbrsp}$ is a one-hole context used in
%  the unicity predicate.
%\end{definition}
%
%
%\begin{theorem}[Bayes' theorem]
%  \begin{mathpar}
%    \Cscshape \C \tv \cbrs \sh \tvp \cbrsp \shp \wedge \Csshape {\parens{\C\where{\cscmmatch \tv \cbrs, \square}}} \tvp \cbrsp \shp \\
%    \iff \\
%    \Cscshape \C \tvp \cbrsp \shp \tv \cbrs \sh \wedge \Csshape {\parens{\C\where{\square, \cscmmatch \tvp \cbrsp}}} \tv \cbrs \sh
%  \end{mathpar}
%  \begin{proof}
%    ??
%  \end{proof}
%\end{theorem}
%
%
%\begin{theorem}[Inversion of suspension]
%  $\semenv \th \C\where{\cmatch \tv \cbrs}$ if and only if
%  $\Cshape \C \tv {\any \tvcs \t}$ and $\semenv \th \C\where{\cexists \tvcs \cunif \tv \t \cand \cmatch \t \cbrs}$ for some $\tvcs, \t$.
%  \begin{proof}
%    The backwards direction is trivial, using \Rule{Susp-Ctx}.
%    We now consider the forwards direction, by induction on the given derivation.
%
%    \proofcase{\Rule{True}, \Rule{Multi-Unif}, \Rule{App}, \Rule{Susp-Use}, \Rule{AppR}, \Rule{Papp}} Impossible.
%
%    \proofcase{\Rule{Conj}}
%      \begin{mathpar}
%	\infer[Conj]
%	  {\semenv \th \cone \\ \semenv \th \ctwo}
%	  {\semenv \th \cone \cand \ctwo}
%      \end{mathpar}
%      We have $\C\where{\cmatch \tv \cbrs} = \cone \cand \ctwo$. Consider
%      cases on $\C$:
%      \begin{itemize}
%	\item \proofcase{$\C = \Cp \cand \ctwo$}
%	  So we have $\semenv \th \Cp\where{\cmatch \tv \cbrs}$.
%	  By induction, we have $\Cshape \Cp \tv {\any \tvcs \t}$ and
%	  $\semenv \th \Cp\where{\cexists \tvcs \cunif \tv \tvcs \cand \cmatch \t \cbrs}$.
%	  By Lemma ??, we have $\Cshape {\parens {\Cp \cand \ctwo}} \tv {\any \tvcs \t}$.
%	  By \Rule{Conj}, we have
%	  \begin{mathpar}
%	    \infer[Conj]
%	      {\semenv \th \Cp\where{\cexists \tvcs \cunif \tv \tvcs \cand \cmatch \t \cbrs} \\
%	       \semenv \th \ctwo}
%	      {\semenv \th \parens{\Cp \cand \ctwo}\where{\cexists \tvcs \cunif \tv \tvcs \cand \cmatch \t \cbrs}}
%	  \end{mathpar}
%	\item \proofcase{$\C = \cone \cand \Cp$}
%	  Symmetric.
%      \end{itemize}
%
%    \proofcase{\Rule{Let}}
%      \begin{mathpar}
%	\infer[Let]
%	  {\semenv \th \cexists \tvb \cone \\
%	   \semenv, \x \is \semenv(\cabs \tvb \cone) \th \ctwo}
%	  {\semenv \th \clet \x \tvb \cone \ctwo}
%      \end{mathpar}
%      We have $\C\where{\cmatch \tv \cbrs} = \clet \x \tv \cone \ctwo$.
%      Consider cases on $\C$:
%      \begin{itemize}
%	\item \proofcase{$\C = \clet \x \tvb \Cp \ctwo$}
%	  So we have $\semenv \th \cexists \tvb \Cp\where{\cmatch \tv \cbrs}$ (1)
%	  and $\semenv, \x \is \semenv(\cabs \tvb \Cp\where{\cmatch \tv \cbrs}) \th \ctwo$ (2).
%
%	  By induction (1), we have $\cexists \tvb \Cshape \Cp \tv {\any \tvcs \t}$ and $\semenv \th \cexists \tvb
%	  \Cp\where{\cexists \tvcs \cunif \tv \t \cand \cmatch \t \cbrs}$. By Lemma ??, we have $\Cshape \Cp \tv {\any \tvcs \t}$.
%
%	  We now claim that $\semenv(\cabs \tvb \Cp\where{\cmatch \tv \cbrs}) = \semenv(\cabs \tvb \Cp\where{\cexists \tvcs \cunif \tv \t \cand \cmatch \t \cbrs})$.
%	  That is $\semenv, \tvb \is \gt \th \Cp\where{\cmatch \tv \cbrs} \iff \semenv, \tvb \is \gt \th \Cp\where{\cexists \tvcs \cunif \tv \t \cand \cmatch \t \cbrs}$.
%	  This follows directly from induction, as we have $\Cshape \Cp \tv {\any \tvcs \t}$.
%
%	  So we have $\semenv, \x \is \semenv(\cabs \tvb \Cp\where{\cexists \tvcs \cunif \tv \t \cand \cmatch \t \cbrs}) \th \ctwo$ from (2).
%	  Applying Lemma ??, we have $\Cshape {\parens {\clet \x \tvb \Cp \ctwo}} \tv {\any \tvcs \t}$, and applying the \Rule{Let} rule
%	  gives us:
%	  \begin{mathpar}
%	    \infer[Let]
%	      {\semenv \th \cexists \tvb \Cp\where{\cexists \tvcs \cunif \tv \t \cand \cmatch \t \cbrs} \\
%	       \semenv, \x \is \semenv(\cabs \tvb \Cp\where{\cexists \tvcs \cunif \tv \t \cand \cmatch \t \cbrs}) \th \ctwo}
%	      {\semenv \th \parens {\clet \x \tvb \Cp \ctwo} \where{\cexists \tvcs \cunif \tv \t \cand \cmatch \t \cbrs} }
%	  \end{mathpar}
%
%
%
%
%	\item \proofcase{$\C = \clet \x \tvb \cone \Cp$}
%	  Similiar to the \Rule{Conj} case.
%
%      \end{itemize}
%
%
%    \proofcase{\Rule{Exists}}
%      \begin{mathpar}
%	\infer[Exists]
%	  {\semenv, \tvb \is \gt \th \c}
%	  {\semenv \th \cexists \tvb \c}
%      \end{mathpar}
%
%      We have $\C\where{\cmatch \tv \cbrs} = \cexists \tvb \c$.
%      Hence $\C = \cexists \tvb \Cp$ and $\Cp\where{\cmatch \tv \cbrs} = \c$.
%
%      By induction, we have $\Cshape \Cp \tv {\any \tvcs \t}$
%      and $\semenv, \tvb \is \gt \th \Cp\where{\cexists \tvcs \cunif \tv \t \cand \cmatch \t \cbrs}$.
%
%      By Lemma ??, we have $\Cshape {\parens {\cexists \tvb \Cp}} \tv {\any \tvcs \t}$.
%      Applying \Rule{Exists} gives us:
%      \begin{mathpar}
%	\infer[Exists]
%	  {\semenv, \tvb \is \gt \th \Cp\where{\cexists \tvcs \cunif \tv \t \cand \cmatch \t \cbrs}}
%	  {\semenv \th \cexists \tvb \Cp\where{\cexists \tvcs \cunif \tv \t \cand \cmatch \t \cbrs}}
%      \end{mathpar}
%
%
%    \proofcase{\Rule{Forall}} Similiar to the \Rule{Exists} case.
%
%    \proofcase{\Rule{Susp-Ctx}}
%      \begin{mathpar}
%	\infer[Susp-Ctx]
%	  {\Cshape \Cp \tvp {\any \tvcs' \typ} \\
%	   \semenv \th \Cp\where{\cexists {\tvcs'} \cunif \tvp \typ \cand \cmatch \typ \cbrs'}}
%	  {\semenv \th \Cp\where{\cmatch \tvp \cbrs'}}
%      \end{mathpar}
%
%      Consider whether $\C = \Cp$:
%      \begin{itemize}
%	\item \proofcase{$\C = \Cp$} Trivial
%	\item \proofcase{$\C \neq \Cp$}
%	  Bayes' theorem + induction
%      \end{itemize}
%  \end{proof}
%\end{theorem}
%
%%
%%We write $\e^i$ for an \emph{implicit} term, which is either: $\eproj \e j$, $\einst \e$, $\epoly \e$.
%%Let us define $x(\e^i, \sh)$ for the explicit elaboration of an implicit term $\e^i$ using the
%%shape $\sh$:
%%\begin{align*}
%%  x(\eproj \e j, \any \tvcs \Pi\iton \tvcs) &= \eproj[n] \e j \\
%%  x(\einst \e, \any \tvcs \tpoly \ts) &= \exinst \e \tvcs \ts \\
%%  x(\epoly \e, \any \tvcs \tpoly \ts) &= \expoly \e {\exi \tvcs \ts}
%%\end{align*}
%%
%%\begin{theorem}
%%  $\G \th \E\where{\e^i} : \ts$ if and only if
%%  $\E\where{\e \mathop{\triangleleft\triangleright} \sh}$ and $\G \th \E\where{x(\e^i, \sh)} : \ts$ where
%%  $\decomp {\e^i} = \e, \triangleleft\triangleright$
%%  \begin{proof}
%%    \TODO{Unable to prove the hard cases yet (\eg \Rule{Proj-I}) --- this theorem is quite strong, }
%%  \end{proof}
%%\end{theorem}
%
%%% Below this line will is a draft
\Draft{}{\end{document}}\color{blue}

\section{DRAFT: a later TODO  list}

Problems to solve or leave unsolved:
\begin{itemize}

\item
  Overloading of the bracket notation for context filling and polytypes,
  as in $E\where{\epoly \e}$.

  A possibility would be to use braces for either one. Although they are
  used for record expressions, I would say the overlapping with either
  polytypes (they never appear simultaneously) or context (idem, since we
  put label accesses but not records in contexts.

  Alternatively, we could use $\ceils \e$ for polytypes---and then $\floors
  \e$ for the projections.


\end{itemize}


\end{document}

% LocalWords:  omnidirectional typecheck polymorphism Hindley Milner kinded
% LocalWords:  GADTs typechecked codomain typechecking subexpressions Bodin
% LocalWords:  monomorphic subexpression Dunfield Riboulet jfla subtyping
% LocalWords:  greek Chargueraud typable monotype polytype Garrigue Remy th
% LocalWords:  impredicative polytypes minimality RCL ary Proj toplevel
% LocalWords:  typability backpropagation arity Compositionality equi
% LocalWords:  equitypable compositionality inlined equitypability nullary
% LocalWords:  metatheoretical finiteness nonvariable
