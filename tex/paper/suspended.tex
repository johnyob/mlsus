%; whizzy section

%% Leave the above line for didier
%% No macros before \documentclass

\documentclass[acmsmall,screen,nonacm]{acmart}

%% \newcommand{\Comments}{\False}
\newcommand{\Draft}{\False}
\newcommand{\Long}{\True}
\newcommand{\Extended}{\False}
\newcommand{\Beamer}{\False}
\newcommand{\Anonymous}{\False}

\newcommand{\acmart}{\True}
\usepackage{suspended}

%% \Xfirstname defined in {mycomments}
%% Use either
%%   \Xfistname[text to comment]{your comment on the text}
%% or
%%   \Xfirstname{free comment}
%% Uncomment this line to hide all comments.
% \UNXXX{}

\usepackage{marginnote}

\title{Omnidirectional type inference with syspended constraints
   for static overloading and semi-explicit polymorphism}

\begin{document}

\begin{abstract}
We propose a new concept of \emph{omnidirectional} type inference, which is
the ability to resolve \ML-style typing constraints in disorder, by
contrast with all known implementations that always typecheck the
bindings before the bodies of let-expressions.
%
This relies on two technical devices: \emph{partial type schemes}
and \emph{suspended constraints}. Partial means  the
possibility of taking an instances of a type scheme that is not yet
completely solved with a mechanism to tack and update their instances when
the type scheme is refined, incrementally.
\emph{Suspended} constraints means a mechanism to freeze the resolution of
some constraints which typically require non-local not-yet-available
contextual information and to discharged then when we have learned enough
information from the resolution of other constraints, typically when some type
variables have been instantiated.
%
We illustrate the benefit of omnidirectional type inference for several
advanced features present in the \OCaml language, including static
overloading of record labels and datatype constructors on the one hand and
semi-explicit polymorphism on the other hand.
\end{abstract}

\maketitle


\section{Introduction}

ML polymorphism, also known as Hindley-Milner type system, has been a very
sweet spot in the design space of strongly typed programming languages, as
it enjoys the principal type property and more importantly is closed by
common program transformations such as let-contraction and let-expansion,
and permutation of the order or arguments.

Over the years, many extensions of \ML have been proposed.  While some of
them, such as flexible records, higher-kinded types, or dimensional types
fits nicely into the \ML framework, others such as GADTs or semi-explicit
polymorphism, or overloading of datatype constructors and record fields, are
less satisfactory, since they require to \emph{twist} the ML-style
specification with some direct or indirect algorithmic flavor in order to
preserve \emph{principality}.

All borderline extensions have in common that they introduce new construct
that require explicit type annotations that may sometimes but not always be
omitted. They can actually be viewed as an elaboration into the slightly weaker
language that makes annotations mandatory, but itself fits nicely into
the \ML framework. The elaboration then defines which annotations can be
omitted and rebuild from context.
%
The difficulty lies in finding a methodical such specification that comes
with a complete (hence principal) and effective elaboration algorithm.

The solutions proposed so far all enforce some ordering in which type
inference is performed, which can then be used to propagate both inferred
types and user-provided type annotations as \emph{known} types that can be
used for disambiguation and enable the omission of some annotations.

\paragraph{Bidirectional type inference}

Bidirectional type inference is a known alternative to unification for
propagating type information. It can be presented by adding an optional
expected type in addition to the expression to be typechecked and the
context in which it should be typechecked to the type inference algorithm.
Type inference is said in checking mode the expected type is present, and in
(the usual) inference mode when it is absent. The checking mode allows to
pass source type annotations but also previously inferred types such as the
codomain of a function in an application position during type inference of
the arguments.  The modes of operation are specified by typing rules.  Hence
bidirectional type inference is a framework that can be instantiated
differently, depending on the choice of modes in typing rules.
Bidirectional type inference has been largely used for languages with
higher-rank polymorphism or subtyping~\TODO{citations}. Still, both \OCaml
and \Haskell use a limited form of bidirectional type inference.



\paragraph{Poly-directional type inference}

\OCaml uses another mechanism for propagation of type
information, based on polymorphism, which actually enforces an ordering in
the typechecking of let-bindings, where the (polymorphic part of) a the
bound expression must be typed before the body of the let-expression.
We call it \emph{poly}-directional type inference.\Xdidier {let-directional?}

Initially introduced with semi-explicit polymorphism to \ML~\cite
{Garrigue-Remy/poly-ml} it has also been used for the overloading of record
fields. The user may define two records with overlapping fields:
\begin{program}[input]
type one = {x : int}
type 'a two = {x : 'a; y : int}
\end{program}
This can be thought of as defining two abstract types \ocaml{foo} and
\ocaml{bar} with two access \ocaml{(_:one).x} and \ocaml{(_:two).x} of
respective types \ocaml{(one -> int)} and \ocaml{(two -> int)}, among
others.
%
In the absence of overloading the latter definition would hide the former
and we fall back in \ML. With overloading enabled, both definitions are
visible and the compiler must statically choose one of them.
%
The question is which one to choose or, equivalently, which type to infer
in the following three expressions? 
\begin{program}[input]
let e_1 r = r.x
let e_2 = let r = {x = 1} in r.x
let e_3 = (fun r -> r.x) {x = 1} 
\end{program}
To be able to resolve the overloading, the type of the projection should be
\emph{known} to be either \ocaml{one} or \ocaml{'a two} when typing the
projection.  Expression \ocaml[indices]{e_1} is clearly ambiguous since
there is no clue on the type of \ocaml{r} and its typechecking
fails\footnote {In fact, \OCaml uses a default resolution strategy instead
of failing when the type is ambiguous, which is to use the last
definition. Here, this will amount to chose the type \ocaml{'a two}
still leading to a failure since the record passed to the function is
actually of type \ocaml{one}. But we shall ignore default resolution
strategies for the moment.}.
%
By contrast, the only possible
type for \ocaml{r} is \ocaml {one} in both \ocaml{e_2} and \ocaml {e_3}.
Still, there is a difference.  Indeed, the type of \ocaml{e_2} is considered
to be unambiguous, while the type of \ocaml{e_3} is ambiguous. To
understand why \ocaml{e_3} fails, consider the equivalent versions of
\ocaml{e_3} where \ocaml{@@} and \ocaml{|>} are the application and the
reverse application functions.
\begin{program}[escapechar={}]
let e_4 = (fun r -> r.x) @@ {x = 1} 
let e_5 = {x = 1}  |> (fun r -> r.x) 
\end{program}
\OCaml  does not make any difference between
\ocaml[indices]{e_1}, \ocaml[indices]{e_2}, or \ocaml[indices]{e_3}
and consider all subexpressions in an application, including the function
and all arguments as being inferred simultaneously, until they are
let-bound. More precisely, polymorphic types are considered to be known
while monomorphic types are consider to be unknown (or rather
not-yet-known). In \ocaml{fun r -> r.x}, the type of \ocaml{r} in
in the subexpression \ocaml{r.x} is not-yet-known even if equal to
\ocaml{one} since this is the lambda-bound variable \ocaml{r}, hence in the
typing context, hence it is monomorphic.\footnote{Technically, all type
nodes are annotated with special $\epsilon$ variables, so that we may
distinguish between the polymorphic binding $\xa : \all \kv. \t^\kv$ that
binds $r$ to the known (raw) type $\t$ and the monomorphic binding $\xb :
\t^\kv$ that binds $\xa$ to the unknown (raw) type $\t$.}


\paragraph{Limitations of directional type inference}

Directional type inference have been in used in different languages and
somehow proven to work. Still, each of them is unsatisfactory.  The main
critics of bidirectional type inference is to make a somewhat arbitrary
choice in the specification of the flow of information, typically in
applications: should the the function be typed first and its codomain be
used to improve the typing of its argument or, on the opposite, should the
argument be typed first and be used as the type of the codomain of the
function?  There are examples when the former is a better choice and others
when the latter is preferable---but typing rules have to choose one of the
two alternative forever.  One may also allow information to flow between
multiple arguments passed to the same function, or on the opposite disallow
it entirety. Again, there is not best choice. We eventually have a complete
algorithm with respect to a spec that made somewhat arbitrary choices.

Poly-directional type inference seems a better first for \ML, since it
relies on polymorphism which is the essence of \ML. Still, it does not even
allow to propagate user-type annotations! Thus \OCaml, which use
poly-directional type inference as the primary mechanisms also added a weak
form of bidirectional propagation. Besides, the implementation of
poly-directional type inference has an algorithmic cost, since the addition
of variable annotations requires to reread dags as trees destroying the
sharing that naturally resulting from unification, which may increase the
size of types and the cost of type inference. For that reason, the implement
of \OCaml cheats and is incomplete by default and the user must explicitly
require the most expensive computation of principal types when desired.

\paragraph{Incompleteness of directional type inference}

Both bidirectional and poly-directional type inference rely on an ordering
for type propagation that is partially specified, explicitly or implicitly,
to take advantage of user-provided type annotations and already-inferred types
to alleviated the need for extra annotations.

While they come with complete algorithms with respect to their
specification, their specifications include some choices that are subjective
and may sometimes look arbitrary.

Indeed, they all reject examples as ambiguous when sometimes they would be a
unique well-typed solution, as seen in the limitation of direction type
inference.  The first sources of incompleteness results from the ordering
enforced between the resolution of core-language constraints that prevents
some core information to be taken into account when resolving ambiguities.

However, the resolution of each ambiguity is another source of
incompleteness, which is actually two-fold:
\begin{enumerate}
\item  
  the resolution is not always taking all contextual type information into
  account.  
\item
  each ambiguity is solved independently of other yet unresolved ones;
\end{enumerate}
To illustrate the latter, the following example fails as ambiguous with
poly-directional inference, while \ocaml{r} can only be of type
\ocaml{int two}.  
\begin{program}[error]
let e_6 r = (r.x = r.y)
\end{program}
Indeed, \ocaml{r} is monomorphic when typing the projection.  This source of
incompleteness should wishfully be eliminated.

On the opposite, the former is a reasonable design to keep type inference
effective, since the complexity of general overloading is NP-hard even in
the absence of let-polymorphism,a s show by an encoding of 3-SAT problem as
shown by~\citet {Chargueraud-Bodin-Dunfield-Riboulet/jfla2025}.


\paragraph{Omnidirectional type inference}

We propose to use omnidirectional type inference to eliminate the first
source of incompleteness.  Our proposal with omnidirectional type inference
will correctly type all examples above (from \ocaml{e_1} to
\ocaml{e_6})\ignorespaces
\INFO {See \url
{https://gitlab.inria.fr/remy/semiun/-/blob/main/su/examples/record_fields.ml}}
---without any default strategy mechanism.

A type inference system without polymorphism, would be solely based on
unification constraints, which can be solved in any order.  The difficulty
originates from \ML let-polymorphism for which all known implementations
choose to always infer the type of a let-binding, turn it into a type scheme
that is assigned to the let-bound variable to extend the typing environment
in which the body is then typed. Algorithm $W$, its variant $M$ or more
flexible constraint solving algorithms all follow this strategy.

However, it is not mandatory.  In fact, typing constraints as presented
in~\cite {Pottier-Remy/emlti} can be instantiated before they are fully
solved, although this comes at the cost of duplicating some computation.
Still, this shows that the let-binding ordering is a convenience, not a
requirement.

Allowing typing constraint to be solved in disorder, type inference become
\emph{omnidirectional}.  This allows to take all program core-expression
constraints into account before resolving ambiguities.  This requires two
technical devices, hence two problems to solve:
\begin{enumerate*}
\item
  we first must work with partial types schemes, \ie. we the ability
  to instantiate type schemes that are not yet fully determined and
  consequently revisit their instances when their are being refined,
  incrementally.
\item
  we then must allow \emph{suspended constraints} as a way to freeze ambiguity
  resolution until sufficient information has been found from context so
  that they can be discharged and hopefully resolved.

\end{enumerate*}
Once the framework established, it just remains to be instantiated
to each use case,

\Xdidier {We should continue with a brief description of both mechanisms}

\paragraph {Refinement of instantiation constraints}

\paragraph {Suspended constraints}

\indent
\medskip\hrule\medskip

\Xdidier {Reconnect pierces of the old intro}


Generalization is the distinguishing feature of Hindley-Damas-Milner type
inference. Consider a subterm $\cletin{x = e_1}{e_2}$ of the program being
inferred. We first infer a type $\tau_1$ for $e_1$, and then we
\emph{generalize}: we look for variables $\bar{\alpha}$ that occur in
$\tau_1$, but nowhere else in the typing context, and we assign to the
variable $x$ the polymorphic type scheme $\tfor{\bar{\alpha}}{\tau_1}$. Each
occurrence of $x$ in $e_2$ can use a different choice of instances
$\bar{\tau'}$ of these polymorphic variables $\bar{\alpha}$, it will be
elaborated into $x [\bar{\tau'}]$ at type $\tau[\bar{\alpha} :=
\bar{\tau'}]$. In other words, generalization turns things that are
\emph{undetermined} in the definition $e_1$ into generic parameters that can
be chosen differently at each use-site $x$.

Qualified types~\citep*{TODO} represent additional knowledge on a type. They
are used in particular in Haskell type-classes, where a constraint
$\mathsf{Show}~\alpha$ represents the fact that a part of the inferred term
needs to print values of type $\alpha$. A constraint on a ground type such
as $\mathsf{Show}~\mathsf{Int}$ can be resolved to a known printer. But if
the undetermined variable $\alpha$ becomes generalizable in a type $\tau$,
we get a type-scheme $\tfor{(\alpha \mid \mathsf{Show}~\alpha)}{\tau}$ that
also includes the type-class constraint.

DRAFT: Type-classes let each use-site choose a different typeclass instance
-- but this implies dictionary-passing or specialization. Sometimes we want
the choice to be shared by all use-sites: TODO introduce suspended
constraints. But: generalization becomes tricky.
\begin{itemize}
\item easy case: the choice is resolved while typing the definition
\item hard case: an a-priori generalizable variable is captured in a frozen
   constraint 
\item question: what to do when the frozen variable is generalizable?
\item another difficulty: supporting default clauses in the failure case
\end{itemize}

Different kind of difficulties:
\begin{itemize}
\item finding a reasonable declarative semantics for the behavior we want
\item implementing a solver, which is quite hard
\end{itemize}

\Xalistair{Is this the right introduction? All the features we describe
below break principality usually. So our work is about principality. This is
a way to guarantee principal types for these advanced features.} 

\Xdidier{We would like to elaborate features with difficult type-checking
into a core language where things fit exactly the ML type-inference
approach. The source language has less annotations, the target language has
more precise annotations.} 

\Xalistair{For tuples we added a projection rule that gives the size of the
tuple. I imagine for polytypes there would be something like that.} 

\paragraph{Organization} 

\section{Suspended constraints}

\begin{mathparfig}%
  {fig:constraint-syntax}%
  {Syntax of types and constraints}
  \begin{array}{rrl@{\hspace{8em}}l}
    \alpha, \beta, \gamma &&& \text{(flexible) inference variables}
    \\
    \tau & ::=
         & \alpha \mid \overline{\alpha} \Fapp
         & \text{(shallow) inference types}
    \\
    \Fapp & ::=
          & \tunit \mid \cdot \tarrow \cdot \mid \dots
          & \text{type constructors}
    \\[2em]

    C & ::=
      & \ctrue
        \mid \cfalse
        \mid C_1 \cand C_2
        \mid \cexists \alpha C
        \mid \cunif {\tau_1} {\tau_2}
      & \text{Constraints}
      \\
      & \mid
      & \cletin {x = \clam \alpha {C_1}} {C_2}
        \mid \cinst x \tau
      &
      \\
      & \mid
      & \cmatch \alpha \Delta f
  \end{array}
\end{mathparfig}

PLAN: Only the constraint syntax, not constraint-solving yet. The point is to
show examples in a concrete way.

% Introduce syntax + explain components
A suspended match constraint has the form $\cmatch \alpha \Delta f$ where:
\begin{enumerate}
\item
  The matchee $\alpha$ is a type variable. The constraint is suspended until
  $\alpha$ is unified to a non-variable type.
\item
  The handler $f$ is a function from shallow types of the form $(\overline{\beta} \Fapp)$ to constraints.
  This
  generates the constraint to solve once $\alpha$ is unified to a non-variable type. One may think
  of this as the functionalization of the branches of the match.
\item
  The closure $\Delta$, the set of type variables $f$ is permitted to
  reference.
\end{enumerate}

% Informal semantics
Informally, the semantics of $\cmatch \alpha \Delta f$ are as follows: the
constraint remains suspended until the type variable $\alpha$ is unified
with a concrete non-variable type $\overline{\beta} \Fapp$, at which point
the handler is applied to this type $f(\overline{\beta} \Fapp)$ and the
generated constraint is scheduled to be solved. If $\alpha$ is never
unified, the constraint remains unsatisfiable.

% When solving $\cmatch \alpha \Delta f$, the solver proceeds as follows:
% \begin{enumerate}
% \item
%   If $\alpha$ is already unified to a non-variable type, apply the handler
%   $f$ immediately to the resolved type.
% \item
%   If $\alpha$ is still unresolved, enqueue the suspended constraint to the
%   list of suspended constraints blocked by $\alpha$.
% \item
%   During unification, whenever a type variable with a pending constraint
%   becomes resolved, the solver retrieves all blocked constraints on that
%   variable and schedules them.
% \end{enumerate}
% Once all constraints have been processed, any pending suspended constraints
% that remain unsolved indicate failure.

\subsection{Constraint-solving: an intuition}


\section{Applications of suspended constraints}

\begin{itemize}
\item
  type-based disambiguation of data constructors (and record fields)
\item
  polytypes (are we sure?) \\ ensuring that there is a unique polytype
  skeleton
\item
  anti-unification-based overloading (are we sure?) \\ there is some
  uniqueness of the head / variational type
\end{itemize}
\Xdidier {For me, the three examples are mandatory.
disambiguation of data constructors and record fields is too simple...
Also, polytypes may or may not include polyparams. Polyparams are a real
challenge because most constraints will be syspended, hence the need for a
strategy to resolve them. Static ad hoc overloading is more general.
}

\Xgabriel{I care most about the semantics and implementation of the
constraint language.} 
\Xdidier{It would be nice to have typing rules for the surface language, and
show that they match the constraints generated.} 
\Xalistair{I agree.}

For each feature:
\begin{itemize}
\item propose *informal* surface-language typing rules
\item elaborate them into the constraint language
\end{itemize}

We will provide *precise* declarative semantics in the ``Semantics of
constraints'' section.

\subsection{Static overloading of constuctors and record labels}

% Intro to the feature 
OCaml permits on to define data types with overloaded constructor or field
names.
\begin{lstlisting}
type m = L;; type n = L;;

let x = L;;
\end{lstlisting}
The type of \code{x} is ambiguous since OCaml could either infer \code{m} or
\code{n}. This amounts to a principality issue, since there is no most
general type for \code{x}. As such, OCaml gives the following error: ...

OCaml's type checker relies of type information provided by the
bidirectional propagation of type information to resolve these
ambiguities. Thus annotating \code{x} with either \code{m} or \code{n} would
be produce a well-typed program.

% Constructor disambiguation in suspended constraints
We propose an alternative approach: suspended constraints. For the ambiguous
constructor application $\econstr K e$, we generate the constraint:
\begin{align*}
\cinfer {\econstr K e} \alpha &= \cexists \beta \cinfer e \beta \cand \cmatch \alpha {\alpha, \beta} (\lambda \overline{\gamma} \Fapp. ~ \Let ~ \sigma = \Delta(\F)(\mathsf{K}) ~ \In ~ \sigma \leq \beta \to \alpha )
\end{align*}
where the function looks up the type scheme associated with the constructor
using the type name $\F$ in the environment ($\Delta(\F)(\mathsf{K})$).

We require a unique type for each type constructor in a given algebraic data
type definition, thus there is no ambiguity with this mapping, despite there
potentially being many constructors in the environment named $\mathsf{K}$.

This not only correctly encodes OCaml's current behaviour, but does better.


\paragraph{Interaction with let-polymorphism}

Our intuition is not entirely correct here. While the above will improve on
OCaml's current inference approach, it is not powerful enough to capture the
simple idea that to resolve an ambiguous constructor we must use contextual
information to show there is a unique type constructor $\F$ for the
application.

We demonstrate this is with the following example:
\begin{lstlisting}
let foo =
  let f x = match x with L -> 1 in 
  f (L : m)
;;
\end{lstlisting}
Our current approach would keep the type of \code{x} as a generalizable
variable, thus the application of \code{f} would not resolve this ambiguity,
requiring an annotation at the definition.

We require some way of splitting the head of the type from the type itself,
treating the head of the variable monomorphically and the rest
polymorphically.
\Xalistair{An explaination of kinded(1) types here }

\subsection{Semi-explicit first-class polymorphism}

% Polytypes (and some stuff about labelled types) 

Semi-explicit first-class polymorphism \citep{Garrigue-Remy/poly-ml} uses
\textit{annotated types} to track the origins of polymorphic types.

The type constructor $\tpoly \sigma \varepsilon$ that boxes a polymorphic
type $\sigma$ turning it into a \textit{polytype}. Each polytype is annotated
with an $\varepsilon$ variable.  Once boxed, polytypes are considered to be
monotypes. This permits impredicative polymorphism.

To instantiate a polytype, one must use an explicit unboxing operator
$\einst e$ which instantiates the top-level quantifiers of a polytype if $e$
has the type $\tpoly \sigma \varepsilon$.

Conversely, the introduction form for polytypes is a boxing operator $\epoly
e \sigma$ with an explicity polytype annotation $\sigma$ and an annotation
$\varepsilon$, has the type $\tpoly \sigma \varepsilon$ for some arbitrary 
$\varepsilon$ if $e$ has the type $\sigma$.

Principality is tracked using generalization.  This is done by tracking the
polymorphism of labels. Type schemes quantify not only over type variables
but $\varepsilon$ variables. As a result, a term $e$ with the type $\tpoly
\sigma \varepsilon$ for some fresh $\varepsilon$ may be generalized to
$\tfor \varepsilon {\tpoly \sigma \varepsilon}$.  The typing rule for
$\einst e$ in fact requires $e$'s type to be $\tfor \varepsilon {\tpoly
\sigma \varepsilon}$, not just a plain polytype $\tpoly \sigma
\varepsilon$. The rational is that a label is polymorphic if and only if it
is principally known and not inferred.  This is why the introduction form
for polytypes $\epoly e \sigma$ is able to introduce a \textit{fresh}
$\varepsilon$ variable.

Additionally, type annotations can be used to freshen labels.  This permits
us to generalize the labels of an annotated term, indicating that the type
is principally known.

\Xalistair[]{Something about how information can flow from let bindings to use sites for principality checking}

Instead of using labelled types to infer whether the type is principally
known, we may use suspended constraints.  Typechecking $\epoly e \sigma$ is
simple:
\begin{align*}
  \cinfer {\epoly e {\tfor {\overline{\beta}} \tau}} \alpha &= \left(\cfor {\overline{\beta}} \cinfer e \tau \right) \cand \alpha = [\tfor {\overline{\beta}} \tau]
\end{align*}

If $e$ is known to have the type $[\sigma]$, then we can simply instantiate
it. However, if the type of $e$ is not principally known, then we must
wait. This is how our approach differs from OCaml's / labelled types.

By waiting for $e$'s type to be known, we guarantee principal types without
the machinery of labelled types.  So we can simply write the constraint
generated by $\einst e$ as:
\begin{align*}
  \cinfer {\einst e} \alpha &= \cexists \beta \cinfer e \beta \cand \cmatch
  \beta {\alpha} (\lambda [\sigma]. ~ \sigma \leq \alpha) 
\end{align*}

\section{Semantics of constraints}


A na\"ive denotation semantics for suspended constraints would be to simply
apply the assignment of $\alpha$ to $f$. More formally,
\begin{mathpar}
  \inferrule* 
    {\phi(\alpha) = \overline{\tau} \Fapp \\ \phi_{\setminus \Delta}[\overline{\beta := \tau}] \vdash f(\overline{\beta} \Fapp)}
    {\phi \vdash \cmatch \alpha \Delta f}
\end{mathpar}

However, our informal solver would be incomplete with respect to this
semantics.  We show this by considering the constraint $\cexists \alpha
\cmatch \alpha \alpha (\lambda \_.~ \alpha = 1)$.  This constraint does not
unify $\alpha$ outside the suspended constraint, but once it is scheduled,
the variable $\alpha$ must be unified with the unit type $1$. So our
informal solver suggests that this constraint is unsatisfiable since
$\alpha$ is never unified outside the match constraint. Yet we can use the
above rule to show that this constraint is satisfiable:
\begin{mathpar}
  \inferrule*   
    {
      \inferrule* 
        {
          (\alpha \mapsto 1)(\alpha) = 1 \\
          \inferrule* 
            {(\alpha \mapsto 1)(\alpha) = 1} {\alpha \mapsto 1 \vdash \alpha
            = 1}
        }
        {\alpha \mapsto 1 \vdash \cmatch \alpha \alpha (\lambda \_. ~ \alpha
        = 1)}
    }
    {\cdot \vdash \cexists \alpha \cmatch \alpha \alpha (\lambda \_.~ \alpha = 1)}
\end{mathpar}

The dependency of suspended constraints can also be complex, as in the
following example: $\cexists {\alpha, \beta} \cmatch \alpha \beta {(\lambda
\_. ~ \beta = 1)} \cand \cmatch \beta \alpha {(\lambda \_. ~ \alpha = 1)}$.
Worse still, is that the binding location of $\alpha$ may not even contain
the sufficient information to show that $\alpha$ is realised.  For instance,
\begin{align*}
  \cexists \beta \beta = 1 \cand \cexists \alpha \alpha = \beta \cand \cmatch \alpha \Delta f 
\end{align*}

Our conclusion is that the na\"ive semantics should be rejected in favour
for the semantics of our informal solver. Our intent with suspended
constraints is that the blocking variable should not be guessed out of thin
air, but deduced from the surrounding context without knowledge of the
constraint generated by the handler.

The idea, inspired by type systems that require principality (such as
FreezeML), is to stipulate that the only satisfying assignments for the
context of the suspended constraint requires $\alpha$ to be assigned to some
type with a unique type constructor.
\Xalistair[]{This doesn't read entirely correctly, we want a unique constructor for all assignments}. 

This motivates our next definition: the principal realisation of a type
variable $\alpha$ by some constraint context $\mathcal{C}[-]$.

\begin{definition}
  A type variable $\alpha$ is principally realised with the constructor $\F$
  by the context $\mathcal{C}[-]$, written $\mathcal{C}[\alpha !] \rhd \F$,
  iff for all assignments $\phi$, there exists an assignment $\phi'$ for the
  hole such that $\phi \vdash \mathcal{C}[\phi' \vdash \ctrue]$ implies
  $\phi'(\alpha) = \overline{\tau} \Fapp$
\end{definition}

\TODO
{Here, we not only rely on the definition of a constraint with a hole
$\mathcal{C}[-]$ but a judgement with a hole and some way to translate
between the two. We'll need formal definitions for this.}

\TODO
{Some examples of principal realisation}

As we've previously hinted, the semantics for a suspended constraint must
require that the surrounding context principally realises the matchee. This
is formally specified as:
\begin{mathpar}
\inferrule* 
   {\mathcal{C}[\alpha !] \rhd \F \\
    \phi \vdash \mathcal{C}[\phi'_{\setminus \Delta}[\overline{\beta :=
      \tau}] \vdash f(\overline{\beta} \Fapp)] \\
    \phi'(\alpha) = \overline{\tau} \Fapp}  
   {\phi \vdash \mathcal{C}[\phi' \vdash \cmatch \alpha \Delta f]}
\end{mathpar}

\TODO: precise declarative semantics of constraints. They should coincide
with the semantics of the typing rules. 

Alistair has yet another proposal derived from his simplification of his
timestamp semantics.

\TODO
{derive precise declarative semantnics for the language features of
Section 2.}


Idea after discussion with Alistair:
\begin{itemize}
\item sketch a two-place satisfiability judgment $\phi \vdash C$
\item define a three-place judgment $\phi \vdash C[\phi' \vdash \square]$
\item from it we can re-derive the two-place judgment
  $$ \phi \vdash C \quad := \quad \phi \vdash (C \wedge \square)[\phi \vdash \square] $$
  and also define a four-place judgment
  \begin{mathpar}
    \infer
    {\phi \vdash \mathscr{C}[\phi' \vdash \square] \and \phi' \vdash C'}
    {\phi \vdash \mathscr{C}[\phi' \vdash C']}
  \end{mathpar}
\item then we have all the bits to introduce the condition
  $\mathscr{C}[\alpha !]$:
  \begin{mathpar}
    \begin{array}{l}
    \exists! \F, \quad \forall \phi, \phi',\\ \qquad \phi \vdash \mathscr{C}[\phi' \vdash \square] \implies \mathsf{head}(\phi'(\alpha)) = \F
    \end{array}
  \end{mathpar}
\item and finally the global rule
  \begin{mathpar}
    \infer
    {\mathscr{C}[\alpha !]
      \and \phi'(\alpha) = \overline{\tau} \Fapp
      \and \phi \vdash \mathscr{C}[\phi', \overline{\beta := \tau} \vdash f(\overline{\beta} \Fapp)] }
    {\phi \vdash \mathscr{C}[\phi' \vdash \cmatch \alpha \Delta f]}
  \end{mathpar}
\end{itemize}

The intuition for the four-place judgment $\phi \vdash \mathscr{C}[\phi' \vdash C]$ is that it is a proof of satisfiability of $\mathscr{C}[C]$ where the $C$ sub-derivation has the valuation $\phi'$. Note that, after the the global rule is added, the derivations of this judgment are not syntactically directed by $\mathscr{C}$ anymore.

\subsection{Type system}

In order to show our approach is sound and complete with respect to our
constraint generation translation and solver, we need to give a calculus and
type system to serve as a specification.

\paragraph{Syntax}

In Figure ??, we give the grammar for our calculus. It contains all of the
ML calculus, including the unit term $\eunit$, lambda abstractions $\efun x
e$, applications $\eapp \eone \etwo$, annotations $(e : \tau)$ and let
bindings $\elet x \eone \etwo$.  For constructor disambiguation, we add
constructor applications $\econstr \K e$ and their eliminators, match
expressions, $\ematch e {\overline{\ecase \Ki \evari \ei}}$.  And for
semi-explicit first-class polymorphism, we have the boxing construct $\epoly
e \sigma$ and the unboxing construct $\einst e$.

\paragraph{Typing}




\begin{mathpar}
  \inferrule*   
    {x : \sigma \in \Gamma}
    {\Gamma \vdash x : \sigma}

  \inferrule* 
    {\Gamma \vdash \tauone \\ \Gamma, x : \tauone \vdash e : \tautwo }
    {\Gamma \vdash \efun x e : \tauone \to \tautwo}

  \inferrule* 
    {\Gamma \vdash \eone : \tauone \to \tautwo \\ \Gamma \vdash \etwo : \tauone}
    {\Gamma \vdash \eapp \eone \etwo : \tautwo}
\end{mathpar}

\section{Solving constraints}

\TODO
{a declarative solver.}

\section{HM Prototype and discussion}

\subsection{Default clauses}

\section{Semi-unification Prototype}

Rank-2 polymorphic type inference with semi-unification!

\Xdidier{When you do rank-2 unification, you don't need a 'let' anymore
because 'lambda' is enough. You have rank-2 types but you cannot bind
them. It's not as exciting as I thought it would be.} 

\section{Comparison with other inference approaches}

\subsection{Suspended constraints in \textsc{OutsideIn}}

\Xalistair{They first solve simple constraints (existentials,
unification). Then they solve ``... constraints'', with implication
constraints, which corresponds to the case of GADT matches. Crucially, they
abandon local let generalization.} 

\subsection{Suspended constraints in dependent-type systems}


\subsection{Bi-directional type inference}

\Xgabriel{Our approach should have better properties for disambiguation, but
which ones?} 

In the simply-typed case our system works better. But with generalization
you can have more issues.

\Xdidier{Should we try to extend what we did to predicative polymorphism?}

\Xalistair{Not ready yet and would probably need too much space.}

\Xgabriel{CoreML + bidirectional disambiguation of constructors?}

\TODO
{do we understand what to say precisely about bidirectional type inference?}

\subsection{Principality tracking in OCaml}

\Xdidier{The point of that is to enforce a directional type inference that
is based on let-bindings. When we check that the level is generic, we check
that we already ``closed'' this thing, it is an earlier 'let'
binding. Because we are omni-directional, we don't have principality issues
anymore -- except with default rules. We are principal by construction, we
never make any choice.} 

\Xdidier{Principality tracking is making the choice that we are going to
make a directional let-binding-based type inference. We get rid of that, we
are omni-directional, and don't have any principality issues.} 

\Xgabriel{We should not claim too much if we don't understand default
clauses well enough. It could people the impression that we hide the issue
under the carpet.} 

\Xdidier{First a declarative/principal system, and then non-principal
heuristics to refine it, a two-phase process. It's fine.} 


% \begin{acks}
% \end{acks}


%% \bibliographystyle{ACM-Reference-Format}
\bibliography{suspended}

\end{document}

% LocalWords:  omnidirectional typecheck polymorphism Hindley Milner kinded
% LocalWords:  GADTs typechecked codomain typechecking subexpressions Bodin
% LocalWords:  monomorphic subexpression Dunfield Riboulet jfla subtyping
% LocalWords:  greek Chargueraud
