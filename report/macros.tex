% Restating theorems for the purpose of the proofs
\usepackage{thmtools}
\usepackage{thm-restate}

%% Work-in-progress comments
\ifcomments
\newcommand{\todo}[1]{{\color{red} {\textsc{TODO:}} #1}}
\newcommand{\draft}[1]{{\color{black!50} #1}}
\newcommand{\outdated}[1]{{\color{black!50} {\textsc{OUTDATED:}} #1}}
\newcommand{\jrc}[1]{{\color{red} {\textsc{JRC:}} #1}}
\newcommand{\fe}[1]{{\color{red}  {\textsc{FE:}}  #1}}
\newcommand{\sam}[1]{{\color{red} {\textsc{SL:}}  #1}}
\newcommand{\js}[1]{{\color{red}  {\textsc{JS:}}  #1}}
\else
%% Use for the final version to make sure no comments are left in the paper
\newcommand{\todo}[1]{}
\newcommand{\draft}[1]{}
\newcommand{\outdated}[1]{}
\newcommand{\jrc}[1]{}
\newcommand{\fe}[1]{}
\newcommand{\sam}[1]{}
\newcommand{\js}[1]{}
\fi

%% Easy way to redefine inline formatting of code
\newcommand{\code}[1]{\lstinline{#1}}

\declaretheorem[name=Theorem]{thm}
\declaretheorem[name=Definition]{defn}
\declaretheorem[name=Proposition]{prop}
\declaretheorem[name=Lemma,sibling=thm]{lem}
\declaretheorem[name=Remark]{remark}
